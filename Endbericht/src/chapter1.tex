\section{Einleitung}
<<<<<<< HEAD
=======
In diesem Kapitel wird ein einführender Überblick über die Projektgruppe Fully Autonomous Intralogistic Swarm Experiments gegeben, die im Rahmen der Masterstudiengänge Informatik und Wirtschaftsinformatik in der Abteilung Systemanalyse und -optimierung der Carl von Ossietzky Universität Oldenburg stattgefunden hat. Das Projekt lief über einen Zeitraum von zwei Semestern: Wintersemester 2013/2014 und Sommersemester 2014.


\subsection{Motivation}
Im Zeitalter der Globalisierung werden hohe Anforderungen an die Leistungsfähigkeit von modernen Intralogistiksystemen gestellt. Neben einem hohen Automatisierungsgrad wird gleichzeitig auch eine möglichst hohe Flexibilität gefordert, da sich Anforderungen im logistischen Umfeld häufig ändern (Vgl.\cite{ieft}).    
\\\\
Stetigförderer bieten die Möglichkeit einen automatisierten Materialfluss einzurichten. Es handelt sich dabei um Transportsysteme, die Güter kontinuierlich und automatisiert entlang eines festgelegten Transportwegs befördern (Vgl.\cite{stf}). Ein solches System könnte beispielsweise ein Netz von Schienen sein. Nachteile dieser Systeme sind insbesondere Unflexibilität und schlechte Skalierbarkeit. Ändern sich Anforderungen in einem Logistiksystem, dann stoßen Stetigförderer schnell an ihre Grenzen. Transportwege sind festgelegt und können nicht ohne einen gewissen Aufwand geändert werden. Auch kann die Anzahl an Gütern, die pro Zeiteinheit befördert werden kann, nicht ohne eine Änderung am Transportnetz maximiert werden.
\\\\
Eine Alternative zu Stetigförderern sind Fahrerlose Transportsysteme (FTS). FTS sind ein Gesamtsystem aus Fahrerlosen Transportfahrzeugen, die Ware automatisiert befördern, und der Infrastruktur, die zum Betrieb der Transporteinheiten notwendig ist (Vgl.\cite{fts}). Fahrerlose Transportsysteme sind wesentlich flexibler als Stetigförderer. Müssen mehr Güter befördert werden, so können zusätzliche Transporteinheiten aktiviert werden. Folglich sind FTS problemlos skalierbar und können schnell auf veränderte Anforderungen in einem Intralogistiksystem eingestellt werden. FTS bieten einen automatisierten Warenfluss bei gleichzeitig hoher Flexibilität und entsprechen somit den Anforderungen, die an moderne Intralogistiksysteme gestellt werden.  
\\\\
Es bietet sich an ein System zu entwickeln, das basierend auf FTS, einen vollautomatisierten Warenfluss implementiert, um verschiedene Fragestellungen zu untersuchen. Wie muss ein solches System aufgebaut sein, welche Kommunikationsabläufe sind zwischen den verschiedenen Akteuren notwendig, welche Anforderungen werden an Hard- und Software gestellt und wie flexibel ist ein solches System?  

\subsection{Zielsetzung}
Im Rahmen der Projektgruppe FAISE soll ein System entwickelt werden, das den vollautomatisierten Warenfluss in einem Lager auf Basis von Fahrerlosen Transportsystemen simuliert. Dabei sollen die Transporteinheiten nicht zentral gesteuert werden, sondern dezentral als Schwarm agieren.
Das Gesamtsystem besteht aus zwei Teilsystemen, einem physisch vorhandenem System und einer softwarebasierten Simulation. 
\\\\
Das physische System beinhaltet Fahrerlose Transporteinheiten und Lagerrampen, die miteinander über ein Sensornetzwerk kommunizieren und deren Steuerung auf Basis von Mikrocontrollern erfolgt. Ziel ist es den Materialfluss von den Transporteinheiten und Rampen vollständig autonom und ohne dezentrale Steuerung durchzuführen. 
\\\\
Das rein softwarebasierte System implementiert ebenfalls einen automatisierten Warenfluss. Die Software soll als Abbild des physischen Systems realisiert werden. Die Akteure, ihre physikalischen Eigenschaften (Geschwindigkeit etc.) und ihr Verhalten im Einzelnen sowie als Schwarm sollen in der Software abgebildet werden. Beide Systeme laufen unabhängig voneinander und sollen in einem festen Einsatzszenario erprobt werden.

\subsection{Einsatzszenario}
Das Einsatzszenario besteht aus n fahrerlosen Transporteinheiten in einem Umschlagslager. Zusätzlich sind m Rampen verfügbar an denen Pakete zwischengelagert werden können. Im Gegensatz zu einem herkömmlichen Lager, werden Waren in einem Umschlagslager nur kurzfristig gelagert, um anschließend weitertransportiert zu werden. Es herrscht ein kontinuierlicher Materialfluss. Jedes Paket, das ins Lager gebracht wird, ist eindeutig identifizierbar und wird zu einem definierten Zeitpunkt ins Lager gebracht und wieder abgeholt. Die Rampen im Lager sollen drei unterschiedliche Zwecke erfüllen. Eingangsrampen dienen der Warenannahme, Zwischenrampen der Zwischenlagerung. Pakete werden zum Ausgangslager gebracht und zum Zwecke des Weitertransports dort abgeholt. Auf Basis des Einsatzszenarios wird im Rahmen der Anforderungen ein Ablaufszenario erstellt, das die Interaktionen zwischen den Akteuren beschreibt auf deren Basis eine automatisierte, dezentrale Abwicklung des Materialflusses erfolgen kann.

\subsection{Komponenten}
  

>>>>>>> 8effb45eed680862593a1536e98c8063b0896e9a

\subsubsection{Pathfinding des Volksbot und in der Simulation}

Die Integration der physischen Zelle und der Simulation erfordert die Anpassung der Pathfinding-Algorithmen. Für die Realisierung eines hybriden Modus gilt es den Dijkstra-Algorithmus der Volksbots möglichst genau in der Simulation abzubilden. Dazu gehört unter anderem die Aufteilung der Umgebungskarte durch ein Gitternetz gleicher Größe. Wie zuvor geschildert kann durch die Übergabe der aktuellen Position der Volksbots an die Simulation eine Synchronisation zwischen den beiden Systemen verwirklicht werden.

Probleme des physischen Systems, wie z.B. die ungenaue Selbstlokalisierung durch fehlerhafte Odometrie und die daraus folgende Verbesserung mittels Laserscan-Daten, können nicht in der Simulationen realisiert werden, wodurch Unterschiede im Verhalten der Roboter in der physischen Zelle und der Simulation bestehen bleiben. 

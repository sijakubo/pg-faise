\subsubsection{Ablauf}

Im folgenden wird der Ablauf für einen Transport innerhalb des physikalischen Aufbau aus Abbildung \ref{fig:physischeZelle} vom Ausgang der Rampe A zum Eingang der Rampe B beschrieben. Dafür muss zunächst ein Paket in das System initialisiert und dadurch ein Auftrag generiert werden. Anschließend kann durch den Materialfluss der Transport ausgelöst werden.

\begin{itemize}
\item Volksbot wird lokalisiert und wartet auf Auftrag
\item Registrierung der Pakete im System (Gateway)
\item Materialfluss sendet Nachricht (Opcode mit Ziel-ID: Ausgang Rampe A) an Volksbot
\item Volksbot entschlüsselt Opcode und startet Navigation zu Rampe A
\item Hubposition am Volksbot wird auf Höhe Ausgang Rampe A gefahren
\item Wenn Volksbot Navigationsziel erreicht hat und Hubhöhe eingestellt ist, wird eine Statusrückmeldung an Materialfluss gesendet
\item Rampe A gibt das Paket ab und der Volksbot nimmt durch Ansteuerung der Floweinheit das Paket auf
\item Volksbot fährt zum Eingang Rampe B und regelt die Hubhöhe auf entsprechende Rampenhöhe
\item Volksbot gibt Statusrückmeldung an Materialfluss und löst Paketübergabe aus
\item Volksbot frei für nächsten Auftrag


\end{itemize}

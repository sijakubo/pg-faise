\subsection{Integration von Simulation und physischer Zelle}
\label{sec:Hybrid}
Bezogen auf das Gesamtsystem bietet es sich an, das physische und das virtuelle Teilsystem mithilfe eines hybriden Modus miteinander zu verknüpfen. Ein solcher Hybridmodus bietet viele weitere Möglichkeiten und könnte auf unterschiedlichen Wegen realisiert werden. Beispielsweise könnte die Software das physische System steuern, indem generierte Aufträge an das reale System geschickt werden. Auch könnten die Aktionen der Fahrzeuge und Rampen in der Software visualisiert werden. Besteht ein Teil der Visualisierung aus der Darstellung der realen Akteure und der andere aus rein virtuellen Akteuren, so könnten physische und virtuelle Akteure ein Gesamtsystem bilden, dass die Skalierbarkeit des physischen Systems erhöht. Dies setzt jedoch voraus, dass die Akteure aus beiden Systemen in ihren Eigenschaften (Geschwindigkeit etc.) weitestgehend aneinander angeglichen sind. Die technischen Voraussetzungen für das einbringen und abhören von drahtlosen Nachrichten in der physischen Zelle sind bereits durch Gateways realisiert. Die einzelnen physischen Module (Volksbots, Rampen) müssten jedoch zusätzlich regelmäßig über ihren Zustand informieren, um eine sinnvolle Auswertung der physischen Zelle in der Simulationsoberfläche zu erlauben. Gleichzeitig muss in der Simulation eine Möglichkeit geschaffen werden, physische Module hinzuzufügen, die nicht vom System simuliert, sondern stattdessen über Agentennachrichten angesprochen werden und Auskunft über ihren Zustand geben.
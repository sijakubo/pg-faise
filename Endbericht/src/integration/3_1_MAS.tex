\subsubsection{Multiagentensystem und dessen Laufzeitumgebung}
Sowohl im Teilsystem Simulation als auch im Materialfluss kommen Agenten zum Einsatz, die die Betriebslogik abbilden. 

Da beide jedoch auf sehr unterschiedlichen Plattformen (Webserver beziehungsweise verteilte Mikrocontroller) zum Einsatz kommen, kann kein einheitliches Multiagentensystem zum Einsatz kommen. Stattdessen wird in der Simulation JADE verwendet, während im Materialfluss ein eigens entwickeltes AgentRTE zum Einsatz kommt, das speziell für den Einsatz auf Mikrocontrollern angepasst wurde. Bedingt durch die zur Verfügung stehenden Ressourcen ist dieses System im Bezug auf die maximale Kommunikationsbandbreite, die maximale Anzahl an Agenten und die Anzahl der gepufferten Nachrichten deutlich eingeschränkter. Auch erlaubt es im Gegensatz zu JADE keine \textit{Behaviors}, sondern das Verhalten der Agenten muss prozedural oder mit Zustandsautomaten abgebildet werden.

Die ausgetauschten Agenten-Nachrichten hingegen sind in beiden Systemen zumindest an den FIPA-Standard angelehnt, eine Umwandlung von einem in das andere Format kann daher durch eine einfache Abbildung realisiert werden. Weiterhin ist im physischen System die genaue Reihenfolge der Nachrichten-Parameter von großer Bedeutung, da hier nicht mit Objekten, sondern mit Byte-Arrays gearbeitet wird, die entsprechend interpretiert werden.

Das Multiagentensystem, dass für die Simulationssoftware entwickelt wurde, enthält neben den vier "Standardagenten" noch einen Job- und Statistikagent (Vgl. Abschnitt 5.4.5). Der Statistikagent hat keinen Einfluss auf den Ablauf der Simulation, der Jobagent hingegen übernimmt die Verteilung von Paketen und ausgehenden Aufträgen. Bei der Integration beider Teilsysteme muss der Jobagent miteinbezogen werden, sofern Pakete vom physischen System zu den Eingängen des virtuellen Systems  gelangen. Ist dies der Fall, so muss der Jobagent in die Datenübertragung miteinbezogen werden, da Eingänge keine eigenen Mechanismen zur Paketannahme besitzen. werden lediglich Pakete aus der Simulation zum physischen System befördert, so muss der Jobagent nicht miteinbezogen werden. Jedoch gibt es noch keinen Mechanismus, um Pakete, die an Ausgangsrampen eintreffen, aus der Simulation zu entfernen. Bei einer Integration beider Systeme über die Ausgangsrampen, könnte ein solcher Mechanismus im Abgleich mit dem physischen System entwickelt werden. 
\\\\
Physisches System und virtuelles System unterscheiden sich hinsichtlich des Paketagenten. In der Simulation wurde ein Paketagent erstellt, der alle Pakete verwaltet anstatt eines Paketagenten für jedes Paket. Grund dafür war, dass die Vielzahl an Agenten, die auf Basis von Jade und alle auf derselben Maschine liefen, zu Performance- und Synchronisationsproblemen führten. Eine weitere Aufsplittung des Paketagenten hätte diese Probleme noch verschärft. Für die Integration der beiden Teilsysteme muss untersucht werden, welche Aufgaben des Orderagenten der Paketagent in der Simulation übernimmt. Damit die Kommunikation zwischen zwei Rampen aus jeweils einem Teilsystem funktioniert, müssen die Kommunikationsprozesse zwischen den Agenten ggf. angepasst werden. Es ist unter Umständen möglich, dass beispielsweise die Zielfindung zwischen zwei Rampen im physischen Teilsystem andere Agenten einbezieht als die Zielfindung zwischen zwei Rampen aus beiden Teilsystemen.
\section{Teilbericht Fahrzeuge}
Ziel der Teilgruppe Fahrzeuge ist es einen autonomen Transport zwischen den Rampen zu realisieren. Dafür sollen die Volksbots in der Lage sein...
\subsection{Anforderungen}
(siehe Materialfluss)
\subsubsection{Funktionale Anforderungen}
\subsubsection{Nicht Funktionale Anforderungen}

\subsection{Beschreibung der Komponenten}
\subsubsection{Volksbot}
\begin{itemize}
\item Bild Volksbot
\item Hub/ Flow
\item Fahreinheit
\end{itemize}
\subsection{Werkzeuge}
Beschreibung ROS. 
Wichtige Funktionen in ROS:
\begin{itemize}
\item Publish
\item Subscribe
\item ...
\end{itemize}
\subsubsection{Robot Operating System (ROS)}
Die Hauptelemente des Projekts FAISE sind die Fahrzeuge (Volksbots). Die Volksbots sind
mit einem „Robot Operating System“ gerüstet, der unter Linux- Betriebssystem läuft.
\subsubsection{Was ist ROS}
Es gibt viele Robotik-Frameworks die spezifisch für präzise Anwendungen, für Prototypen,
erstellt wurden. ROS strebt da eher das Allgemeine an. Das »Robot Operating System«
(ROS) ist ein Open-Source Framework für individuelle Roboter, das sich in der
Robotikforschung in den letzten Jahren etabliert hat und ein großes Repertoire an Software-
Komponenten und -Werkzeugen für Robotikapplikationen bietet.\\
Die Entwicklung begann 2007 am Stanford Artificial Intelligence Laboratory im Rahmen des
Stanford-AI-Robot-Projektes (STAIR). Heute wird es hauptsächlich am Robotik institut Willow
Garage weiterentwickelt. Seit April 2012 wird ROS von der neu gegründeten,
gemeinnützigen Organisation Open Source Robotics Foundation (OSRF) unterstützt. Die
Bibliotheken von ROS setzen auf Betriebssysteme wie Linux, Mac OS X oder Windows auf.
ROS ist nicht von einer spezifischen Sprache abhängig. Heutzutage gibt es 3 Grundlibraries
für ROS, die jeweils auf Python, Lisp und C++ ausgerichtet sind. Zwei Exmperimentier-
Librairies sind für Java und Lua erhältlich.
\subsubsection{Was will ROS?}
\begin{itemize}
 \item ROS will unterstützen, Code für Forschung und Entwicklung wiederzuverwenden
 \item loser Verbund von individuellen Programmteilen (Nodes)
 \item einzelne Programmteile können einfach geteilt und verbreitet werden (Packages und Stacks)
 \item ROS stellt Repositories zu Verfügung, um dort Code zu teilen \cite{ROS:2014:Online}
(http://www.ros.org/browse)
\end{itemize}
\subsubsection{Was kann ROS?}
Die Hauptbestandteile und Hauptaufgaben von ROS sind Hardwareabstraktion; Gerätetreiber; Implementierung 
von viel genutzten Funktionalitäten; Inter-Prozess-Kommunikation; Paket-Management
\subsubsection{Aufgaben des ROS}
\begin{itemize}
 \item Interprozesskommunikation (IPC)
 \begin{itemize}
\item Problematik der Kommunikation zwischen verschiedenen Systemen des Roboters
\item Sicherheitseinstellung bei der Übertragung
\item Anforderung an die Geschwindigkeit / Schnelligkeit der Kommunikation
\item Koordination von Nachrichten durch zentralen Master
\end{itemize}
\item Paketverwaltung – Packages
\begin{itemize}
 \item ROS ist durch Softwarepakete (sogn. Packages) aufgebaut
 \item Ein Package beinhaltet Laufzeitprozesse (Nodes); ROS abhängige Bibliotheken;
Datensätze; Konfigurationsdateien;3rd Party Software
 \item Packages sind dazu, da um Code wiederverwendbar zu machen
\end{itemize}
\item Paketverwaltung – Stacks
\begin{itemize}
\item Sammlung von Paketen (Packages)
\item Der Sinn ist, dass Stacks die Verteilung und Verwendbarkeit von Code
vereinfachen
\item Meist viele Packages ähnlicher Aufgaben in einem Stack verpackt
\end{itemize}
\item Message (msg)
\begin{itemize}
 \item  Messages werden verwendet um unter ROS Nachrichten zwischen Knoten und
Topics auszutuaschen
\item Dafür verwendet ROS eine einfache Beschreibung der Datentypen in Textdateien
\item Durch diese Beschreibung kann für unterschiedliche Sprachen Code autogeneriert
werden
\item Diese sind in .msg-Dateien im msg- Unterverzeichnis eines ROS-Pakets abgelegt
\item Eigene Message-Typen sind mit Paket Ressource-Namen bezeichnet
\item Standard Messages sind mit std\_msg/msg/String.msg bezeichnet
\end{itemize}
\item Service
\begin{itemize}
 \item ROS verwendet eine eigene vereinfachte Service Description Language ("srv") für die
Beschreibung von ROS Service-Typen
\item Setzt direkt auf die ROS msg-Format auf
\item Ermöglicht die Anfrage / Antwort-Kommunikation zwischen den Knoten
\item Service-Beschreibungen sind in .srv-Dateien im srv- Unterverzeichnis eines Pakets
gespeichert
\item Service-Beschreibungen werden für die Verwendung mit dem Paket Ressource-
Namen bezeichnet
\item Z. B.: wird die Datei robot\_srvs/srv/SetJointCmd.srv als Service
robot\_srvs/SetJointCmd bezeichnet
\end{itemize}
\item Notes
\begin{itemize}
 \item Der Nachrichtenaustausch findet bei Nodes durch 3 Möglichkeiten statt: Parameter
Server;Topics; Services
\item Nodes werden wie in einem Graph angeordnet
\item In einem System laufen viele Nodes Parallel
\item Diese werden zu Beginn gestartet
\item Beispiele sind Nodes für: Laserscanner; Kinect; Pfadplanung
\end{itemize}
\item Topica
\begin{itemize}
 \item Topics verhalten sich wie ein virtuelles BUS-System Nodes können von Topics lesen
(subscribe)
\item Nodes können an Topics senden (publish)
\item Es gibt keine Begrenzung wie viele Nodes publsih oder subscribe auf ein Topic
machen
\end{itemize}
\end{itemize}
\subsubsection{ROS-Datensystem}
ROS-Ressourcen sind in rangmäßiger Gliederung eingeordnet. Zwei Konzepte sind zu
verstehen:
\begin{itemize}
\item \textbf{Le package}: Es handelt sich hier um die Zentraleinheit der Softwareorganisation von
ROS. Ein Package ist ein Verzeichnis der die Knoten beinhaltet (wir werden hier
unten erklären, was ein Knoten ist) sowie die externen Librairies, Daten und XML
Konfigurationsdateien die manifest.xml genannt wird.
\item \textbf{Stack}: Stack bezeichnet eine Sammlung von Packagen. Sie ermöglicht mehrere
Funktionen wie Navigation, Lokalisierung und viele mehr. Ein Stack beinhaltet
mehrere Verzeichnisse sowie eine Konfigurationsdatei die stack.xml genannt wird.
\begin{itemize}
\item Vorhandene wichtige Stacks
\begin{itemize}
\item TF – Koordinatentransformation
\item Navigationstack
\item URDF - Modelle
\end{itemize}
\begin{itemize}
\item Beispiel Navigationstack
\begin{itemize}
\item Wertet Sensordaten aus z.B.: Laserdaten
\item Baut daraus mit gmapping (ebenfalls ein ROS-Stack) eine Begehbarkeitskarte
\item Warum? Zur Kollisionsvermeidung
\item Bei erfolgreicher Erstellung einer Map kann dann ein Ziel übergeben werde (Pfadplanung durch Navigationstack,
Kollisionsvermeidung, Reaktion auf sich ändernde Umgebung, Aufbau einer globalen Karte)
\end{itemize}
\end{itemize}
\end{itemize}
\end{itemize}
\subsubsection{Vorteile und Nachteile des ROS}
\paragraph{Vorteile}
\begin{itemize}
 \item Nachrichten-basierte Software Architektur
\begin{itemize}
\item Verschiedene Komponenten sind unabhängig voneinander mit dem System verbunden
\item Unterschiedliche Komponenten können miteinander verbunden werden, ohne jedes Mal das Programm neu zu Kompilieren
\item Netzwerkfähigkeit
\item Einfaches Debugging und Simulieren
\end{itemize}
\item Absturz eines Nodes führt nicht zum Absturz des ganzen
Systems
\item Für ROS lässt sich in mehreren Sprachen programmieren
\item ROS hat eine große Community, die viele Daten und Programme zu Verfügung
stellen
\end{itemize}
\paragraph{Nachteile}
\begin{itemize}
 \item Durch Nachrichten-basierte Systemarchitektur Bottleneck bei großer Datenmenge
\item Steuerung des Systems über Kommandozeile
\end{itemize}

\subsection{Architektur}
Volksbot: ROS, MAXON, MICAZ
\subsection{Implementierung}
Einführung in den Quellcode
\begin{itemize}
\item Was machen die Dateien? 
\item Diagramm Michael
\end{itemize}
Wegfindung/ Navigation
\begin{itemize}
\item Odemetrie, Laserscan
\end{itemize}
Hub, Flow
Kommunikation Volksbot, Micaz, Materialfluss

\subsection{Erreichte Funktionalitäten}
\subsection{Herausforderungen}
\subsection{Ausblick}



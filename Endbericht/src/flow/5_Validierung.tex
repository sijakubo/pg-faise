\subsection{Probleme und Herausforderungen}

Während der Projektdurchführung sind wir auf einige Probleme und Herausforderung gestoßen. Auf diese soll in diesem Abschnitt noch einmal ein besonderes Augenmerk gelegt werden. 

In der Produktbeschreibung der \textsc{Mica}z-Module wurde eine Funkreichweite von bis zu 100 Metern beworben. Beim Testen der Funkfunktionalität ist aber sehr schnell deutlich geworden, dass dies mit den Modulen nicht oder nur unter optimalen Laborbedingungen möglich sein würde. Es wurden lediglich Entfernung von zwei bis drei Metern erreicht, wenn die Module während des Funkvorganges im Raum bewegt wurden. Bei starren Positionen wurden Entfernungen von bis zu 5 Metern auf freier Fläche erreicht. Durch Recherche in Datenblättern stellte sich heraus, dass der Flaschenhals der Funkübertragung sehr wahrscheinlich in den mitgelieferten Antennen zu finden sei. Es wurde deshalb neue Antennen mit einem Gewinn von bis zu acht dBi an die \textsc{Mica}z-Module angeschlossen. Außerdem wurde ein selbstentwickeltes Networkflooding-Protokoll implementiert. Bei diesem Verfahren wird jede Nachricht durch ihre Sender-ID und eine Message-ID identifiziert und außerdem wird eine Time-To-Live (kurz TTL) Information angefügt. Wenn ein Empfänger eine ihm noch unbekannte Nachricht empfängt, überprüft er, ob er selbst das beziehungsweise ein Ziel der Nachricht ist. Wenn ja, verarbeitet er die Nachricht. Ansonsten wird die TTL dekrementiert und die Nachricht nochmals an alle erreichbaren Empfänger gesendet.
Diese beiden Maßnahmen erlaubten auch in der Praxis eine sehr stabile Funkverbindung zwischen allen Modulen, die gleichzeitig nicht durch zu viel Overhead beeinträchtigt wurde.

Eine weitere Herausforderung lag im geringen Arbeitsspeicher der Module. Dieser liegt bei lediglich zwei Kilobyte, die allein vom Betriebssystem und dem Kommunikationsstack schon zu etwa 50\% genutzt werden. Dies führte dazu, dass komplexe Routing-Protokolle nicht durchführbar sind. Die Herausforderung wurde dadurch gemeistert, dass ein nachrichtenbasiertes Routing durchgeführt wird, bei dem nahezu alle relevanten Informationen in den einzelnen Agenten-Nachrichten und im Zustand des zugrunde liegenden Automaten gespeichert werden können. 

Die dritte Herausforderung, die hier kurz beschrieben werden soll ist das Debugging in verteilten Hardware-Systemen. Eine genaue Programmabfolge kann in solchen Systemen nur schwer beobachtet werden und Fehler im Ablauf treten durch die nicht-synchronisierten Systeme häufig nicht-deterministisch auf.
Eine große Hilfe war hier der JTAGICE3 von Atmel, der es durch das Setzen von Haltepunkten im Programmcode ermöglicht, Zustände von Variablen und Ausgangspins direkt im laufenden System zu überprüfen,zu analysieren und gar zu verändern.



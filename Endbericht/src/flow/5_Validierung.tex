\subsection{Validierung}
Das folgende Kapitel soll einen kurzen Überblick über den Erfolg der Teilgruppe Materialfluss geben. Dazu wird zu Anfang kurz die erreichten Funktionalitäten aufgezeigt. Im Anschluss daran sollen die größten Probleme und Herausforderungen, die während der Projektarbeit aufgetreten sind, aufgezeigt. Zum Schluss des Kapitels wird ein Ausblick gegeben, in dem beschrieben wird, welche Funktionalitäten durch spätere Projektarbeiten eingebracht werden können.

Im folgendem wird mit der erreichten Funktionalität begonnen.

\subsubsection{Erreichte Funktionalität}
Während der Projektarbeit wurde ein Lauffähiges Teilsystem im Bereich Materialfluss fertig gestellt. Es wird gewährleistet, dass ein am Eingang liegendes Paket komplett durch das System bis zum Ausgang geroutet werden kann. Dazu gehört zum einen die Funktionalität, dass die Module alle untereinander über ihre Funkschnittstelle kommunizieren können. Es wurde außerdem die Kommunikation zwischen den \textsc{Mica}z-Modulen und Volksbots über ein definiertes Kommunikaitonsprotokoll erfolgreich umgesetzt. Ein weiteres Merkmal in der Funktionalität ist das gelungene Zusammenspiel zwischen Rampen und Modulen. Dadurch wird es einerseits erlaubt die Bolzen zu öffnen und zu schließen. Außerdem wird die Überwachung der Rampen durch die angebrachten Lichtschranken so ermöglicht.

Eine weitere erfolgreich umgesetzte Funktionalität, ist die Möglichkeit Pakete komplett durch das System routen zu lassen. Dabei wird eine Kommunikation der Agenten Plattformübergreifend eingesetzt. Zusätzlich wird ein funktionierendes Agenten-System auf den unterschiedlichen Modulen umgesetzt. Es ermöglicht die Überwachung und Lokalisierung einzelner Pakete, die sich im System befinden.

\subsubsection{Probleme und Herausforderungen}
Während der Projektdurchführung sind ein paar Probleme und gleichzeitig auch Herausforderung aufgetreten. Auf diese soll in diesem Abschnitt nocheinmal ein besonderes Augenmerk gelegt werden. Bei der Produktbeschreibung der \textsc{Mica}z-Module wurde geschrieben, dass das Funken über eine Reichweite von bis zu 100 Metern möglich ist. Beim testen der Funkfunktionalität ist aber sehr schnell aufgefallen, dass dieses mit den gelieferten Modulen ohne Modefizierung nicht möglich ist. Es wurden lediglich Entfernung von zwei bis drei Metern erreicht, wenn die Module während des Funkvorganges im Raum bewegt wurden. Bei starren Positionen wurden Entfernungen von bis zu 15 Metern auf freier Fläche erreicht. Da die Module aber zum Teil in dem Projekt an den Volksbots angebracht wurden und diese sich im Raum bewegen wurde schnell klar, dass die Durchführung so nicht möglich ist. Durch Recherche in Datenblättern zu der mitgelieferten Antenne, stellte sich heraus, dass der Flaschenhals der Funkübertragung sehr wahrscheinlich hier liegt. Es wurde deshalb neue Antennen mit einem Gewinn von bis zu acht dBi an die \textsc{Mica}z-Module angeschlossen. Außerdem wurde ein selbstgeschriebenes Übertragungsverfahren implementiert. Bei diesem Verfahren wird jeder Nachricht eine Time To Live (kurz TTL) angefügt. Wenn ein Empfänger eine ihm noch unbekannte Nachricht Empfängt, überprüft er das Ziel der Nachricht ist. Wenn ja, verarbeitet er die Nachricht. Ansonsten setzt er die TTL um einen herunter und schickt diese wieder an alle erreichbaren Empfänger wieder raus.

Eine weitere Herausforderung war der geringe Arbeitsspeicher der Module. Dieser liegt bei wenigen KB. Dies führte dazu, dass komplexe Routing-Protokolle nicht durchführbar sind. Die Herausforderung wurde dadurch gemeistert, dass ein vereinfachtes Routing durchgeführt wird. 

Die dritte Herausforderung, die hier kurz beschrieben werden soll ist das Debugging in verteilten Systemen. Dies stellt ein fast unmöglich zu lösendes Problem dar. Eine genaue Programmabfolge kann in solchen Systemen nicht genau definiert werden. Durch setzen von Haltepunkten im System, wird es aber ermöglicht, Zustände von Variablen zu überprüfen und zu analysieren.

\subsubsection{Ausblick}
Im folgendem Abschnitt soll ein kurzer Überblick darüber gegeben werden, durch was das Projekt in späteren Projektarbeiten erweitert werden kann. Hierzu zählt unter anderem das Zwischenlager-Rampen im Routing eine besondere Rolle spielen können. Dadurch kann zum Beispiel überlegt werden, ob es kostengünstiger ist, ein Paket von Punkt A nach Punkt B über eine Zwischenrampe zu liefern oder nicht.

Außerdem ist ein Punkt für spätere Arbeiten das Einführen eines erweiterten Routing. Darunter ist zu verstehen, dass Zeitslots und Reservierungen auf Plattformen berücksichtigit werden. Außerdem wird im Moment beim Routing so vorgegangen, dass wenn ein Paket ein Ziel A hat, dann wird es erst nach A gebracht. Dies geschieht auch dann, wenn sich das Ziel auf dem Weg dorthin ändert. Das erweiterte Routing sollte also auch die dynamische Zieländerung berücksichtigen.

Noch ein weiterer Punkt für spätere Arbeiten ist die Vorbereitung für einen Hybriden-Modus mit der Teilgruppe Simulation. Für diesen Punkt müsste ein neues Protokoll definiert, abgestimmt und eingebunden werden. Außerdem müsste eine Schnittstelle über die kommuniziert werden soll definiert werden.

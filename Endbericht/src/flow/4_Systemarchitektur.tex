\subsection{Systemarchitektur}

Das Materialflusssystem auf den \textsc{Mica}z-Modulen ist in einer Schichtenarchitektur aufgebaut. Diese ist an den Aufbau von AUTOSAR \cite{AUTOSAR:2014:Online} angelehnt. Abbildungen \ref{fig:architecture_ramp} und \ref{fig:architecture_vb} zeigen den Aufbau des Systems auf den Rampen beziehungsweise Volksbots.

\begin{figure}[h!]
 \centering
		\includegraphics[width=1\textwidth]{flow/Architektur_Rampe.png}
	\caption{Architektur der Rampe \cite{Stasch:Hahn}}
	\label{fig:architecture_ramp}
\end{figure}

\begin{figure}[h!]
 \centering
		\includegraphics[width=1\textwidth]{flow/Architektur_VB.png}
	\caption{Architektur der Volksbots aus Sicht des Materialfluss \cite{Stasch:Hahn}}
	\label{fig:architecture_vb}
\end{figure}

Ganz unten in der Hierarchie befindet sich die eigentliche \textsc{Mica}z Hardware (Hardware Level) mit allen Peripherie-Komponenten. Diese wird vom den darüber liegenden Background Level angesteuert. Im Backgrund Level befinden sich im je nach Modul hardwareabhängige Treiber für drahtlose und serielle Kommunikation, Lichtschranken, Bolzen und einen externen Flash-Speicher. Diese agieren meist auf Pin-Ebene, steuern also die einzelnen GPIOs des Mikrocontrollers. Eine Besonderheit stellt hier der Radio-Driver dar, der nicht direkt auf die Hardware, sondern auf den Kommunikationsstack des Echtzeitbetriebssystems Contiki zugreift.

Darüber finden sich Interfaces, die die Funktionen der Treiber aufbereiten und in Funktionen gliedern, die es den oberen Schichten erlauben, ohne großen Aufwand und Kenntnis der Implementierungsdetails (konkreter Ein- beziehungsweise Ausgangs-Pin, Timing, usw.) auf die Hardware zuzugreifen. Neben den Treibern und Interfaces befindet sich im Background Level auch das Echtzeitbetriebssystem Contiki OS. Dieses beinhaltet unter anderem einen Scheduler, eine Prozessverwaltung und den Kommunikationsstack \textit{Rime} (siehe \autoref{sec:rime}). 

Schließlich folgt auf der höchsten Hierarchieebene das Agent Level. Hier befindet sich zunächst das AgentRTE, eine Laufzeitumgebung für Agenten. Dieses ist weitgehend hardwareunabhängig. Lediglich bei der Prozessverwaltung gibt es noch Unterschiede die in \autoref{sec:AgentRTE} noch näher betrachtet werden.
Aufgaben des AgentRTE sind vor allem die Verwaltung aller Agenten auf der Plattform und deren Scheduling, sowie der Austausch von Nachrichten untereinander.

Das letzte Glied in der Kette bilden letztendlich die Agenten. Sie werden vom AgentenRTE verwaltet und sind grundsätzlich hardwareunabhängig. Die Agenten bilden die echte Betriebslogik des Systems ab und kommunizieren dafür untereinander mit Nachrichten. Auf jedem Modul gibt es einen Platform-, einen Order- und einen Routing-Agenten. Dazu können auf den Volksbots ein und auf den Rampen bis zu vier Paket-Agenten registriert sein.

In den folgenden Abschnitten wird nun die Implementierung des Materialflusssystems anhand dieser Architektur erläutert, beginnend beim Echtzeitbetriebssystem Contiki, über die Treiber und Interfaces hin zum AgentenRTE und schließlich den Agenten.

\subsubsection{Contiki}
Contiki ist ein quelloffenes Echtzeitbetriebssystem (RTOS: Real Time Operating System), das in dieser Projektgruppe auf den \textsc{Mica}z-Modulen eingesetzt wird \cite{Contiki:2014:Online}. Es ist speziell für die Anforderungen des Internet of Things und von Wireless Sensor Networks zugeschnitten und bietet einen einfachen ereignisgesteuerten Betriebssystemkern mit sogenannten Protothreads (Threads, die sich einen gemeinsamen Stack teilen und daher schnell gewechselt werden können), optionalem präemptives Multithreading, Interprozess-Kommunikation via Message-Passing mit Events, eine dynamische Prozessstruktur mit Unterstützung für das Laden und Beenden von Prozessen und einen nativen Kommunikationsstack für die drahtlose Kommunikation gemäß dem IEEE-Standard \textit{802.15.4} \cite{IEEE802154:2014:Online}.
 
\paragraph{Build-Vorgang}\mbox{}\\
Es existieren Implementierungen von und Treiber für Contiki für eine Vielzahl von Plattformen. Dazu gehören neben \textsc{Mica}z-Modulen auch etwa der \textit{MSP430x} von Texas Instruments oder der \textit{Atmega128 RFA1} von Atmel. Für welche Plattform ein Contiki-System und die darauf geplanten Anwendungen gebaut wird, wird zur Compile-Zeit entschieden. Das heißt, um die selbe Anwendung mit Contiki auf mehrere Plattformen zu bringen, muss die Anwendung für jede Zielplattform neu gebaut werden.

Für jede Plattform existiert dafür ein eigener Ordner im \textit{platforms}-Verzeichnis der Contiki-Quelldateien. Um nun das Zielsystem zu wählen, muss lediglich das \textit{TARGET} beim Aufruf des entsprechenden Makefiles angegeben werden und das Build-System inkludiert automatisch die passenden Treiber und Definitionen.

Projektdateien können über die Makefile-Variable \textit{PROJECT\_SOURCEFILES} hinzugefügt werden. \autoref{lst:contikimakefile} zeigt exemplarisch das Makefile der Rampen.

\lstinputlisting[language=C, style=customc, captionpos=b, caption={Makefile des Contiki-Systems der Rampen}, label=lst:contikimakefile]{src/flow/lst/makefile.lst}


\paragraph{Prozesse}\mbox{}\\
Prozesse in Contiki implementieren folgen einem Konzept namens Protothreads. Dies erlaubt es Prozessen, ohne den Speicher-Overhead und die langen 
Prozesswechselzeiten von normalen Threads auszukommen, indem sie sich einen gemeinsamen Stack auf dem Hauptspeicher teilen.
Einzige Einschränkungen dieser Entwicklung sind, dass in Prozessen keine Switch-Case-Anweisungen auftreten dürfen und dass nur statische und globale Variablen zwischen zwei Aufrufen erhalten bleiben. Dynamisch erzeugte Variablen werden dagegen überschrieben. Entsprechend sollte der Zustand eines Prozesses mithilfe von statischen Variablen gespeichert werden. \autoref{lst:process} zeigt eine solche statische Variable (i) und den vollständigen Aufbau eines Prozesses.

\lstinputlisting[language=C, style=customc, captionpos=b, caption={Einfacher Beispiel-Prozess in Contiki}, label=lst:process]{src/flow/lst/process_example.lst}

In Zeile 1 wird der Prozess initialisiert und in Zeile 2 automatisch beim Boot von Contiki gestartet. Zeile 4 beinhaltet die Deklaration. So können andere Prozesse diesem Prozess Events (mit oder ohne Daten) schicken, auf die unser Beispielprozess mit ev und data zugreifen kann. Zeile 6 kennzeichnet den Beginn der tatsächlichen Ablauflogik. Code über dieser Zeile wird bei jedem Prozessaufruf ausgeführt, dies wird jedoch in den meisten Fällen nicht benötigt. Zeile 13 schließlich beendet den Prozess und entfernt ihn aus der Prozess-Liste des Kernels. In diesem Beispiel wird die Zeile jedoch nie erreicht, sodass der Prozess immer wieder aufgerufen wird, bis er von einem anderen Prozess beendet wird.

\paragraph{Prozesskommunikation}\mbox{}\\
In Contiki kommunizieren Prozesse über Events. Auch der Kernel versendet Events, um Prozesse über ihren Zustand (Init, Continue, Exit) oder über abgelaufene Timer zu informieren. Zur Identifikation werden dabei Event IDs genutzt. Die Event IDs 0-127 können vom Benutzer frei vergeben werden, während die Prozess IDs ab 128 vom System genutzt werden. Grundsätzlich unterscheidet Contiki zwischen synchronen und asynchronen Events. 

\begin{itemize}
\item \textbf{Asynchrone Events} werden vom Kernel in einer Warteschlange gespeichert. Die Scheduling-Funktion des Kernels läuft nach Systemstart in einer Endlosschleife. In jedem Durchlauf wird ein Event aus der Schlange entnommen und an den Zielprozess weitergeleitet.
\item \textbf{Synchrone Events} gleichen einem Funktionsaufruf.
Sie werden ohne Umweg über die Warteschlange direkt an den Empfänger-Prozess
zugestellt \cite{Contiki:2014:Online}.  Mit der Funktion \textit{process\_post\_synch(\&example\_process, EVENT\_ID, msg)} wird gezielt ein Prozess aufgerufen (ein Broadcast ist nicht möglich). Während der aufgerufene Prozess aktiv ist, blockiert der Aufrufer und setzt seine Ausführung erst fort, wenn der aufgerufene Prozess die Kontrolle wieder abgibt.
\end{itemize}

Um auf Events zu reagieren, können in Prozessen die folgenden Funktionen genutzt werden:

\begin{itemize}
\item PROCESS\_WAIT\_EVENT() - Wartet auf ein beliebiges Event, bevor die Ausf\"{u}hrung fortgesetzt wird.
\item PROCESS\_WAIT\_EVENT\_UNTIL(condition) - Wartet auf ein beliebiges Event, setzt die Ausf\"{u}hrung aber nur fort, wenn die Bedingung erf\"{u}llt ist.
\item PROCESS\_WAIT\_UNTIL() - Wartet, bis die Bedingung erf\"ullt ist. Muss den Prozess nicht zwangsl\"{a}ufig anhalten.
\end{itemize}

Prozesse können neben Events auch über Polling-Anfragen kommunizieren. Polls werden bei der Bearbeitung von Hardware-Interrupts genutzt, da Interrupt-Handler keine Events absetzen dürfen. Sie können als Events mit erhöhter Priorität betrachtet werden. Ein Prozess, der einen Poll erhalten hat, wird in der Warteschlange für Prozesse priorisiert. \cite{Contiki:2014:Online, Walter:2010}.

\paragraph{Scheduling und Timer}\mbox{}\\
Grundsätzlich nutzt Contiki ein Event-getriebenes Modell von Nebenläufigkeit, wobei einzelne Events nach dem Run-To-Completion (RTC) Prinzip abgearbeitet werden. Das heißt, einmal angelaufen können Prozesse nur noch von Hardware-Interrupts unterbrochen werden oder selbst die Kontrolle abgeben. Dies ermöglicht es, alle Prozesse auf dem selben Stack arbeiten zu lassen und so Hauptspeicher zu sparen. Auf diese Weise muss kaum Speicher dynamisch alloziert werden. Außerdem werden so Race-Conditions auf geteilten Speicher nahezu ausgeschlossen. Dabei haben alle Prozesse und Events vorerst die gleiche Priorität und werden streng nacheinander in Reihenfolge abgearbeitet.

Es existiert eine Bibliothek die auch echte Threads mit jeweils einzelnen Stacks ermöglicht. Aufgrund des ohnehin knappen Hauptspeichers auf den \textsc{Mica}z-Modulen wurde diese Möglichkeit jedoch in der Projektgruppe nicht weiter betrachtet.

Ein Problem der Event-getrieben Nebenläufigkeit ist jedoch das Reaktionsvermögen auf Echtzeitanforderungen und externe Events: Sollte ein Prozess eine aufwändige Berechnung durchführen, kann es zu spät sein, bis er die Kontrolle abgibt. Aus diesem Grund führt Contiki eine zweite Prioritätsebene ein, sogenannte Polls. Diese werden zwischen asynchron auftretende Events geplant und rufen in Reihenfolge einer Priorität alle Prozesse auf, die ein Polling-Flag gesetzt haben. Üblicherweise sind dies insbesondere hardwarenahe Prozesse, die auf Änderungen an den Ein- und Ausgangspins beziehungsweise auf Timer reagieren müssen.
\paragraph{Der Rime Kommunikationsstack}\mbox{}\\
Die drahtlose Kommunikation in Contiki erfolgt über einen leichtgewichtigen Netzwerstack namens \textit{Rime} \cite{Dunkels:2007:Proc}. . Dieser übergibt seine Daten an und erhält seine Daten von der sogenannten \textit{Charmeleon}-Architektur

Der \textit{Rime}-Stack implementiert das Network- und MAC- (beziehungsweise Data Link-)Layer des ISO OSI Referenzmodells. Darunter folgt eine sogenannte \textit{Radio Duty Cycling}-Schicht (RDC), die der Stromersparnis in drahtlosen Sensornetzwerken dient: Es ermöglicht, die Übertragungs- und Empfangseinheit des Moduls auszuschalten, während es nicht benötigt wird.

Auf der physikalischen Schicht schließlich wird die Übertragungseinheit über die Ein- und Ausgangspins angesteuert. Im Falle des \textsc{Mica}z-Moduls ist dies in CC2420-Chip für paketbasierte Funkübertragung auf einer Frequenz von 2.4 GHz \cite{CC2420:2014:Online}.

\autoref{fig:rime} zeigt den kompletten Stack, ausgehend vom Radio Driver, der in der Projektgruppe implementiert wurde bis hin zum Treiber des Funkmoduls.

\begin{figure}[h!]
 \centering
		\includegraphics[width=0.7\textwidth]{flow/Rime.png}
	\caption{Der Rime Netzwerkstack \cite{Dunkels:2007:Proc}}
	\label{fig:rime}
\end{figure}

Der Radio Driver der Projektgruppe implementiert ein einfach Network-Flooding, auf das im nächsten Abschnitt näher eingegangen wird. Dieses greift schließlich auf einen \textit{Atomic Broadcast Channel} (abc) zu. Dieser einfachste Channel in Contiki fügt der Nachricht lediglich die ID des Senders und eine Time-To-Live Angabe für das Network-Flooding als Header-Informationen hinzu. Der Channel sendet das Paket an den \textit{Rime Network Layer}. Dieser ruft den Chameleon-Service auf. Dieser sorgt für eine Trennung von Header-Informationen und Ebenen und reduziert den Header, indem er redundante Informationen entfernt. 

Anschließend wird das Paket der \textit{Carrier Sense Multiple Access}-Schicht (CSMA) übergeben. Da drahtlose Kommunikation immer über ein geteiltes Medium erfolgt, muss vor dem Senden geprüft werden, ob nicht gerade ein anderer Knoten sendet. Ist dies der Fall, wird mit der Übertragung gewartet, bis das Medium wieder frei ist.

Wurde ein freies Medium erkannt, wird das Paket an den RDC-Treiber übergeben. Im Falle unser Projektgruppe implementiert dieser das \textit{X-Mac}-Protokoll für energiesparende Kommunikation in drahtlosen Sensornetzwerken \cite{Buettner:2006:Proc}. Dieses nutzt Early Acking und möglichst frühe Übertragung der Adresse, um die nötigen Wachzeiten der Übertraungseinheiten der teilnehmenden Module möglichst gering zu halten und so Strom zu sparen. X-Mac sorgt weiterhin dafür, dass auch der Empfänger aufgeweckt und damit empfangsbereit ist. Ist dies der Fall, wird das Paket schließlich an den CC2420-Driver übergeben, der die passenden Ausgangspins bedient, um den externen Funkchip anzusprechen.

Ein eingehendes Paket nimmt exakt den umgekehrten Weg: Der CC2420-Chip löst einen Interrupt aus, sobald er ein Paket empfangen hat. Dieses wird von jeder der genannten Schichten verarbeitet und bis zum Radio Driver aus der Projektgruppe weitergeleitet.
\label{sec:rime}

\subsubsection{Treiber, Services und Interfaces}
Auf dieser Ebene werden der Agenten RTE alle Funktionalitäten zur Verfügung gestellt. Das RTOS stellt Infrastruktur wie z.~B. Task Management, Timing, Events usw.
Die Driver Ebene kümmert sich um die Schnittstellen/Pins des Controllers wie z.~B. Protokolle angeschlossener
Devices\cite[S. 26]{Stasch:Hahn}. Die Aufgabe des Services ist es, Funktionen für spezielle Anwendungen zur Verfügung zu stellen.
Darüber kommen die Interfaces. Diese sind nötig, damit bestimmte Funktionen immer gleich der Agenten RTE zur Verfügung gestellt werden, auch wenn der Service anders ist oder wenn ein Service verschiedene Interfaces bedienen soll\cite[S. 26]{Stasch:Hahn}. Im Laufe der Projektarbeit wurden unterschiedliche Treiber, Services und Interfaces selbst geschrieben. Der folgende Abschnitt soll diese etwas näher beleuchten.

\paragraph{Treiber Bolzen}

\paragraph{Interface Bolzen}

\paragraph{Treiber Externer Speicher}

\paragraph{Interface Externer Speicher}

\paragraph{Service Externer Speicher}

\paragraph{Treiber Lichtschranken}

\paragraph{Interface Lichtschranken}

\paragraph{Treiber Funkmodul}

\paragraph{Treiber UART-Schnittstelle}

\paragraph{Interface UART-Schnittstelle}


\subsubsection{Agenten RTE}

Zur Umsetzung der dezentralen Steuerung werden Softwareagenten eingesetzt. Bei Software-Agenten handelt es sich um Prozesse, die lose gekoppelt 
und leicht austauschbar sind \cite[vgl.][S. 31-37]{GH:2010}. Es existieren verschiedene Definitionen eines Agenten, von denen
sich keine als Standard etablieren konnte. Die hier verwendete Definition stammt von
Brenner, Zarnekow und Wittig. Sie definieren einen Agenten als „ ... ein Softwareprogramm,
das für einen Benutzer bestimmte Aufgaben erledigen kann und dabei einen Grad an
Intelligenz besitzt, der es befähigt, seine Aufgaben in Teilen autonom durchzuführen und mit
seiner Umwelt auf sinnvolle Art und Weise zu interagieren“\cite{BZW:1998}. Die Fähigkeit von Agenten, miteinander zu kommunizieren 
und zu interagieren, ermöglicht das Erstellen eines Multiagentensystems (MAS). Ein wesentlicher Vorteil von MAS 
bzw. von verteilten Steuerungssystemen ist die Fähigkeit, dynamisch auf Veränderungen zu reagieren. Ein Beispiel für eine solche Veränderung ist der
Ausfall einer Steuerungseinheit bzw. eines Agenten. Der Ausfall einer Einheit hat nicht unbedingt zur Folge, dass das gesamte System ausfällt. 
Die restlichen Einheiten können sich eigenständig auf eine solche Veränderung einstellen und diese beim weiteren Ablauf
berücksichtigen\cite[S. 13]{Roidl:2012}. Diese Eigenschaft bringt eine ganze Reihe von Vorteilen für ein dezentral
gesteuertes Materialflusssystem mit sich.\\
Die Modellierung von agentenbasierten Systemen für industrielle Bereiche wird durch die
Entwicklung von Standards festgelegt. Diese beschreiben Modelle für die Architektur
sowie die Kommunikation zwischen Agenten. Die FIPA (Foundation of Intelligent Physical Agents) ist das Standardisierungsorgan für Agentensysteme.
Seit der Gründung 1996 in der Schweiz wurden verschiedene Standardisierungen veröffentlicht, so zum Beispiel auch die Agentenkommunikation (Agent Communication), die als FIPA/ACL (Agent Communication Language, ACL) bekannt geworden ist. Jeder Agent ist mit einem eindeutigen
Identifikationsnamen (Agent Identifier oder AID) versehen und wird im Agent Management System verwaltet\cite[S. 24]{Roidl:2012}.
Ein Verzeichnisdienst kann optional vom Directory Facilitator gestellt
werden. Die Komponenten sind über das Message Transport System (MTS) verbunden,
sodass alle Komponenten untereinander kommunizieren können\cite[S. 24]{Roidl:2012}. \\
Als Referenzmodell zur Verwaltung von Softwareagenten wurde die Softwarearchitektur der Materialflussgruppe laut den FIPA Standards aufgebaut. 
Die nächste Abbildung zeigt den Aufbau der Softwarearchitektur und ist an die AUTOSAR Softwarearchitektur angelehnt:
\begin{figure}[h!]
	\centering
		\includegraphics[width=0.9\textwidth]{ArchitekturMicazRampe.png}
	\caption{Architektur Micaz Rampe\cite{Stasch:Hahn}}
	\label{ArchitekturMicazRampe}
\end{figure}
\paragraph{Hardware Level}
Auf der Hardwareebene werden die Treiber für die Steuerung der Lichtschranke und Bolzen implementiert. Die Treiber bekommen ein
allgemeines Interface für die Agenten RTE.
\subsection{Agenten}
Oberhalb des AgentRTE sind die Agenten implementiert. Sie bilden die oberste Ebene des Systems und implementieren die eigentliche Funktionalität. Dabei greifen sie auf das AgentenRTE und die verschiedenen Interfaces zu und kommunizieren untereinander über Agenten-Nachrichten. Auf jedem Modul agieren ein Plattform-Agent, der die Sensoren und Aktoren der Plattform steuert, ein Order-Agent, der zusammen mit anderen Order-Agenten die Betriebslogik des Materialflusssystems darstellt, ein Routing-Agent, der gemeinsam mit anderen Routing-Agenten die Pakete durch das System leitet und eventuell bis zu vier Paket-Agenten, die die physischen Pakete repräsentieren und durch das System wandern. Die einzelnen Agenten sind im Folgenden beschrieben.
\subsubsection{Paket-Agent}
Ein Paket-Agent repräsentiert ein physisches Paket. Seine ID ist gleichzeitig auch Paketnummer. Beim Eintritt in das System wird vom Plattform-Agenten ein neuer Paket-Agent initialisiert. Wechselt ein Paket von einem Modul auf das nächste, so wandert auch der Paket-Agent auf das neue Modul. Dies wird erreicht, indem der Agent auf dem einen Modul beendet und auf dem nächsten mit seinem Ziel und seiner ID neu initialisiert wird. Dies ist möglich, da sich verschiedene Pakete nur durch ihr Ziel und ihre ID voneinander unterscheiden. Die Übertragung der Paket-Agenten wird von den Plattform-Agenten der jeweiligen Module übernommen.

Das Ziel von Paket-Agenten ist dynamisch. Es wird von den Order-Agenten verwaltet. Kommt ein neuer Auftrag ins System, wird vom Order-Agenten, der den Auftrag verwaltet, eine Nachricht an das entsprechende Paket gesendet. Erhält der Paket-Agent eine solche Nachricht, wird dem Order-Agenten der Empfang bestätigt und das eigene Ziel wird angepasst.

Hat der Paket-Agent ein gültiges Ziel und wird ihm vom Plattform-Agenten per Flag die Erlaubnis erteilt, sendet er dem Routing-Agenten seiner Plattform eine Routing-Anfrage, bestehend aus dem Ziel des Pakets. Wenn diese nicht abgelehnt wird, wartet der Paket-Agent, bis sein Ziel geändert wird, oder er sich auf einer Plattform befindet, die nicht seinem aktuellen Ziel entspricht und er die erneute Erlaubnis bekommt, eine Routing-Anfrage zu stellen. Wird die Anfrage dagegen abgelehnt, stellt er beim nächsten Aufruf eine neue, bis die Anfrage vom Routing-Agenten angenommen wird.

Schließlich kann der momentane Zustand des Pakets über eine sogenannte \textit{UPDATE\_PHYSICAL}-Nachricht von einem Gateway abgefragt werden. Der Agent antwortet darauf mit der ID der Plattform, auf der er sich zur Zeit befindet und dem eigenen Ziel.
\subsubsection{Plattform-Agent}
Der Plattform-Agent ist der einzige plattformabhängige Agent des Systems. Während alle anderen Agenten auf allen Modultypen identisch sind, ist der Plattform-Agent modulspezifisch. Seine Aufgabe ist die Steuerung und Überwachung seines Moduls.

Beiden gemeinsam ist jedoch die Übertragung, also das Beenden und die erneute Initialisierung, von Paket-Agenten. Diese wird immer vom Volksbot initialisiert. Es wird eine Anfrage an den Plattform-Agenten der jeweiligen Rampe gesendet, entweder nach einem Lagerplatz oder nach einem speziellen Paket, abhängig davon, ob ein Paket abgelegt oder aufgenommen werden soll. Soll ein Paket abgelegt werden, muss die Anfrage bestätigt werden, bevor ein Registrierungs-Auftrag mit den Details des Paket-Agenten gesendet wird und der Agent auf seiner momentanen Plattform terminiert wird. Soll ein Paket aufgenommen werden, prüft die Rampe, ob das Paket mit der geforderten ID vorhanden ist und ausgegeben werden kann und sendet im Erfolgsfall ebenfalls einen Registrierungs-Auftrag und terminiert seinerseits den  entsprechenden Paket-Agenten, der dann auf dem Volksbot neu initialisiert wird.

Außerdem geben beide auf eine \textit{UPDATE\_PHYSICAL}-Nachricht die IDs aller Paket-Agenten zurück, die sich derzeit auf dem Modul befinden.

Im Folgenden werden nun die plattformspezifischen Eigenschaften der Plattform-Agenten beschrieben.

\paragraph{Plattform-Agent der Rampen}\mbox{}\\
Der Plattform-Agent auf einer Rampe ist neben der Verwaltung der Paket-Agenten auch für die Steuerung und das Auslesen der Bolzen über das Bolt\_Interface beziehungsweise Lichtschranken über das Photosensor\_Interface verantwortlich. Er sorgt dafür, dass die Pakete korrekt vereinzelt werden und verwaltet ihre Reihenfolge. Außerdem prüft er auf die Anfrage eines Volksbots, ein Paket abzulegen, anhand der Lichtschranken, ob noch ein Platz auf der Rampe zur Verfügung steht.
\paragraph{Plattform-Agent der Volksbots}\mbox{}\\
Der Plattform-Agent auf einem Volksbot kommuniziert mit dem Laptop, der den Volksbot steuert. Er erhält eine Nachricht, wenn der Volksbot seine Ziel-Position erreicht hat. Daraufhin initiiert er die Übergabe des Paketes. War die Übergabe des Paket-Agenten erfolgreich, sendet der dem Laptop eine Nachricht über die serielle UART-Nachricht, um das Fließband auf dem Volksbot zu starten und die physische Übernahme des Pakets zu starten.
\subsubsection{Order-Agent}
Die Order-Agenten übernehmen die Rolle eines zentralen Materialflussrechners im Materialflusssystem. Ihre Aufgabe ist die Verarbeitung von Aufträgen, sprich die Zuweisung von Zielen an Pakete. Aufträge bestehen aus Paket- und Ziel-ID und werden über ein Gateway in das System eingegeben. Dafür wird eine entsprechende Agenten-Nachricht an einen einzelnen Order-Agenten gesendet, der den Empfang bestätigt oder ablehnt, falls kein Platz in seinem Speicher zur Verfügung stand. In diesem Fall muss ein anderer Order-Agent mit dem Auftrag betraut werden. Ein Auftrag wird mit seiner Paket- und Ziel-ID sowie einem Status ("`zu bearbeiten"' beziehungsweise "`in Verteilung"') in einer Warteschlange ablegt.

Hat der Order-Agent in einem Aufruf keine Nachricht erhalten, durchsucht er diese Warteschlange nach zu bearbeitenden Aufträgen und sendet eine Nachricht an das entsprechende Paket, sein Ziel zu ändern. Wird der Empfang dieser Nachricht vom Paket bestätigt, wird der Auftrag gelöscht, von nun an ist das Paket für die Erreichung seines Ziels verantwortlich. Sind alle Aufträge in Verteilung muss davon ausgegangen werden, dass die entsprechenden Nachrichten nicht angekommen sind oder die Pakete noch nicht im System sind. Daher wird in diesem Fall der Status aller Aufträge zurückgesetzt und es wird erneut versucht, Nachrichten an die einzelnen Pakete zu senden.
\subsubsection{Routing-Agenten}

Die Routing-Agenten kümmern sich um die Wegplanung der Pakete im Materialflusssystem. Sie suchen nach einem Volksbot, der ein Paket möglichst günstig zu seinem Ziel bringen kann. Dafür führen sie untereinander eine Auktion durch, bei der ein Transportauftrag zu möglichst geringen Kosten an einen Volksbot vergeben wird. Ein Routing-Agent reagiert dabei auf die Routing-Anfrage eines Paket-Agenten. Ein Routing-Agent kann gleichzeitig nur an einer Auktion teilnehmen beziehungsweise diese initiieren. Hiermit wird verhindert, dass ein Routing-Agent gleichzeitig zwei Auktionen gewinnt und deshalb eine der Auktionen zurückgerollt und wiederholt werden muss.

Die Routing-Agenten werden als kommunizierende Zustandsautomaten implementiert. Zustandsübergänge können durch eingehende Nachrichten oder Timer ausgelöst werden. \autoref{fig:routing_agent_fsm} zeigt den zugrunde liegenden Automaten.

\begin{figure}[h!]
  \centering
    \includegraphics[width = 1.35\textwidth, angle=90]{flow/RoutingAgent_FSM.png}
    \caption{Zustandsautomat des Routing Agenten}
    \label{fig:routing_agent_fsm}
\end{figure}

Ein Routing-Agent startet stets in Zustand 0. In diesem Zustand wartet er auf eingehende Routing-Anfragen. Diese können entweder von Paketen auf der eigenen Plattform oder von anderen Routing-Agenten kommen. Sie unterscheiden sich im ersten Byte der Conversation-ID einer Agenten-Nachricht. Hier ist die Auktions-ID gespeichert, die für neue Routing-Anfragen von Paketen erst durch den Routing-Agenten bestimmt werden muss und daher auf 0 gesetzt ist. 

Bei Eingang einer Routing-Anfrage durch ein Paket wird eine neue Routing-Anfrage an alle Routing-Agenten versendet, ein Timer gestartet, der abläuft, wenn die Bearbeitungszeit erreicht ist, und in Zustand 3 übergegangen. Kommt die Anfrage von einem anderen Routing-Agenten prüft der Agent, ob das Ziel erreichbar ist. Für Module auf einem Volksbot ist dies immer wahr, für Module an Rampen immer falsch. Ist das Ziel erreichbar, wird eine UART-Nachricht an den Volksbot gesendet, der die Kosten der Fahrt vom anfragenden Routing-Agenten zum Ziel des Pakets berechnen soll. Anschließend geht der Automat in Zustand 1 über.

In Zustand 1 wartet der Agent auf die Antwort des Volksbots auf seine Kostenanfrage. Geht diese ein und sind die Kosten größer als null, bedeutet dies, dass das Ziel erreichbar ist. Der Agent sendet anschließend diese Kosten an den Initiator der Auktion und geht in Zustand 2 über, wo er auf die Antwort des Initiators wartet. Wird das Angebot bestätigt, wird der Volksbot beauftragt, sich zum Ausgang des Initiators zu bewegen und der Automat geht in Zustand 6 über. Wird die Anfrage abgelehnt, geht der Automat zurück in Zustand 0 und wartet auf neue Anfragen. In Zustand 6 wartet der Routing Agent auf eine Nachricht des Plattform-Agenten, der bestätigt, dass das Paket abgegeben wurde und neue Anfragen angenommen werden können.

In Zustand 3 wartet der Routing-Agent auf Angebote von anderen Routing-Agenten, nachdem er eine Routing-Anfrage für ein Paket auf der eigenen Plattform verschickt hat. Er speichert diese Angebote mit Absender-ID und Kosten. Läuft schließlich der Bearbeitungs-Timer ab, wird geprüft ob mindestens ein Angebot eingangen ist. Ist das der Fall, wird das beste Angebot bestimmt und der Agent geht in Zustand 4 über. Andernfalls wird ein neuer Timer gesetzt, bis die Anfrage wiederholt wird und der Agent geht in Zustand 5. In Zustand 5 wartet der Agent auf den Ablauf des Timers, versendet die Routing-Anfrage erneut an alle Routing-Agenten und geht zurück in Zustand 3. Zustand 4 dagegen bleibt aktiv, bis alle Teilnehmer der Auktion benachrichtigt wurden. In jedem Aufruf des Agenten wird eine Nachricht mit Cancel oder Acknowledge an einen weiteren Teilnehmer der Auktion gesendet. Sind alle Teilnehmer benachrichtigt, geht der Agent zurück in Zustand 0 und wartet auf neue Anfragen.


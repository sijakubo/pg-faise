\subsection{Anforderungen}
Aus einem drathlosen Sensorknoten-Netzwerk müssen die Rampen auf Mikrocontrollerebene miteinander kommunizieren können und die Fördereinheiten steuern.
Die Steuerung und Überwachung muss dezentral durch Softwareagenten ablaufen. Die Mikrocontroller dienen als Agentenplattform und bei der Entwicklung muss 
auf Leistungsressourcen (CPU, Kommunikationsbandbreite, Speicher) geachtet werden.
\subsubsection{Funktionale Anforderungen}
\begin{enumerate}
 \item \textbf{Aktorik/Sensorik}: Die Rampen verfügen über Magnetstifte zum Vereinzeln der Pakete und Lichtschranken zum Erkennen von ein- bzw. ausgehenden Paketen.
An den ATMega128 Mikrocontroller der Rampen werden die Lichtschranken und Magnetstifte angeschlossen.
 \item \textbf{Echtzeitsteuerung}: An den ATMega128 Mikrocontroller der Rampen werden die Lichtschranken und Magnetstifte angeschlossen.
Der Controller führt die Befehle für die Rampe aus. 
 \item \textbf{Kommunikation}: Die Micaz-Module kommunizieren drahtlos mit anderen Rampen und Volksbots.
 \item \textbf{Synchronisation}: Die Simulation wird über ein Micaz-Modul an die drahtlose Kommunikation angebunden.
Die Synchronisation der Zustände erfolgt mittels vorhandenem Ice Interface.
 \item \textbf{Disposition}: Die Controller kennen den Belegungszustand der Rampe und generieren entsprechend Aufträge, die sie an die Fahrzeuge vergeben. 
 Das ganze soll nach dem FIFO-Verfahren ablaufen.
 \item \textbf{Übergabe}: Wenn eine Ein- oder Auslagerung an einer Rampe ausgeführt werden soll, so übernimmt der Controller der Rampe die Kontrolle über die Fördereinheit des Fahrzeugs und sorgt dafür, dass das Paket verladen wird.
 \item \textbf{Kooperation}: Einsatz Kooperativer Lösungsstrategien für die Materialflusssteuerung, Überwachung und Steuerung mittels Multi-Agentensystem.
\end{enumerate}
\subsubsection{Nicht-funktionale Anforderungen}
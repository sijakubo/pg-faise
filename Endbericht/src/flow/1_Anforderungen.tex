\subsection{Anforderungen}
In diesem Abschnitt werden die gestellten Anforderungen zusammengetragen. Wir unterscheiden dabei zwischen funktionalen Anforderungen, die die direkte Funktionalität des fertigen Systems beschreiben, und nicht-funktionalen Anforderungen, die die qualitativen Eigenschaften des Systems widerspiegeln.
\subsubsection{Funktionale Anforderungen}
\begin{enumerate}
\item \textbf{Plattform}: Die physische Zelle wird als Netzwerk von Knoten in einem drahtlosen Sensornetzwerk implementiert. Als Plattform dienen MICAz-Module mit Atmel ATMega 128 Mikrocontroller und CC2420 Funkchip (siehe Abschnitt \ref{MICAZ}).
 \item \textbf{Aktorik/Sensorik}: Die Rampen verfügen über Magnetstifte zum Vereinzeln der Pakete und Lichtschranken zum Erkennen von Paketen. Sie werden von den MICAz-Modulen angesteuert beziehungsweise ausgelesen.
 \item \textbf{Kommunikation}: Die MICAz-Module auf Rampen und Volksbots kommunizieren drahtlos untereinander auf Basis von Agenten-Nachrichten.
 \item \textbf{Synchronisation}: Die Simulation wird über ein Micaz-Modul, das als Gateway fungiert, an die drahtlose Kommunikation angebunden. Die Synchronisation der Zustände erfolgt über eine serielle Schnittstelle.
 \item \textbf{Disposition}: Die Controller kennen den Belegungszustand der Rampe und generieren nach dem FIFO-Prinzip Aufträge, die sie an die Volksbots vergeben.
 \item \textbf{Übergabe}: Wenn eine Ein- oder Auslagerung an einer Rampe ausgeführt werden soll, so übernimmt der Controller der Rampe die Kontrolle über die Fördereinheit des Fahrzeugs und sorgt dafür, dass das Paket verladen wird.
 \item \textbf{Kooperation}: Einsatz kooperativer Lösungsstrategien für die Materialflusssteuerung, Überwachung und Steuerung mittels Multi-Agentensystem.
\end{enumerate}

\subsubsection{Nicht-funktionale Anforderungen}
\begin{enumerate}
\item \textbf{Ressourcen}: Bei der Entwicklung muss 
auf den sparsamen Umgang mit Hardwareressourcen (Rechenzeit, Kommunikationsbandbreite, Speicher) geachtet werden. Insbesondere der vorhandene Arbeitsspeicher und das Kommunikationsmedium dürfen nicht überlastet werden, um einen stabilen Betrieb zu garantieren. 
\item \textbf{Stabilität}: Es müssen Maßnahmen getroffen werden, um ein stabiles System zu schaffen. Dies gilt insbesondere für die möglichst verlustfreie Übertragung von drahtlosen Nachrichten.
\item \textbf{Volksbots}: Die Module des Materialfluss über eine definierte Schnittstelle mit den Fahrzeugen kommunizieren, um auf die Aktorik und Sensorik der Volksbots zugreifen und schließlich einen Transport der Pakete gewährleisten zu können.
\end{enumerate}
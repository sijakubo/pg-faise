\subsection{Agenten}
Auf diesem Level werden die Verschiedene Agenten implementiert. Die Agenten RTE soll den Agenten ihnen alle benötigten Schnittstellen zur
Verfügung stellen. Sie soll die Agenten aktivieren und deaktivieren können und die Kommunikation zwischen den Agenten managen
(Messages innerhalb und außerhalb der Plattform trennen)\cite[S. 26]{Stasch:Hahn}. 
\paragraph{Plattformagent}
Jeder Plattformagent ist für die Steuerung von der verantwortlichen Rampe zuständig.
Der Plattform Agent kümmert sich um die Annahme und Abgabe der Pakete mit den Volksbots. Er
hat die alleinige Zugriffsrechte auf die Lichtschranken und ausfahrbaren Stifte. Er überwacht auch die reale Position 
der Pakete und damit auch deren Reihenfolge.
\paragraph{Orderagent}
Der Orderagent spielt die Rolle der alten Materialflussrechner im Materialflusssystem. Seine aufgabe liegt bei der Bearbeitung der aufträge
wo und wann die ware zu sein hat. Sie befinden sich auf jeder Plattform und können den Paketen ihre Destination (Zielmodul)mit Prioritäten (z.B.
eine Ausgangszuweisung zu einem näheren Zeitpunkt hat höhere Priorität als eine Spätere und diese ist höher als eine Zwischeneinlagerung usw.) 
und ihre Agent ID zuweisen\cite[S. 30]{Stasch:Hahn}. 
\paragraph{Paketagent}
Der Paket Agent repräsentiert die physische Fördereinheit. Beim Eintritt im System kümmert sich der Paketagent um 
eine destination und schon bei vorhandener Destination gibt er dem Routingagent sein ziel mit Priorität zur weiteren Planung durch.
Er aktualisiert regelmäßg seinen Status und gibt neue Routinganfragen wenn sich seine Destination ändert.
Wenn das physische Paket das Modul wechselt, wandert der paketagent auch von der jeweiligen Plattform zur nächsten. 
Da jeder Paket Agent gleich ist wird der Plattformwechsel realisiert, in dem seine Parameter weitergegeben
werden, ein neuer Paket Agent auf der nächsten Plattform aktiviert wird und auf der vorigen Plattform deaktiviert\cite[S. 31]{Stasch:Hahn}.
\paragraph{Routingagent}
Der Routingagent kümmern sich um die Wegplanung der Agenten im Materialfluss. Der Weg bestimmt die Hops über welche Module das Paket geleitet wird.
Das Ziel der Paket-Agents ist es, an ihrem Ziel in der effizientesten Art und Weise durch die Wahl optimaler Routen 
mit minimalen Abständen und kürzesten Fahrzeiten anzukommen. Das heisst, dass Die Planungsanfragen
von Paketagent kommen und bei erfolgter Planung oder neu Planung werden die betroffenen Paket Agenten informiert. Für die Planung der 
Route wurde ein Routingalgorithmus entwickelt und implementiert.

\begin{figure}[!h]
  \centering
    \includegraphics[width = 1.38\textwidth, angle=90]{flow/RoutingAgent_FSM.png}
    \caption{Zustandsautomat des Routing Agenten}
    \label{fig:routing_agent_fsm}
\end{figure}

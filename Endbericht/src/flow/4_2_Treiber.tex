
\subsubsection{Treiber, Services und Interfaces}
Auf dieser Ebene werden der Agenten RTE alle Funktionalitäten zur Verfügung gestellt. Das RTOS stellt Infrastruktur wie z.~B. Task Management, Timing, Events usw.
Die Driver Ebene kümmert sich um die Schnittstellen/Pins des Controllers wie z.~B. Protokolle angeschlossener
Devices\cite[S. 26]{Stasch:Hahn}. Die Aufgabe des Services ist es, Funktionen für spezielle Anwendungen zur Verfügung zu stellen.
Darüber kommen die Interfaces. Diese sind nötig, damit bestimmte Funktionen immer gleich der Agenten RTE zur Verfügung gestellt werden, auch wenn der Service anders ist oder wenn ein Service verschiedene Interfaces bedienen soll\cite[S. 26]{Stasch:Hahn}. Im Laufe der Projektarbeit wurden unterschiedliche Treiber, Services und Interfaces selbst geschrieben. Der folgende Abschnitt soll diese etwas näher beleuchten.

\paragraph{Treiber Bolzen}

\paragraph{Interface Bolzen}

\paragraph{Treiber Externer Speicher}

\paragraph{Interface Externer Speicher}

\paragraph{Service Externer Speicher}

\paragraph{Treiber Lichtschranken}

\paragraph{Interface Lichtschranken}

\paragraph{Treiber Funkmodul}

\paragraph{Treiber UART-Schnittstelle}

\paragraph{Interface UART-Schnittstelle}


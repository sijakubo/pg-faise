\section{Projektorganisation}
Die nachfolgenden Abschnitte beschreiben, wie die Projektgruppe organisiert ist, um die vorgegebene Aufgabenstellung umzusetzen. Ziel ist es zu beschreiben, wie die Aufteilung in Teilgruppen stattfindet, welche Vorgehensmodelle für die Gesamt- und Teilgruppen verwendet werden, welche Rollen einzelne Personen haben und welche Softwaretools zur Unterstützung angewandt werden.   
\subsection{Organisation in drei Teilgruppen}
Die Projektgruppe ist in die drei Teilgruppen Materialfluss, Fahrzeuge und Simulation unterteilt. Die Teilgruppe Materialfluss befasst sich mit Programmierung der Sensorik und Aktorik für die Rampen sowie dem Aufbau eines Sensornetzwerkes zur Kommunikation zwischen den verschiedenen Akteuren der Simulation. Die Teilgruppe Fahrzeuge befasst sich mit allen Aspekten, die für das Funktionieren der Fahrzeuge verantwortlich sind. Dazu zählen u.a. Navigation, Odometrie und Lokalisierung. Auch ist die Einrichtung der Versorgungsinfrastruktur für die Fahrzeuge in Form von Ladestationen Aufgabe der Fahrzeuggruppe. Zusammen entwickeln die Teilgruppen Fahrzeuge und Materialfluss das physische System, so dass Kommunikation und Abstimmung zwischen diesen beiden Gruppen besonders wichtig sind. Die Teilgruppe Simulation entwickelt die Software mit der eine virtuelle Simulation erstellt werden kann. Außerdem beinhaltet die Software einen hybriden Modus, in dem das physische System auf die Software abgebildet wird und beide Teilsysteme ein Gesamtsystem bilden. Für die Entwicklung des Hybridmodus muss ein funktionierendes physisches System vorliegen. Die Aufteilung in drei Teilgruppen gliedert das Gesamtprojekt in eindeutig abgrenzbare Aufgabenfelder, so dass Kompetenzen und Verantwortlichkeiten klar definiert werden können.
\subsection{Vorgehensmodell}
Für die Durchführung der Projektgruppe muss ein Vorgehensmodell sowohl für die Gesamt- als auch für die Teilgruppen festgelegt werden. Durch ein Vorgehensmodell wird die Arbeit im Team strukturiert und es wird festgelegt, wie bestimmte Aufgaben, wie z.B. Abgleich mit Kunden und Anwendern, umgesetzt werden sollen. Sowohl für die Teilgruppen als auch für die Gesamtgruppe wurde Scrum als Vorgehensmodell gewählt. Da es für ein sehr komplexes Projekt, wie das vorliegende, schwierig ist nur von einer groben Vision sowie von User Stories auszugehen, wurden zunächst auf Basis des vorliegenden Lastenhefts in jeder Teilgruppe Pflichtenhefte erstellt, um die Aufgabenstellung ausreichend genau zu definieren und Stabilität zu schaffen. Anschließend wurde dazu übergegangen User Stories zu definieren, die die Anforderungen aus dem Pflichtenheft berücksichtigen und aus Anwendersicht darstellen. Die Sprints in den Teilgruppen sind mit einem Monat bemessen und werden zur Durchführung der User Stories genutzt. Die Rolle des Product Owners wird von den beiden Betreuern eingenommen, die sowohl für die Teil- als auch für die Gesamtgruppen zur Verfügung stehen, um entwickelte Funktionalität abzugleichen. Die Durchführung von Daily Scrums ist zeitlich nicht möglich, da es sich um eine studentische Projektgruppe handelt, deren Stundenplan keine täglichen Treffen ermöglicht. Deshalb wurde das Scrum Vorgehensmodell dahingehend angepasst, dass statt Daily Scrums Weekly Scrums durchgeführt werden. Die Weekly Scrums finden sowohl in den Teilgruppen als auch in der Gesamtgruppe statt. In den Daily Scrums der Gesamtgruppe wird zunächst von jeder Person berichtet, welche Aufgaben in der vorherigen Woche erledigt wurden. Damit verbundene Probleme und Hindernisse können direkt in der Gruppe besprochen und eventuell beseitigt werden. Die Ergebnisse aus den Teilgruppen werden ebenfalls vorgestellt und mit den Product Ownern abgeglichen. Das Scrum Vorgehensmodell wird mit  Prototyping kombiniert. Durch das Prototyping sollen zu bestimmten Meilensteinen die kombinierten Ergebnisse aus den Teilgruppen vorgestellt werden, um den Stand des Gesamtsystems begutachten zu können. Die genauere Beschreibung des Scrum Vorgehens für die Teilgruppen wird in den entsprechenden Kapiteln beschrieben, in der die Arbeit in den Teilgruppen aufgegriffen wird.   
\section{Integration der Teilsysteme}
Nachdem in den vorigen Abschnitten die jeweiligen Teilsysteme eingehend besprochen und vorgestellt wurden, wird in diesem Kapitel nun deren Integration beschrieben.

Dafür wird zunächst auf die fertiggestellte physische Zelle und auf deren Struktur und Verhalten eingegangen. Im nächsten Schritt werden dann die bereits erfolgten und die aus unserer Sicht nötigen Maßnahmen für eine vollständige Integration von Simulation und physischer Zelle beschrieben. Abschließend werden die Besonderheiten in der Implementierung der Teilsysteme erläutert, die bei der zukünftigen beziehungsweise weiteren Umsetzung dieser Integration von Bedeutung sind.
 
\subsection{Die physische Zelle}

\subsubsection{Aufbau}
\subsubsection{Ablauf}

Im folgenden wird der Ablauf für einen Transport innerhalb des physikalischen Aufbau aus Abbildung \ref{fig:physischeZelle} vom Ausgang der Rampe A zum Eingang der Rampe B beschrieben. Dafür muss zunächst ein Paket in das System initialisiert und dadurch ein Auftrag generiert werden. Anschließend kann durch den Materialfluss der Transport ausgelöst werden.

\begin{itemize}
\item Volksbot wird lokalisiert und wartet auf Auftrag
\item Registrierung der Pakete im System (Gateway)
\item Materialfluss sendet Nachricht (Opcode mit Ziel-ID: Ausgang Rampe A) an Volksbot
\item Volksbot entschlüsselt Opcode und startet Navigation zu Rampe A
\item Hubposition am Volksbot wird auf Höhe Ausgang Rampe A gefahren
\item Wenn Volksbot Navigationsziel erreicht hat und Hubhöhe eingestellt ist, wird eine Statusrückmeldung an Materialfluss gesendet
\item Rampe A gibt das Paket ab und der Volksbot nimmt durch Ansteuerung der Floweinheit das Paket auf
\item Volksbot fährt zum Eingang Rampe B und regelt die Hubhöhe auf entsprechende Rampenhöhe
\item Volksbot gibt Statusrückmeldung an Materialfluss und löst Paketübergabe aus
\item Volksbot frei für nächsten Auftrag


\end{itemize}


\subsection{Integration von Simulation und physischer Zelle}
\label{sec:Hybrid}
Bezogen auf das Gesamtsystem bietet es sich an, das physische und das virtuelle Teilsystem mithilfe eines hybriden Modus miteinander zu verknüpfen. Ein solcher Hybridmodus bietet viele weitere Möglichkeiten und könnte auf unterschiedlichen Wegen realisiert werden. Beispielsweise könnte die Software das physische System steuern, indem generierte Aufträge an das reale System geschickt werden. Auch könnten die Aktionen der Fahrzeuge und Rampen in der Software visualisiert werden. Besteht ein Teil der Visualisierung aus der Darstellung der realen Akteure und der andere aus rein virtuellen Akteuren, so könnten physische und virtuelle Akteure ein Gesamtsystem bilden, dass die Skalierbarkeit des physischen Systems erhöht. Dies setzt jedoch voraus, dass die Akteure aus beiden Systemen in ihren Eigenschaften (Geschwindigkeit etc.) weitestgehend aneinander angeglichen sind. Die technischen Voraussetzungen für das einbringen und abhören von drahtlosen Nachrichten in der physischen Zelle sind bereits durch Gateways realisiert. Die einzelnen physischen Module (Volksbots, Rampen) müssten jedoch zusätzlich regelmäßig über ihren Zustand informieren, um eine sinnvolle Auswertung der physischen Zelle in der Simulationsoberfläche zu erlauben. Gleichzeitig muss in der Simulation eine Möglichkeit geschaffen werden, physische Module hinzuzufügen, die nicht vom System simuliert, sondern stattdessen über Agentennachrichten angesprochen werden und Auskunft über ihren Zustand geben.

\subsection{Plattformbedingte Besonderheiten}

Im vorigen Unterabschnitt wurde die Integration der Teilsysteme behandelt. Um diese in einem nächsten Schritt weiter voranzutreiben, sind jedoch einige plattformbedingte Besonderheiten in der bisherigen Implementierung zu beachten, die im Folgenden angesprochen werden.

\subsubsection{Multiagentensystem und dessen Laufzeitumgebung}
Sowohl im Teilsystem Simulation als auch im Materialfluss kommen Agenten zum Einsatz, die die Betriebslogik abbilden. 

Da beide jedoch auf sehr unterschiedlichen Plattformen (Webserver beziehungsweise verteilte Mikrocontroller) zum Einsatz kommen, kann kein einheitliches Multiagentensystem zum Einsatz kommen. Stattdessen wird in der Simulation JADE verwendet, während im Materialfluss ein eigens entwickeltes AgentRTE zum Einsatz kommt, das speziell für den Einsatz auf Mikrocontrollern angepasst wurde. Bedingt durch die zur Verfügung stehenden Ressourcen ist dieses System im Bezug auf die maximale Kommunikationsbandbreite, die maximale Anzahl an Agenten und die Anzahl der gepufferten Nachrichten deutlich eingeschränkter. Auch erlaubt es im Gegensatz zu JADE keine \textit{Behaviors}, sondern das Verhalten der Agenten muss prozedural oder mit Zustandsautomaten abgebildet werden.

Die ausgetauschten Agenten-Nachrichten hingegen sind in beiden Systemen zumindest an den FIPA-Standard angelehnt, eine Umwandlung von einem in das andere Format kann daher durch eine einfache Abbildung realisiert werden. Weiterhin ist im physischen System die genaue Reihenfolge der Nachrichten-Parameter von großer Bedeutung, da hier nicht mit Objekten, sondern mit Byte-Arrays gearbeitet wird, die entsprechend interpretiert werden.

Das Multiagentensystem, dass für die Simulationssoftware entwickelt wurde, enthält neben den vier "Standardagenten" noch einen Job- und Statistikagent (Vgl. Abschnitt 5.4.5). Der Statistikagent hat keinen Einfluss auf den Ablauf der Simulation, der Jobagent hingegen übernimmt die Verteilung von Paketen und ausgehenden Aufträgen. Bei der Integration beider Teilsysteme muss der Jobagent miteinbezogen werden, sofern Pakete vom physischen System zu den Eingängen des virtuellen Systems  gelangen. Ist dies der Fall, so muss der Jobagent in die Datenübertragung miteinbezogen werden, da Eingänge keine eigenen Mechanismen zur Paketannahme besitzen. werden lediglich Pakete aus der Simulation zum physischen System befördert, so muss der Jobagent nicht miteinbezogen werden. Jedoch gibt es noch keinen Mechanismus, um Pakete, die an Ausgangsrampen eintreffen, aus der Simulation zu entfernen. Bei einer Integration beider Systeme über die Ausgangsrampen, könnte ein solcher Mechanismus im Abgleich mit dem physischen System entwickelt werden. 
\\\\
Physisches System und virtuelles System unterscheiden sich hinsichtlich des Paketagenten. In der Simulation wurde ein Paketagent erstellt, der alle Pakete verwaltet anstatt eines Paketagenten für jedes Paket. Grund dafür war, dass die Vielzahl an Agenten, die auf Basis von Jade und alle auf derselben Maschine liefen, zu Performance- und Synchronisationsproblemen führten. Eine weitere Aufsplittung des Paketagenten hätte diese Probleme noch verschärft. Für die Integration der beiden Teilsysteme muss untersucht werden, welche Aufgaben des Orderagenten der Paketagent in der Simulation übernimmt. Damit die Kommunikation zwischen zwei Rampen aus jeweils einem Teilsystem funktioniert, müssen die Kommunikationsprozesse zwischen den Agenten ggf. angepasst werden. Es ist unter Umständen möglich, dass beispielsweise die Zielfindung zwischen zwei Rampen im physischen Teilsystem andere Agenten einbezieht als die Zielfindung zwischen zwei Rampen aus beiden Teilsystemen.
\subsubsection{Pathfinding des Volksbot und in der Simulation}

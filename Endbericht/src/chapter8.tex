\section{Ergebnisse}
Dieses Kapitel gibt einen abschließenden Überblick über die erreichten Ergebnisse der Projektgruppe. Zunächst sollen zusammenfassend beschrieben werden, welche Komponenten erfolgreich entwickelt wurden. Im Fazit werden die aus der Projektarbeit gewonnenen Erkenntnisse und Erfahrungen beschrieben. Der letzte Teil dieses Kapitels beinhaltet eine Vision, in der potenzielle Weiterentwicklungen und Ideen aufgegriffen werden. 

\subsection{Realisierte Ziele}
%Todo Was haben Materialfluss und Fahrzeuge letztendlich umgesetzt?

Die Teilgruppe Simulation hat ein Simulationstool erstellt, das als Webanwendung realisiert wurde. Das Tool erlaubt das Erstellen von Szenarien mit einer vom Nutzer festgelegten Anzahl von Fahrzeugen und Rampen. Auch kann ein Nutzer eine beliebige Anzahl an Aufträgen generieren, die als Input für das Simulieren eines Szenarios dienen. Die Aktionen der Akteure werden serverseitig durch das Multiagentenframework JADE realisiert und clientseitig dargestellt. Jeder Akteur wird durch mehrere Agenten realisiert, die Zielsuche und Suche eines Transportmittels autonom und ohne zentrale Steuerung durchführen. Die Agenten der Fahrzeuge sind in der Lage mit einem selbst erstellten Pathfinding-Algorithmus Routen zu berechnen und diese anschließend abzufahren. Für einen Simulationsdurchlauf werden Daten über Paketdurchlaufzeit und Auslastung der Roboter mitgeloggt, die in der Statistik eingesehen werden können.

\subsection{Fazit}
Im Rahmen der Projektgruppe wurde von zwölf Teilnehmern ein System entwickelt, dass auf die drei Teilgruppen Materialfluss, Fahrzeuge und Simulation verteilt wurde. Durch die gemeinsame Arbeit an einem Projekt konnten Erfahrungen und Erkenntnisse auf dem Gebiet der Projektarbeit und Softwareentwicklung gewonnen werden. Ein wesentlicher Erfolgsfaktor eines Projekts ist das effektive Nutzen der vorhanden Zeit bei einem Projekt mit begrenzter Dauer. Grundsätzlich wurden Aufgaben sorgfältig und möglichst schnell durchgeführt. Für zukünftige Projekte empfiehlt es sich jedoch das Zeitmanagement zu optimieren, indem versucht wird Aufgaben noch weiter zu parallelisieren. Im Rahmen der Anforderungserhebung hatte sich bereits herausgestellt, dass bestimmte Technologien, wie beispielsweise ein Multiagentensystem zu einem späteren Zeitpunkt benötigt werden. Die Einarbeitung in benötigte Technologien erfolgte jeweils kurz vor der Implementierung einer Technologie, wodurch sich die Entwicklungszeit verlängerte. Um die Entwicklungszeit zu verkürzen, sollten Teilnehmer eines Projekts sich vor der Implementierung einer Komponente in die benötigte Technologie einarbeiten und diese für die anderen Teilnehmer aufbereiten. 
\\\\
Die Durchführung des Projekts mithilfe des Scrum Vorgehensmodell beinhaltete Sitzungen mit den Auftraggebern in denen die festgelegten Anforderungen mit den tatsächlich realisierten abgeglichen wurden. So konnten Eigenschaften des Produkts, die nicht den Vorstellungen der Auftraggeber entsprachen, direkt und mit geringerem Aufwand korrigiert werden. Ein solcher Abgleich sollte unabhängig vom dem gewählten Vorgehensmodell regelmäßiger Bestandteil eines Projekts sein. 
\\\\
Für die Implementierung der Agenten sowohl im physischen als auch im softwarebasierten System, wurden die Funktionalitäten, die ein Fahrzeug oder eine Rampe besitzen soll, auf mehrere Agenten verteilt. Ziel war es die Agentensysteme zu modularisieren. Modularisierung soll grundsätzlich die Wartbarkeit und Übersichtlichkeit eines Systems erhöhen. Jedoch hat sich gezeigt, dass eine Modularisierung auch Nachteile haben kann. Durch die Vielzahl der Agenten und die Vielzahl an Nachrichten, die ausgetauscht werden, wurde die Testbarkeit des Systems verringert, da auftretende Fehler schwer einzelnen Methoden der Agenten zugeordnet werden konnten. Für zukünftige Projekte sollte differenzierter betrachtet werden, ob eine Modularisierung unter Berücksichtigung der systemspezifischen Eigenschaften sinnvoll ist. 

\subsection{Vision}
%Todo Vision für Materialfluss und Fahrzeuge
Für die Simulationssoftware bieten sich folgende Optimierungsmöglichkeiten: Die Fahrzeuge fahren selbstständig zu einem Ziel ohne dabei Kollisionen mit anderen Fahrzeugen zu vermeiden. Die Fahrzeuge fahren durcheinander durch. Um Kollisionen vorzubeugen, könnten Pfade entweder reserviert werden oder Roboter könnten untereinander Nachrichten austauschen, um festzustellen, ob ein Feld frei ist oder nicht. Um den Kommunikationsaufwand zu verringern, empfiehlt es sich Pfade zu reservieren. Weiterhin sollte es die Möglichkeit geben Roboter zur Laufzeit hinzuzufügen, da ein physisches System auch dynamisch skaliert werden kann, indem ein weiterer Roboter hinzugefügt wird. Durch die Implementierung der genannten Funktionalitäten, wird das System noch weiter an das physische System angeglichen und eine Simulation erzeugt realistischere Ergebnisse.
\\\\
Bezogen auf das Gesamtsystem bietet es sich an das physische und das virtuelle Teilsystem mithilfe eines hybriden Modus miteinander zu verknüpfen. Ein solcher Hybridmodus bietet viele weitere Möglichkeiten und könnte auf unterschiedliche Weisen realisiert werden. Beispielsweise könnte die Software das physische System steuern, indem generierte Aufträge an das reale System geschickt werden. Auch könnten die Aktionen der Fahrzeuge und Rampen in der Software visualisiert werden. Besteht ein Teil der Visualisierung aus der Darstellung der realen Akteure und der andere aus rein virtuellen Akteuren, so könnten physische und virtuelle Akteure ein Gesamtsystem bilden, dass die Skalierbarkeit des physischen Systems erhöht. Dies setzt jedoch voraus, dass die Akteure aus beiden Systemen in ihren Eigenschaften (Geschwindigkeit etc.) weitestgehend vollständig aneinander angeglichen sind.





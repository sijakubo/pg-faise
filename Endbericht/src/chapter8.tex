\section{Integration der Teilsysteme}
Nachdem in den vorigen Abschnitten die jeweiligen Teilsysteme eingehend besprochen und vorgestellt wurden, wird in diesem Kapitel nun deren Integration beschrieben.

Dafür wird zunächst auf die fertiggestellte physische Zelle und auf deren Struktur und Verhalten eingegangen. Im nächsten Schritt werden dann die bereits erfolgten und die aus unserer Sicht nötigen Maßnahmen für eine vollständige Integration von Simulation und physischer Zelle beschrieben. Abschließend werden die Besonderheiten in der Implementierung der Teilsysteme erläutert, die bei der zukünftigen beziehungsweise weiteren Umsetzung dieser Integration von Bedeutung sind.
 
\subsection{Die physische Zelle}
In der physischen Zelle wird durch die Integration der Fahrzeuge und des Materialflusses ein einfaches Umschlagslager abgebildet. Dabei werden die Betriebslogik und die Steuerung der Warenlagerplätze (Rampen) durch den Materialfluss übernommen, während die Fahrzeuge (Volksbots) für den eigentlichen Transport der Waren (Pakete) innerhalb des Lagers verantwortlich sind.

Im Folgenden werden nun zunächst der allgemeine Aufbau und anschließend der erreichte Ablauf beschrieben.

\subsubsection{Aufbau}
In der physischen Zelle wird durch zwei Rampen (Lagerplätze) und einen Volksbot (AGV) ein kleines Umschlagslager dargestellt. Zudem können über ein \textsc{Mica}z-Modul, das als Gateway fungiert, Agenten-Nachrichten und damit etwa Pakete oder Aufträge in das System eingespeist werden.

\autoref{fig:physischeZelle} zeigt den Beispielaufbau. Die beiden Rampen A und B sind über den Volksbot in der Lage, Pakete auszutauschen. Initialisiert werden sowohl Pakete als auch entsprechende Aufträge für diese Pakete über das \textsc{Mica}z-Gateway unten links.

\begin{figure}[h!]
	\centering
		\includegraphics[width=0.5\textwidth]{physischeZelle.png}
	\caption{Aufbau der physischen Zelle}
	\label{fig:physischeZelle}
\end{figure}

Auf allen Modulen (Rampen und Volksbots) sind \textsc{Mica}z-Module angebracht, die sich um die Betriebslogik kümmern und etwa den Volksbots ihre nächsten Ziele zuweisen. Die \textsc{Mica}z-Module kommunizieren dabei untereinander drahtlos auf Agentenbasis (siehe \autoref{sec:FlowAgents}). Die Module auf den Volksbots kommunizieren außerdem mit den auf den Volksbots angebrachten Laptops über eine UART-Schnittstelle. Das entsprechende Protokoll ist in den Tabellen \ref{tab:UART_FD} und \ref{tab:UART_DF} zu sehen. Jede Nachricht hat dabei genau fünf Byte, ein Byte Opcode und bis zu vier Byte Daten. Benötigt ein Befehl weniger Daten, werden die restlichen Bytes mit Nullen aufgefüllt.

Ausgehende Nachrichten auf Seiten der \textsc{Mica}z-Module werden dabei direkt über das UART-Interface verschickt, während eingehende Nachrichten vom CommunicationInterface aufbereitet und dem jeweiligen Zielagent als Agentennachrichten bereitgestellt werden. So kann die asynchrone UART-Nachricht vom Agenten bei seinem nächsten Aufruf abgerufen und verarbeitet werden.

\begin{table}[h!]
\centering
\begin{tabular}{| l | l | l | l |}
  \hline
  Nachricht & Opcode & Parameter & Agent (\textsc{Mica}z)\\
  \hline
  Kostenanfrage & 0x10 & Start-ID + Ziel-ID & Routing-Agent\\
  Hole Paket ab & 0x02 & Rampen-ID & Plattform-Agent\\
  Transportiere Paket & 0x01 & Rampen-ID & Plattform-Agent \\
  \hline
\end{tabular}
\caption{UART-Kommunikation von \textsc{Mica}z-Modul zu Volksbot}
\label{tab:UART_FD}
\end{table}

\begin{table}[h!]
\centering
\begin{tabular}{| l | l | l | l |}
  \hline
  Nachricht & Opcode & Parameter & Zielagent (\textsc{Mica}z) \\
  \hline
  Antwort auf Kostenanfrage & 0x11 & Start-ID + Kosten & Routing-Agent \\
  Rampe mit Paket erreicht & 0x50 & Rampen-ID & Plattform-Agent\\
  Rampe ohne Paket erreicht & 0x55 & Rampen-ID & Plattform-Agent\\
  \hline
\end{tabular}
\caption{UART-Kommunikation von Volksbot zu \textsc{Mica}z-Modul}
\label{tab:UART_DF}
\end{table}

\subsubsection{Ablauf}

Zunächst wird ein Paket innerhalb des Systems initialisiert und dadurch ein Auftrag generiert.
%wird heute nachm. fertig gemacht !

\subsection{Integration von Simulation und physischer Zelle}

\subsection{Plattformbedingte Besonderheiten}

Im vorigen Unterabschnitt wurde die Integration der Teilsysteme behandelt. Um diese in einem nächsten Schritt weiter voranzutreiben, sind jedoch einige plattformbedingte Besonderheiten in der bisherigen Implementierung zu beachten, die im Folgenden angesprochen werden.

\subsubsection{Multiagentensystem und dessen Laufzeitumgebung}
Sowohl im Teilsystem Simulation als auch im Materialfluss kommen Agenten zum Einsatz, die die Betriebslogik abbilden. 

Da beide jedoch auf sehr unterschiedlichen Plattformen (Webserver beziehungsweise verteilte Mikrocontroller) zum Einsatz kommen, kann kein einheitliches Multiagentensystem zum Einsatz kommen. Stattdessen wird in der Simulation JADE verwendet, während im Materialfluss ein eigens entwickeltes AgentRTE zum Einsatz kommt, das speziell für den Einsatz auf Mikrocontrollern angepasst wurde. Bedingt durch die zur Verfügung stehenden Ressourcen ist dieses System im Bezug auf die maximale Kommunikationsbandbreite, die maximale Anzahl an Agenten und die Anzahl der gepufferten Nachrichten deutlich eingeschränkter. Auch erlaubt es im Gegensatz zu JADE keine \textit{Behaviors}, sondern das Verhalten der Agenten muss prozedural oder mit Zustandsautomaten abgebildet werden.

Die ausgetauschten Agenten-Nachrichten hingegen sind in beiden Systemen zumindest an den FIPA-Standard angelehnt, eine Umwandlung von einem in das andere Format kann daher durch eine einfache Abbildung realisiert werden. Weiterhin ist im physischen System die genaue Reihenfolge der Nachrichten-Parameter von großer Bedeutung, da hier nicht mit Objekten, sondern mit Byte-Arrays gearbeitet wird, die entsprechend interpretiert werden.

\begin{itemize}
\item TODO für Simulation:
\begin{itemize}
\item Welche Agenten gibt es zusätzlich zu Platform-, Routing-, Order- und Paketagenten?
\item Warum musste von dieser Struktur abgewichen werden?
\item Begründung: Warum konnten Paketagenten nicht ein einzelnes Paket repräsentieren?
\item Kann all das bei der Integration zu Schwierigkeiten führen? Worauf muss evtl geachtet werden?
\end{itemize}
\end{itemize}
\subsubsection{Pathfinding des Volksbot und in der Simulation}

\subsection{Allgemeine Hardware Initialisierung}

Der erste Schritt der Aufgabenstellung sah die Bereitstellung von Treibern für die komplette Hardware des Volksbots vor. Die Initialisierung und Ansteuerung der an den EPOS2-Controllern angeschlossenen Peripherie erfolgte unter der Verwendung der EPOS2 Bibliothek. Diese Bibliothek verfügt über alle benötigten Funktionen zum Ansteuern und Auslesen der Motoren und Lichtschranken, welche mit den EPOS-Controllern verbunden sind. Die Funktionen wurden dem ROS-Package \textit{epos2\_control} zusammengeführt und sind in der Epos2MotorController 	-Node des Packages enthalten. 
Die Verwendung der Laserscanner, welche nicht mit den EPOS2-Controllern verbunden sind, wurde durch die Implementierung und Parametrisierung von im ROS vorhandenen Treibern gewährleistet. Dem an der Front des Roboters angebrachten SICK LMS100 Laserscanner, wurde dafür unter Windows eine feste IP-Adresse zugeordnet. Mit Hilfe dieser bekannten Adresse, konnte auch unter Ubuntu die Verbindung über Ethernet mit dem Laserscanner durch das Package „LMS1xx“  hergestellt werden.



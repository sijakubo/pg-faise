\subsection{Ausblick}

Der Volksbot sollte um folgende Funktionalitäten erweitert werden:

\begin{itemize}

\item \textbf{Genauigkeit bei der Übergabe}: Die Genauigkeit des Heranfahrens an eine Zielposition, wie z.B. den Ausgang einer Rampe, muss verbessert werden, um eine erfolgreiche Übergabe zu gewährleisten. Der Volksbot stürzt bei direktem anfahren an die Rampe ab, da der Bewegungsraum eingeschränkt wird.

\item \textbf{Kollisionserkennung mobile Hindernisse - Schwarmverhalten}: Der Volksbot sollte Hindernisse, die sich selbstständig bewegen, erkennen und entsprechend reagieren. Dies könnte beim erkennen eines Objektes mittels einer Neuplanung der Route, einer kurzen Unterbrechung der Fahrt oder eines Ausweichmanövers umgesetzt werden.

\item \textbf{Rückkopplung an Simulation}: Die Volksbots sollten ihre aktuelle Position, sowie den Beladungs- und Eneriezustand an die Simulation zurückgeben.

\item \textbf{Fahralgorithmus}: Es können alternative Alorithmen implementiert werden, um die Berechnungszeiten zu verkürzen oder optimalere Routen zu finden.

\item \textbf{Kostenabschätzung}: Eine optimale Berechnung zur Bestiimmung der benötigten Zeit, Strecke und Energieverbrauchs würde die Jobverteilung effizienter gestalten.

\item \textbf{Ladestation}: Die Ladestation muss angebracht und getestet werde. Danach fehlt die automatische Erkennung des kritischen Zustandes, sowie das selbstständige anfahren der Ladestation, damit der Akku geladen werden kann. Hierbei handelt es sich um einen Prototypen der noch getestet werden muss.

\end{itemize}

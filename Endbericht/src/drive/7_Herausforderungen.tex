\subsection{Herausforderungen}

Bei der Implementierung der Funktionen des Volksbot traten einige Herausforderungen auf, deren Lösungen besonderen Aufwand hervorriefen oder nicht umgesetzt wurden. Ein Beispiel dafür ist die Selbstlokalisation des Roboters durch die Odometrieberechnung und dessen Verbesserung durch die adaptive Monte Carlo Lokalisation (AMCL). Aufgrund der Fehleranfälligkeit der einfachen Odometrie durch Einflüsse wie die Beschaffenheit des Bodens oder Unterschiede im Reifendruck, ist der Einsatz von Algorithmen wie die adaptive Monte Carlo Lokalisation, welcher die Daten des Laserscanners nutzt, notwendig.  Die Qualität der Odometrie musste durch das Kalibrieren des Parameters der Achsenlänge des Roboters erhöht werden, um die nötige Voraussetzung für einen funktionierenden AMCL-Algorithmus zu schaffen. Eine gute Odometrie lässt sich dadurch erkennen, dass die Punkte des Laserscans auch bei Bewegung weiterhin mit den Wänden der Umgebungskarte übereinstimmen. Durch die Verwendung der Selbstlokalisation über AMCL führt der Roboter während seiner Navigation Drehungen um die eigene Achse aus, um seine Position zu berechnen. Dieses Verhalten führt dazu, dass der Roboter deutlich mehr Zeit als vermutet benötigt, um seinen Weg zurückzulegen. Bei der Parametrisierung für die Wegplanung und des AMCL-Algorithmus musste ein Gleichgewicht zwischen Schnelligkeit und Genauigkeit gefunden werden, um ein stabiles System zu schaffen. Durch die Vielzahl der Möglichkeiten war die Suche nach geeigneten Einstellungen problematisch.  

Eine weitere Herausforderung stellt die Erkennung von Hindernissen dar, die ober- oder unterhalb des Laserscanners in den Raum ragen. Zur Lösung dieses Problems könnte ein weiterer bildgebender Sensor genutzt werden, um den gesamten Raum nach potenziellen Hindernissen zu untersuchen.

Mit großem Aufwand war die korrekte Ansteuerung der Maxon Controller verbunden. Besonders die Parametrisierung des kleineren Maxon Controllers unter ROS, welcher für den Betrieb des Förderbandes verwendet wird, stellte sich als Problem heraus. Schon kleinere Abweichungen der Parameter für Spannungs- und Beschleunigungswerte in der Ansteuerung des Controllers führten zu Systemabstürzen.
 
Die Programmierung und Parametrisierung der Teilfunktionen musste mit Blick auf
die CPU-Auslastung der Steuereinheit geschehen, da einige Berechnungen bei falscher
Verwendung zu sehr hohen Auslastungen und einer unzureichenden Performance des
Roboters führten.

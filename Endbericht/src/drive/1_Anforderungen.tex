\subsection{Anforderungen}

\begin{itemize}

\item \textbf{Aufträge}: Aufträge enthalten den Start- und Zielpunkt, welche in der Umgebungskarte gesetzt werden. Diese werden für weitere Berechnungen verwendet.

\item \textbf{Bieten}: Ein Volksbot ist in der Lage an einer Auftragsverteilung teilzunehmen, sofern er noch keinen Auftrag ausführt und sein Energievorrat nicht das kritische Minimum erreicht hat. Die Teilnahme beinhaltet eine Aufwandsschätzung anhand einer Distanzfunktion, sowie dem aktuellen Energiezustand. 

\item \textbf{Planung}: Bei der Erteilung eines Auftrages und dem Bieten für einen Auftrag, muss der Volksbot die Route vom aktuellen Standort zum Startpunkt berechnen. Dies gilt ebenso vom Startpunkt zum Endpunkt.

\item \textbf{Navigation}: Für die Routenplanung soll der kürzeste Weg verwendet werden, sofern möglich die direkte Verbindung zum gewüschten Ziel. Die Planung, Navigation und Positionierung erfolgt dabei auf dem entsprechenden Notebook.

\item \textbf{Positionierung}: Die Umgebungskarte ist jedem Volksbot bekannt. Bei Bewegungen aktualisiert dieser seine Position mittels Odometrie, um die geplante Route korrekt zu befahren. Zur lokalen Unterstützung werden die Laserscanner verwendet werden. Startpunkt eines Volksbots wird statisch definiert.

\item \textbf{Synchronisation}: Jegliche Information eines Volksbots wird an die Simulation übermittelt. Dies beinhaltet neben der aktuellen Position auch den Ladezustand, sowie den Energievorrat eines Volksbots.

\item \textbf{Feinsteuerung}: Das genaue Heranfahren an ein Objekt, sei es die Ladestation oder eine Rampe, erfolgt mit Hilfe der Laserscanner. 

\item \textbf{Überabe}: Bei Einnahme der korrekten Position zum Be- oder Entladen des Volksbots findet ein Datenaustausch mit der entsprechenden Rampe statt. Es folgt eine kooperative Interaktion beider, bis der Volksbot das Paket erhalten oder abgeladen hat.

\item \textbf{Energiemanagement}: Sobald ein Volksbot seinen kritischen Energiezustand erreicht, werden alle Auftragsverteilungen ignoriert und der Bot setzt die Dockingstation als primäres Ziel. Sollten mehrere Bots die Ladestation ansteuern, oder diese bereits belegt sein, so wird der Folgeablauf durch eine Queue oder ein anderes Verfahren geregelt.

\item \textbf{Kollisionsvermeidung}: Sobald eine Kollision mit einem festen oder mobilen Objekt erkannt wird, wird eine Neuberechnung, Umplanung oder ein Ausweichverfahren eingeleitet um entsprechend zu reagieren.

\end{itemize}
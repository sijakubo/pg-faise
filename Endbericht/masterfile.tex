\documentclass[	11pt,		% 11 Punkte hohe Schrift
				a4paper,	% Einstellung des Ausdrucks auf DIN A4
				fleqn,		% linksbündige, abgesetzte Formeln
				reqno,		% rechts stehende Formelnummer
				BCOR12mm,	% Bindekorrektur von 12mm
				chapterprefix=false,	% Ausgabe von Kapitel: 
				appendixprefix=true, % Überschrift im Anhang X und dann neue Zeile und Titel
				parskip=half,
				DIV=calc,
				final,		% Abgabe Version
]{scrartcl}

\usepackage{packages}
\usepackage{layout}

\graphicspath{{image/}}
\DeclareGraphicsExtensions{.png,.pdf}

\bibliography{src/literatur}	% Literaturverzeichnis

\newtheorem{defi}{Definition}
\newtheorem{bsp}{Beispiel}
\newtheorem{satz}{Satz}
\newtheorem{theo}{Theorem}

\newcommand{\bspautorefname}{Beispiel}
\newcommand{\defiautorefname}[1]{Definition #1}
\newcommand{\theoautorefname}{Theorem}

\newcommand{\RM}[1]{\MakeUppercase{\romannumeral #1{}}} 

\begin{document}
	% Cover
	\begin{titlepage}
  \begin{centering}
  \begin{figure}[h!]
    \centering
    \includegraphics[width=310pt]{UniLogo}
  \end{figure}

  \vspace*{-0.8cm}

  \begin{figure}[h!]
    \centering
    \includegraphics[width=280pt]{BUIENGLOGO}
  \end{figure}

  \vspace*{0.4cm}
  
  \textsf{\Huge \textbf{Projektgruppe FAISE\\}}

  \vspace*{0.5cm}
  \noindent Endbericht\\
  im Rahmen des Masterstudiums

  \end{centering}
  
  \vspace*{1.5cm}
  \begin{tabbing}
  xxxxxxxxxxxxxxxx\= \kill
  
  \small Betreuer: \>Prof. Dr.-Ing. Jürgen Sauer\\
  \small \>Dipl.-Ing. (FH) Arne Stasch\\
  \small \>Dipl.-Inform. Jan-Hinrich Kämper\\
  \small \>Prof. Dr.-Ing. Axel Hahn\\\\

  \small Vorgelegt von: \>Berthe Pulcherie Ongnomo\\
  \small \>Chancelle Merveille Tematio Yme\\
  \small \>Christopher Schwarz\\
  \small \>Jan Paul Vox\\
  \small \>Jan-Gerd Meß\\
  \small \>Jannik Fleßner\\
  \small \>Malte Falk\\
  \small \>Matthias Aden\\
  \small \>Michael Goldenstein\\
  \small \>Nagihan Aydin\\
  \small \>Raschid Alkhatib\\
  \small \>Simon Jakubowski\\\\

  \small Abgabetermin:\> 30. September 2014
  \end{tabbing}
\end{titlepage}
	\pagenumbering{Roman}
	% Abstract
	\include{src/abstract}

	% Table of contents
	\clearpage
\ohead[Inhaltsverzeichnis]{Inhaltsverzeichnis}
\chead[Uni Oldenburg]{Uni Oldenburg}
\ihead[PG FAISE]{PG FAISE}
\setheadtopline{1pt}
\setheadsepline{0.5pt}

\ofoot[Endbericht]{Endbericht}
\cfoot[\pagemark]{\pagemark}
\ifoot[30. September 2014]{30. September 2014}
\setfootsepline{0.5pt}
\setfootbotline{1pt}

\tableofcontents
\clearpage

\ohead[Abbildungsverzeichnis]{Abbildungsverzeichnis}
\chead[Uni Oldenburg]{Uni Oldenburg}
\ihead[PG FAISE]{PG FAISE}
\setheadtopline{1pt}
\setheadsepline{0.5pt}

\ofoot[Endbericht]{Endbericht}
\cfoot[\pagemark]{\pagemark}
\ifoot[30. September 2014]{30. September 2014}
\setfootsepline{0.5pt}
\setfootbotline{1pt}

\listoffigures
\listoftables
	
	% Bibliography
	\clearpage
	\clearscrheadfoot
	\ohead[Inhaltsverzeichnis]{Inhaltsverzeichnis}
	\chead[Uni Oldenburg]{Uni Oldenburg}
	\ihead[Malte Falk]{Malte Falk}
	\ofoot[Seminararbeit]{Seminararbeit}
	\cfoot[\pagemark]{\pagemark}
	\ifoot[WS 2013/14]{WS 2013/14}
	\ohead[Literaturverzeichnis]{Literaturverzeichnis}
	\printbibliography

\end{document}
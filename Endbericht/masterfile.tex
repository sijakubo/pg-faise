\documentclass[	11pt,		% 11 Punkte hohe Schrift
				a4paper,	% Einstellung des Ausdrucks auf DIN A4
				fleqn,		% linksbündige, abgesetzte Formeln
				reqno,		% rechts stehende Formelnummer
				BCOR12mm,	% Bindekorrektur von 12mm
				chapterprefix=false,	% Ausgabe von Kapitel: 
				appendixprefix=true, % Überschrift im Anhang X und dann neue Zeile und Titel
				parskip=half,
				DIV=calc,
				final,		% Abgabe Version
]{scrartcl}
%\usepackage{natbib}
\usepackage{packages}
\usepackage{layout}
\usepackage{hyperref}
\usepackage[ngerman,bookmarks=true,bookmarksopen=true,bookmarksnumbered=true]{hyperref}
\usepackage{listingsutf8}
\usepackage{caption}
\graphicspath{{image/}}
\DeclareGraphicsExtensions{.png,.pdf}

\bibliography{src/literatur}	% Literaturverzeichnis

\newtheorem{defi}{Definition}
\newtheorem{bsp}{Beispiel}
\newtheorem{satz}{Satz}
\newtheorem{theo}{Theorem}

\newcommand{\bspautorefname}{Beispiel}
\newcommand{\defiautorefname}[1]{Definition #1}
\newcommand{\theoautorefname}{Theorem}

\newcommand{\RM}[1]{\MakeUppercase{\romannumeral #1{}}} 

\setcounter{tocdepth}{4}
\setcounter{secnumdepth}{4} 
\lstset{inputencoding=utf8/latin1}
\lstdefinestyle{customc}{
  belowcaptionskip=1\baselineskip,
  breaklines=true,
  linewidth=1\textwidth,
  %frame=L,
  xleftmargin=\parindent,
  language=C,
  numbers=left,
  tabsize=3, 
  showstringspaces=false,
  basicstyle=\footnotesize\ttfamily,
  keywordstyle=\bfseries\color{green!40!black},
  morekeywords={uint8_t,uint16_t,int8_t,PROCESS,PROCESS_THREAD, PROCESS_BEGIN, PROCESS_END, AUTOSTART_PROCESSES},
  commentstyle=\itshape\color{purple!40!black},
  identifierstyle=\color{blue},
  stringstyle=\color{orange},
}

\DeclareCaptionFont{white}{ \color{white} }
\DeclareCaptionFormat{listing}{
  \colorbox[cmyk]{0.43, 0.35, 0.35,0.01 }{
    \parbox{\textwidth}{\hspace{15pt}#1#2#3}
  }
}
%\captionsetup[lstlisting]{ format=listing, labelfont=white, textfont=white, singlelinecheck=false, margin=0pt, font={bf,footnotesize} }

\begin{document}
	% Cover
	\begin{titlepage}
  \begin{centering}
  \begin{figure}[h!]
    \centering
    \includegraphics[width=310pt]{UniLogo}
  \end{figure}

  \vspace*{-0.8cm}

  \begin{figure}[h!]
    \centering
    \includegraphics[width=280pt]{BUIENGLOGO}
  \end{figure}

  \vspace*{0.4cm}
  
  \textsf{\Huge \textbf{Projektgruppe FAISE\\}}

  \vspace*{0.5cm}
  \noindent Endbericht\\
  im Rahmen des Masterstudiums

  \end{centering}
  
  \vspace*{1.5cm}
  \begin{tabbing}
  xxxxxxxxxxxxxxxx\= \kill
  
  \small Betreuer: \>Prof. Dr.-Ing. Jürgen Sauer\\
  \small \>Dipl.-Ing. (FH) Arne Stasch\\
  \small \>Dipl.-Inform. Jan-Hinrich Kämper\\
  \small \>Prof. Dr.-Ing. Axel Hahn\\\\

  \small Vorgelegt von: \>Berthe Pulcherie Ongnomo\\
  \small \>Chancelle Merveille Tematio Yme\\
  \small \>Christopher Schwarz\\
  \small \>Jan Paul Vox\\
  \small \>Jan-Gerd Meß\\
  \small \>Jannik Fleßner\\
  \small \>Malte Falk\\
  \small \>Matthias Aden\\
  \small \>Michael Goldenstein\\
  \small \>Nagihan Aydin\\
  \small \>Raschid Alkhatib\\
  \small \>Simon Jakubowski\\\\

  \small Abgabetermin:\> 30. September 2014
  \end{tabbing}
\end{titlepage}
	\pagenumbering{Roman}
	% Abstract
	\include{src/abstract}

	% Table of contents
	\clearpage
\ohead[Inhaltsverzeichnis]{Inhaltsverzeichnis}
\chead[Uni Oldenburg]{Uni Oldenburg}
\ihead[PG FAISE]{PG FAISE}
\setheadtopline{1pt}
\setheadsepline{0.5pt}

\ofoot[Endbericht]{Endbericht}
\cfoot[\pagemark]{\pagemark}
\ifoot[30. September 2014]{30. September 2014}
\setfootsepline{0.5pt}
\setfootbotline{1pt}

\tableofcontents
\clearpage

\ohead[Abbildungsverzeichnis]{Abbildungsverzeichnis}
\chead[Uni Oldenburg]{Uni Oldenburg}
\ihead[PG FAISE]{PG FAISE}
\setheadtopline{1pt}
\setheadsepline{0.5pt}

\ofoot[Endbericht]{Endbericht}
\cfoot[\pagemark]{\pagemark}
\ifoot[30. September 2014]{30. September 2014}
\setfootsepline{0.5pt}
\setfootbotline{1pt}

\listoffigures
\listoftables
	
	% Chapter 1 Einleitung
	\pagenumbering{arabic}
	\clearpage
	\ohead[Einleitung]{Einleitung}
	\chead[Uni Oldenburg]{Uni Oldenburg}
	\ihead[PG FAISE]{PG FAISE}
	\setheadtopline{1pt}
	\setheadsepline{0.5pt}
	\ofoot[Endbericht]{Endbericht}
	\cfoot[\pagemark]{\pagemark}
	\ifoot[31. Oktober 2014]{31. Oktober 2014}
	\setfootsepline{0.5pt}
	\setfootbotline{1pt}
	\section{Einleitung}
<<<<<<< HEAD
=======
In diesem Kapitel wird ein einführender Überblick über die Projektgruppe Fully Autonomous Intralogistic Swarm Experiments gegeben, die im Rahmen der Masterstudiengänge Informatik und Wirtschaftsinformatik in der Abteilung Systemanalyse und -optimierung der Carl von Ossietzky Universität Oldenburg stattgefunden hat. Das Projekt lief über einen Zeitraum von zwei Semestern: Wintersemester 2013/2014 und Sommersemester 2014.


\subsection{Motivation}
Im Zeitalter der Globalisierung werden hohe Anforderungen an die Leistungsfähigkeit von modernen Intralogistiksystemen gestellt. Neben einem hohen Automatisierungsgrad wird gleichzeitig auch eine möglichst hohe Flexibilität gefordert, da sich Anforderungen im logistischen Umfeld häufig ändern (Vgl.\cite{ieft}).    
\\\\
Stetigförderer bieten die Möglichkeit einen automatisierten Materialfluss einzurichten. Es handelt sich dabei um Transportsysteme, die Güter kontinuierlich und automatisiert entlang eines festgelegten Transportwegs befördern (Vgl.\cite{stf}). Ein solches System könnte beispielsweise ein Netz von Schienen sein. Nachteile dieser Systeme sind insbesondere Unflexibilität und schlechte Skalierbarkeit. Ändern sich Anforderungen in einem Logistiksystem, dann stoßen Stetigförderer schnell an ihre Grenzen. Transportwege sind festgelegt und können nicht ohne einen gewissen Aufwand geändert werden. Auch kann die Anzahl an Gütern, die pro Zeiteinheit befördert werden kann, nicht ohne eine Änderung am Transportnetz maximiert werden.
\\\\
Eine Alternative zu Stetigförderern sind Fahrerlose Transportsysteme (FTS). FTS sind ein Gesamtsystem aus Fahrerlosen Transportfahrzeugen, die Ware automatisiert befördern, und der Infrastruktur, die zum Betrieb der Transporteinheiten notwendig ist (Vgl.\cite{fts}). Fahrerlose Transportsysteme sind wesentlich flexibler als Stetigförderer. Müssen mehr Güter befördert werden, so können zusätzliche Transporteinheiten aktiviert werden. Folglich sind FTS problemlos skalierbar und können schnell auf veränderte Anforderungen in einem Intralogistiksystem eingestellt werden. FTS bieten einen automatisierten Warenfluss bei gleichzeitig hoher Flexibilität und entsprechen somit den Anforderungen, die an moderne Intralogistiksysteme gestellt werden.  
\\\\
Es bietet sich an ein System zu entwickeln, das basierend auf FTS, einen vollautomatisierten Warenfluss implementiert, um verschiedene Fragestellungen zu untersuchen. Wie muss ein solches System aufgebaut sein, welche Kommunikationsabläufe sind zwischen den verschiedenen Akteuren notwendig, welche Anforderungen werden an Hard- und Software gestellt und wie flexibel ist ein solches System?  

\subsection{Zielsetzung}
Im Rahmen der Projektgruppe FAISE soll ein System entwickelt werden, das den vollautomatisierten Warenfluss in einem Lager auf Basis von Fahrerlosen Transportsystemen simuliert. Dabei sollen die Transporteinheiten nicht zentral gesteuert werden, sondern dezentral als Schwarm agieren.
Das Gesamtsystem besteht aus zwei Teilsystemen, einem physisch vorhandenem System und einer softwarebasierten Simulation. 
\\\\
Das physische System beinhaltet Fahrerlose Transporteinheiten und Lagerrampen, die miteinander über ein Sensornetzwerk kommunizieren und deren Steuerung auf Basis von Mikrocontrollern erfolgt. Ziel ist es den Materialfluss von den Transporteinheiten und Rampen vollständig autonom und ohne dezentrale Steuerung durchzuführen. 
\\\\
Das rein softwarebasierte System implementiert ebenfalls einen automatisierten Warenfluss. Die Software soll als Abbild des physischen Systems realisiert werden. Die Akteure, ihre physikalischen Eigenschaften (Geschwindigkeit etc.) und ihr Verhalten im Einzelnen sowie als Schwarm sollen in der Software abgebildet werden. Beide Systeme laufen unabhängig voneinander und sollen in einem festen Einsatzszenario erprobt werden.

\subsection{Einsatzszenario}
Das Einsatzszenario besteht aus n fahrerlosen Transporteinheiten in einem Umschlagslager. Zusätzlich sind m Rampen verfügbar an denen Pakete zwischengelagert werden können. Im Gegensatz zu einem herkömmlichen Lager, werden Waren in einem Umschlagslager nur kurzfristig gelagert, um anschließend weitertransportiert zu werden. Es herrscht ein kontinuierlicher Materialfluss. Jedes Paket, das ins Lager gebracht wird, ist eindeutig identifizierbar und wird zu einem definierten Zeitpunkt ins Lager gebracht und wieder abgeholt. Die Rampen im Lager sollen drei unterschiedliche Zwecke erfüllen. Eingangsrampen dienen der Warenannahme, Zwischenrampen der Zwischenlagerung. Pakete werden zum Ausgangslager gebracht und zum Zwecke des Weitertransports dort abgeholt. Auf Basis des Einsatzszenarios wird im Rahmen der Anforderungen ein Ablaufszenario erstellt, das die Interaktionen zwischen den Akteuren beschreibt auf deren Basis eine automatisierte, dezentrale Abwicklung des Materialflusses erfolgen kann.

\subsection{Komponenten}
  

>>>>>>> 8effb45eed680862593a1536e98c8063b0896e9a

	
	% Chapter 2 Stand der Technik
	\clearpage
	\ohead[Stand der Technik]{Stand der Technik}
	\chead[Uni Oldenburg]{Uni Oldenburg}
	\ihead[PG FAISE]{PG FAISE}
	\setheadtopline{1pt}
	\setheadsepline{0.5pt}
	\ofoot[Endbericht]{Endbericht}
	\cfoot[\pagemark]{\pagemark}
	\ifoot[31. Oktober 2014]{31. Oktober 2014}
	\setfootsepline{0.5pt}
	\setfootbotline{1pt}
	\section{Stand der Technik}
Die Fahrerlosen Transportsysteme und die Materialflusssysteme sind Prozesse der Logistik. In den vergangenen Jahren hat die Verbreitung Fahrerloser Transportsysteme (FTS) stark zugenommen. Beim Einsatz von FTS stellen sich vielfältige Konfigurierungs- und Planungsprobleme, so auch die Einsatzplanung f\"ur die einzelnen Fahrerlosen Transportfahrzeuge. (vgl. Günther; Krüger; Schrecker; 2000, S. 2). Der innerbetriebliche Materialfluss von Industrieunternehmen bietet fahrerlosen Transportsystemen (FTS) zahlreiche Einsatzgebiete: Sie verketten Produktionsprozesse, verknüpfen Fertigungsstationen oder ganze Betriebsbereiche und beschicken Montageplätze. Darüber hinaus dienen sie als mobile Werkbank oder versorgen und entsorgen Lager unterschiedlicher Art. Um die Systemvorteile von Fahrerlosen Transportsystemen und Materialflusssystemen zu optimieren, braucht man ein ma\ss gerechtes Wissen auf Ihr spezifisches Anlagekonzept abzustimmen. Wichtige Kriterien sind allerdings z.~B. die Einbindung der Fahrerlosen Transportsysteme in den gesamtbetrieblichen Materialfluss, die Anpassung an die vorhandenen Steuerungshierarchien und die optimale Auslegung der Technik in Bezug auf Fahrzeugbauart, Lastaufnahmemittel, Energiekonzept, Kommunikation und Leitsystem. (vgl. Werner Swoboda, Industrie Anzeiger). Ziele von Fahrerlosen Transportsystemen und Materialflusssystemen sind Kostensenkung durch Personaleinsparung, Verringerung von Transportsch\"aden, hohe Zuverl\"assigkeit in Vorg\"angen und bessere Materialflussplanung.
Dieses Kapitel wird in drei Teile gegliedert. Der erste Teil wird die Fahrerlosen Transportsysteme bzw die Orientierungs- und die Steuerungssysteme vorstellen und erkl\"aren; der zweite Teil ist eine Darstellung der Materialflusssysteme und ihrer verschiedenen Funktionen und der dritte Teil wird erkl\"aren, wie fahrerlose Transport- und Materialflusssysteme in gro\ss en Firmen wie Volkswagen und BMW Anwendung finden.

\subsection{Fahrerlose Transportsysteme}
Nach dem Verein Deutscher Ingenieure 2510 bestehen FTS im Wesentlichen aus "`einem oder mehreren Fahrerlosen Transportfahrzeugen (FTF), einer Leitsteuerung, Einrichtung zur Standortbestimmung und Lagererfassung, Einrichtungen zur Daten\"ubertragung sowie Infrastruktur und peripheren Einrichtungen"'. In seinem Buch Transport und Lagerlogistik fasst Martin die Definition von VDI 2510 eines FTS zusammen. Er beschreibt ein FTS als mit FTF ausgestattete rechnergesteuerte Materialflussanlagen zum automatischen Transport von G\"utern im innerbetrieblichen Materialfluss. (vgl. Martin H, 2006, S.262f). Bei FTS handelt es sich um flurgebundene F\"ordersysteme mit automatisch gef\"uhrten FTF. Die einzelnen FTF bef\"ordern Ladungstr\"ager zwischen zwei oder mehrere Stationen innerhalb eines Gebietes. Die Fahrzeugsteuerung erfolgt automatisch und rechnergest\"utzt. Der Einsatzbereich von FTS ist generell \"uberwiegend innerbetrieblich ausgerichtet. In diesen Rahmen \"ubernehmen FTS sowohl reine F\"orderaufgaben, wie Verkettung von Fertigungs- und Montageeinrichtungen als auch Aufgaben der Lagerbedienung und Kommissionierung. (vgl. G\"unther; Kr\"uger; Schrecker; 2000, S. 3). Das FTS ist eine Technik, die im Vergleich gegen\"uber Stetigf\"ordersystemen eine hohe Anpassungsf\"ahigkeit an die sich \"andernden Marktsituationen zum Vorteil hat. Daher konzentrieren die Forschungs- und Entwicklungsaktivit\"aten sich heutzutage auf die sog. „Zellul\"aren F\"ordersysteme“, in welchen stetige F\"orderanlagen zur Verkn\"upfung von Logistischen Funktionen durch individuelle, autonom arbeitende FTF ersetzt werden (vgl. Ten Hompel; Heidenblut, 2008). Die Haupteinsatzgebiete des FTS liegen nun in der Intralogistik. Also bei der Organisation, der Steuerung, der Durchf\"uhrung und der Optimierung des innerbetrieblichen Waren- und Materialflusses und Logistik, der Informationsstr\"ome sowie des Warenumschlags in Industrie, Handel und \"offentlichen Einrichtungen. Z.~B. Automobil- und Zulieferindustrie, Papiererzeugung und –verarbeitung, Elektroindustrie, Getr\"anke-, Lebensmittelindustrie, Baustoffe, Stahlindustrie, Kliniklogistik (G\"unter Ullrich, 2011 S. 13). FTS bestehen im Wesentlichen aus drei Systemkomponenten: Die Fahrerlosen Transportfahrzeuge, das Orientierungssystem, das Steuerungssystem. 

\subsection{Fahrerlose Transportfahrzeuge}
Die FTF sind flurgebundene F\"ordermittel mit eigenem Fahrantrieb, die automatisch gef\"uhrt, gesteuert und ber\"uhrungslos gef\"uhrt werden. Sie dienen dem Materialtransport, und zwar zum Ziehen und/oder Tragen von F\"ordergut mit aktiven oder passiven (FTF mit passiver Lastaufnahme werden von anderen F\"ordermitteln gezogen oder manuell mit den G\"utern best\"uckt) Lastaufnahmemittel (VDI 2510). Da das FTS mit fahrerlosen Aspekten systematisiert ist, ergeben sich dann auf der funktionalen Ebene Unterschiede zu fahrerbedienten Fahrzeugen, wie z.~B. den klassischen Gabelstaplern und FTF. Im Rahmen dieser Arbeit wird nur auf eine Kategorie von FTF tiefer eingegangen: das Mini-FTF. Die Mini-FTF sind kleine, schnelle, intelligente und flexible Fahrzeuge, die extrem schnell Bed\"urfnisse befriedigen k\"onnen. Heutzutage arbeiten viele Universit\"aten in der ganzen Welt im Bereich der Schwarm-Experimente. Hier sollen die kleinen FTF intelligent miteinander arbeiten. Die Fahrzeuge sollen sich ohne eine eigene separate FTS-Leitsteuerung untereinander verst\"andigen, Strategien entwickeln und gemeinsam Arbeiten ausf\"uhren. Die Forschungsgebiete hei\"ssen Agentensysteme und Schwarmtheorie. Die Mini-FTF k\"onnen nur intralogistische Aufgaben ausf\"uhren. Dennoch sind viele unkonventionelle Einsatzf\"alle denkbar. Die Kommissionierung (eine ausf\"uhrliche Begriffserkl\"arung wird im Teil Materialfluss gegeben) ist die verbreitetste Anwendungsm\"oglichkeit von Mini-FTF (G\"unter Ullrich, 2011 S. 105).
Als Zusammenfassung kann man sagen, dass die Fahrzeugsteuerung die Systemsicherheit, das Energiemanagement, das Lastaufnahmemittel und die Lenkung eines FTF gew\"ahrleistet. Ein FTF kann ohne Energie nicht funktionieren. Damit ein FTF seine Aufgabe erf\"ullen kann, ist eine Energieversorgung notwendig. Die FTF können durch Akkus oder Traktionsbatterien oder mit Hilfe eines Induktionssystems oder einer Stromschiene mit Energie versorgt werden. Jedoch k\"onnen die beiden Versorgungsarten gekoppelt werden, um ein Hybridsystem zu bekommen. Die Notwendigkeit der Existenz einer Ladestation in einem FTS ist unumstritten. Die FTF m\"ussen immer mit Energie versorgt werden. Je nachdem wie die FTF programmiert sind, kann ein FTF bei Energiebedarf selber zur Ladestation fahren oder von einem Auftraggeber (Mensch) zur Ladestation gef\"uhrt werden.

\subsubsection{Orientierungssystem bzw. Navigation}
Das Orientierungssystem bzw. die Navigation dient zur Lokalisierung des Fahrzeugs. Sie ist ein Hilfsmittel zur Berechnung des sichersten Wegs, um das Ziel zu erreichen. Au\ss erdem dient die Navigation auch zur Vermeidung von eventuellen Kollisionen. Sie gilt sowohl f\"ur die Orientierung als auch f\"ur die Sicherheit des Fahrzeuges und seines Umfeldes. W\"ahrend seiner Bewegung bzw. Orientierung folgt das FTF einer physischen oder virtuellen Linie (Spur), damit es sein Ziel gefahrlos erreichen kann. Aufgrund eines Sicherheitssystems sollte das FTF bei Kollisionsgefahr oder wenn Hindernisse vor ihm stehen, sofort anhalten. 
Unter Navigationshilfe versteht man nicht nur die Positionierung und Orientierung des Fahrzeuges sondern auch, wohin das Fahrzeug gelangen w\"urde, wenn keine auf seine Bewegung ver\"andernden Ma\ss nahmen ergriffen würden.
Die Steuerung sagt, was zu tun ist, und die Navigation bestimmt, auf welchem Weg das gew\"unschte Ziel sicher zu erreichen ist bzw. ob das FTF einen vorgegebenen Weg weiter verfolgen oder einen alternative Weg nehmen soll.
Die Steuerung von fahrerlosen Transportfahrzeugen, deren Grundfunktionen und der Umgang mit diesen werden in den VDI-Richtlinien [VDI92], [VDI94], [VDI04] vorgestellt.
F\"ur das Konstrukt der fahrerlosen Transportsysteme werden verschiedene Ans\"atze verfolgt, die abh\"angig vom System verschiedene Konstruktionsbem\"uhungen auf das Fahrzeug oder auf der Strecke erfordern.
Es gibt mehrere Navigationsverfahren: die physische Leitlinie, die Orientierung durch Magnetmarken, das Global Positioning System (GPS) und die Lasernavigation (vgl. G\"unter Ullrich, 2011 S. 112).
\begin{itemize}
	\item \textbf{Die physische Leitlinie:}  Fahrerlose Transportsysteme, die auf physischen Leitlinien navigieren bzw. fahren, benutzen Einrichtungen am oder im Fu\ss boden. Die verschiedenen Varianten sind:
 \item \textbf{Orientierung durch optische Leitspur:} Bei dieser Methode wird ein Farbstrich mit deutlichem Farbkontrast zum umgebenden Boden entweder lackiert oder mit einem speziellen Gewebeband aufgebracht.
Eine geeignete Kamerasensorik unter dem Fahrzeug nutzt Kantendetektions-Algorithmen und errechnet so die Ansteuerungssignale f\"ur den Lenkmotor (G\"unter Ullrich, 2011 S. 112).
Optische Verfahren dienen durch eine st\"andige Kurskorrektur dazu, eine hohe Fahrgenauigkeit zu erreichen. 
\item \textbf{Orientierung durch induktive Leitspur:} Diese Methode der Navigation fahrerloser Transportfahrzeuge ist profitabel aufgrund der permanenten Kurskorrektur und au\ss erdem besonders zuverl\"assig und fahrzeugseitig durch die Nutzung einfacher Komponente zu realisieren.
Es ist m\"oglich, die Stromversorgung der Fahrzeuge fahrbahnseitig zu realisieren, sodass die Nutzung schwerer Akkumulatoren entf\"allt.
Jedoch sind Systeme mit Leitdrahtsteuerung nicht flexibel und in der Konstruktion sehr teuer.
\begin{figure}[h!]
	\centering
		\includegraphics[width=0.9\textwidth]{Prinzipskizze_induktiven.jpg}
	\caption{Prinzipskizze zur induktiven und optischen Spurf\"uhrung (Quelle: G\"unter Ullrich, 2011 S. 79)}
	\label{Prinzipskizze_induktiven}
\end{figure}  
	\item \textbf{Orientierung durch Magnetmarken:} Eine weitere M\"oglichkeit der Steuerung ist die Abtastung von Magnetstreifen oder magnetischen Markierungen auf der Straßenoberfl\"ache.
Dabei bedarf es zur Berechnung der Leitlinie entweder der Koppelnavigation, oder der Peilung von in regelm\"a\ss igen Abst\"anden in den Boden eingelassenen Marken.
Diese Marken k\"onnen rein passive Dauermagnete oder aber quasi-aktive Transponder sein (G\"unter Ullrich, 2011 S. 80).
Das Bild \ref{Prinzipskizze_Koppelnavigation_rechts} ist eine Repr\"asentation der Navigation durch Magnetstreifen.
	\begin{figure}[h!]
		\centering
			\includegraphics[width=0.9\textwidth]{Prinzipskizze_Koppelnavigation_rechts.jpg}
			\caption{Prinzipskizze zur Koppelnavigation (links) und zur Magnet- bzw. Transpondernavigation (rechts) (Quelle: G\"unter Ullrich, 2011 S. 79)}
			\label{Prinzipskizze_Koppelnavigation_rechts}
	\end{figure}	
	\item Bei der Lasernavigation bestimmt der Laserscanner die Position des FTF, dazu kommen noch optische Sensoren f\"ur die Erkennung von Hindernissen wie z.~B. Menschen.
Lasergef\"uhrte FTS bieten einen hohen Wert an Flexibilit\"at, da sie ohne Bodeninstallation funktionieren.
Nur bei engerem Raum kann die Lasernavigation nicht so effizient wie z.~B. eine induktive Spurf\"uhrung sein, wenn viele Fahrzeuge zum Einsatz kommen.
Um die Systemvorteile einer Lasernavigation optimal zu nutzen, ben\"otigt man allerdings ein passendes Anlagenkonzept.
Die wichtigsten Kriterien sind: die Einbindung in das gesamtbetriebliche Materialflusssystem, die Anpassung an die vorhandenen Steuerungshierarchien und die optimale Auslegung der Technik in Bezug auf Fahrzeugbauart, Lastaufnahmemittel, Energiekonzept, Kommunikation und Leitsystem.
Ein Aspekt, der f\"ur das Laser-gef\"uhrte FTS spricht, ist die Wirtschaftlichkeit.
Und dies trotz der Alternativen Elektro-, Low-Cost- sowie induktiv gef\"uhrten FTS.
Letztere lassen sich so einrichten, dass sie auch auf leitdrahtlosen, rein rechnergef\"uhrten Teilstrecken verkehren k\"onnen.
Keinerlei kostenintensive Bodeninstallation ben\"otigt dagegen das \"uber Lasersensor gesteuerte, v\"ollig frei navigierende Laser-FTS.
Die Fahrzeuge orientieren sich lediglich an im Raum verteilten Reflektoren und mit Hilfe der Kombination von Winkel- und Distanzmessung. (Werner Swoboda, Industrie Anzeiger). Das Bild \ref{Prinzipskizze_Koppelnavigation_links} ist eine Visualisierung der Lasernavigation.
	\begin{figure}[h!]
		\centering
		\includegraphics[width=0.8\textwidth]{Prinzipskizze_Koppelnavigation_links.jpg}
		\caption{Prinzipskizze zur Koppelnavigation (links) und zur Magnet- bzw. Transpondernavigation (rechts) (Quelle: G\"unter Ullrich, 2011 S. 79)}
		\label{Prinzipskizze_Koppelnavigation_links}
	\end{figure}

	\item \textbf{Orientierung durch GPS:} Seine Anwendung im Bereich der Fahrzeugsteuerung wird in Form des DGPS eingesetzt.
DGPS bedeutet differential GPS und meint die Verwendung eines zus\"atzlichen GPS-Empf\"angers, der nicht auf dem FTF, sondern station\"ar fest installiert ist.
Mit Hilfe dieses ortsfesten GPS-Empf\"angers wird der sich zeitlich \"andernde Fehler ermittelt, der dem GPS-System eigen ist.
Mit Hilfe dieser Kenntnis k\"onnen zeitgleich die fahrenden GPS-Empf\"anger auf den FTF exakte Positionen ermitteln (Quelle: G\"unter Ullrich, 2011 S. 27).
Diese Navigationstechnik benötigt einen freie Sichtkegel von 15 Grad nach oben (siehe Bild 4), um zuverl\"assig arbeiten zu k\"onnen.
Die Schritte zur Erlangung der erforderlichen Fahr- und Positioniergenauigkeit sind:
	\begin{itemize}
		\item Pr\"ufung der \"ortlichen Gegebenheiten, insb. der Empfangsst\"arken der Satelliten
 \item Einsatz des Differential-GPS
 \item Real Time Kinematic Differential GPS. 
\end{itemize}
	\begin{figure}[h!]
		\centering
		\includegraphics[width=0.9\textwidth]{Prinzipskizze_zur_Navigation_mittels_GPS.jpg}
		\caption{Prinzipskizze zur Koppelnavigation (links) und zur Magnet- bzw. Transpondernavigation (rechts) (Quelle: G\"unter Ullrich, 2011 S. 79)}
		\label{Systemarchitektur_FTS}
\end{figure}
Im Rahmen des Projekt FAISE wird die Navigation durch den Laser durchgef\"uhrt. Es kann hier kein Global Positioning System (GPS) verwendet werden, da das ganze Experiment in einem geschlossenen Raum gemacht wird.
Weiterhin wird auch keine Navigation durch die physische Leitlinie oder durch die St\"utzpunkte im Boden erzielt, weil dazu der Boden gebrochen werden m\"usste.
\end{itemize}

\subsubsection{Steuerungstechnik}
Die interne Materialflusssteuerung ist eine Vorstufe der Transportauftragsabwicklung und wird nur dann ben\"otigt, wenn die Transportauftr\"age nicht klar dezidiert \"ubertragen, sondern aufbereitet werden m\"ussen. Eine Anforderung wie z. B. ben\"otige Ware A an Maschine B erfordert eine Umsetzung in einen oder mehrere Transportauftr\"age nach dem klassischen Muster. Hole von C und Bringe nach D. Die FTS-interne Materialflusssteuerung kombiniert also Quelle und Senke \"uber die in ihr hinterlegten Transportbeziehungen zu einem Transportauftrag und schickt diesen zur Durchf\"uhrung an die Transportauftragsverwaltung. Diese ganze Transportauftragsverwaltung ist in der FTS-Leisteuerung geregelt. 
Die FTS-Leitsteuerung ist die Kommandozentrale, um das FTS in das Umfeld zu integrieren. Au\"sserdem steuert es die FTF, die sich im System befinden. Damit ist das FTS dann in der 
Lage, die ihm \"ubertragenen Auftr\"age zu erf\"ullen. "`Eine FTS-Leitsteuerung besteht aus Hard- und Software. Kern ist ein Computerprogramm, das auf einem oder mehreren Rechnern abl\"auft. Sie dient der Koordination mehrerer Fahrerloser Transportfahrzeuge und/oder \"ubernimmt die Integration des FTS in die innerbetrieblichen Abl\"aufe."' (VDI 4451). Die Leitsteuerung bringt das FTS in seinem Umfeld zusammen, bietet seinen Bedienern vielf\"altige Service-M\"oglichkeiten und nimmt Transportauftr\"age entgegen. Weiterhin stellt sie den Aufgaben entsprechende Funktionsbl\"ocke zur Verf\"ugung. 
Die FTS-Leitsteuerung ist der Kern der FTS. In Rahmen des Projekt FAISE, wird es auch eine Leisteuerung ben\"otigt. Eine Leitsteuerung ist nur mit Hilfe eine Systemarchitektur zu implementieren und zu verstehen. In seinem Buch Fahrerlose Transportsysteme, hat G\"unter Ulrich zwei verschiedene Systemarchitekturen dargestellt. Eine f\"ur eine einfache FTS und eine andere f\"ur eine komplexe FTS. Da es bei FAISE nur mit vier FTF gearbeitet wird, ist es sinnvoll mit einer einfachen Systemarchitektur zu arbeiten. Das Bild 3 ist eine Repr\"asentation einer einfachen Systemarchitektur.
	\begin{figure}[h!]
		\centering
		\includegraphics[width=0.9\textwidth]{Systemarchitektur_FTS.jpg}
		\caption{Die Systemarchitektur eines einfachen FTS (Quelle: G\"unter Ullrich, 2011 S. 93)}
		\label{Systemarchitektur_FTS}
	\end{figure}

Es gibt eine geringe Anzahl von FTF, mit denen die Leitsteuerung per WLAN in Verbindung ist. Au\"sserdem gibt es ein LAN, \"uber das es eine direkte Verbindung mit einem \"ubergeordneten Rechner gibt, von dem die Transportauftr\"age kommen. \"uber die angedeutete Telefonleitung ist eine VPN-Verbindung zur Ferndiagnose eingerichtet. Die Daten\"ubertragung zu den \"ubergeordneten Host-Rechnern erfolgt meist \"uber lokale, Ethernet basierte Netzwerke mit dem Protokoll TCP/IP. Solche Host-Rechner k\"onnen beispielweise Materialflusssteuerungssysteme zur Produktionssteuerung (z. B. SAP) Produktionsplanungssysteme (PPS) Lagerverwaltungssysteme (LVS) sein.“( vgl. G\"unter Ullrich, 2011 S. 96). 
Au\"sserdem nach der VDI 4451(Blatt 3) „zum internen Umfeld der FTF-Steuerung geh\"oren das Lastaufnahmemittel (LAM), Sensoren und Aktoren, Bedienfeld am Fahrzeug und das Sicherheitssystem. Das externe Umfeld besteht aus der FTS-Leisteuerung, anderen FTF, automatischen Stationen und Geb\"audeeinrichtungen“. Die Abbildung 1 stellt eine Darstellung eine FTF-Steuerung und ihr Steuerungsumfeld dar.
	\begin{figure}[h!]
		\centering
		\includegraphics[width=0.9\textwidth]{Systemarchitektur_FTS.jpg}
		\caption{Allgemeine Darstellung einer FTF-Steuerung mit Datenschnittstellen (vgl. VDI 4451)}
		\label{Wertschoepfungskette}
	\end{figure}
Die administrative Ebene, die h\"aufig \"uber einen station\"aren Leitrechner realisiert wird, verwaltet die Transportauftr\"age der ganzen Materialflusssteuerung. Die operative Ebene, die auch als Fahrzeugsteuerung bezeichnet wird, erh\"alt ihre Informationen \"uber die Fahrzeugdisposition der administrativen Ebene. Der Funktionsblock Kommunikation leitet den stattgefundenen Datenaustausch zum Manager weiter. Dieser sorgt f\"ur die Koordination, indem er die Fahrauftr\"age in einzelne Befehle aufteilt, sowie f\"ur ein reibungsloses Zusammenwirken der einzelnen Funktionsbl\"ocke. Neben dem Block Kommunikation sind weitere Bl\"ocke vorhanden. Dazu geh\"ort f\"ur die gesamte Last\"ubergabe inklusive der Lastlagererfassung verantwortliche Lastaufnahme, das Energiemanagement, welches den Lade- und Allgemeinzustand der Batterien \"uberwacht, und der Block \"uberwachung/Sicherheitsschnittstelle, welcher zum Schutz der Personen und Sachgegenst\"ande dient. Der Funktionsblock Fahren und die damit verbundene Sensorik bzw. Aktorik koordinieren die Ablaufsteuerung der Funktionen des Orientierungssystems (Langenbach Maik, 2012, S. 33).

\subsection{Materialflusssysteme}
Damit ein Produkt auf den Markt kommen kann, muss man ihn denken, ihn erstellen und dann ihn vermarken. Die Produkterstellung und -vermarktung sind Prozesse des Wirtschaftens. Vorprodukte oder Materialen werden von Beschaffungsm\"arkten in die Unternehmen gef\"uhrt und dort werden sie durch besondere Produktionsprozesse transformiert. Am Ende der Produktion, steht ein Endprodukt, der f\"ur den Konsum bereits ist. 
Die Produktion und Logistik von G\"utern sind daher sehr wichtige Bereiche f\"ur den Unternehmenserfolg. Allerdings f\"uhren heute die unterschiedlichen Auspr\"agungen der Logistik z.B. in Produktions-, Handels-, oder Verkehrsunternehmen zu einer terminologischen Differenzierung der Logistik. Der Materialflussbegriff leitet sich einfach von dem logistische Konzept ab, in anderen W\"ortern das Materialflusssystem f\"uhrt in der Logistik zur\"uck. Die Abbildung 2. dient zur Erl\"auterung einer konventionellen Wertsch\"opfungskette. 
	\begin{figure}[h!]
		\centering
		\includegraphics[width=0.9\textwidth]{Wertschoepfungskette.jpg}
	\caption{Elemente einer Wertsch\"opfungskette (vgl. Wulz, J, 2008, S. 7)}
	\label{Wertschoepfungskette}
\end{figure}

Der Begriff Materialfluss bedeutet die Verkettung aller Prozesse bei der Beschaffung, Bearbeitung, Verarbeitung sowie bei der Distribution von G\"utern innerhalb festgelegter Bereiche. Deswegen l\"asst sich der Materialfluss in vier Stufen unterordnet: externer Transport, betriebsinterner Materialfluss, geb\"audeinterner Materialfluss und Materialfluss am Arbeitsplatz. Nach dem Verein Deutscher Ingenieur bzw. VDI-241 beinhaltet die Logistik f\"unf Hauptfunktionen. Diese Funktionen sind Bearbeiten, Pr\"ufen, Handhaben, F\"ordern, Lagern und Aufenthalten. Neben diesen Hauptfunktionen z\"ahlen auch Nebenfunktionen wie z.B. Montieren, Umschlagen, Kommissionieren, Palettieren und Verpacken (VDI 2411). Jedoch ist auf der Ebene des Materialflusssystems nur drei Funktionen zu ber\"ucksichtigen: F\"ordern, Lagern, Handhaben. Die anderen Funktionen setzen sich normalerweise aus den erl\"auterten Funktionen zusammen. Dieses Arbeitsteil wird in zwei Teile gegliedert. Im ersten Teil werden die drei Funktionen der Materialflusssysteme vorgestellt Im zweiten Teil wird eine Planung von Materialflusssystemen dargestellt.

\paragraph{Funktionen von Materialflusssystemen}
\begin{itemize}
	\item \textbf{Funktion F\"ordern} \\
	F\"ordern bedeutet Transportieren und ist eine der wichtigsten Aspekte innerhalb des Materialflusssystems. Nach der VDI 2411 ist F\"ordern das Fortbewegen von Arbeitsgegenst\"anden in einem System. „Die Fortbewegung oder Ortver\"anderung von G\"utern oder Personen mit technischen Mitteln wird allgemein als Transport bezeichnet. Findet diese Ortsver\"anderung in einem r\"aumlich begrenzten Gebiet wie beispielsweise innerhalb eines Betriebes oder Werkes statt, so wird dieser Vorgang durch den Begriff F\"ordern pr\"azisiert. Das F\"ordern bzw. die F\"ordertechnik umfasst also das Bewegen von G\"utern und Personen \"uber relativ kurze Entfernungen einschlie\"sslich der dazu notwendigen technischen organisatorischen und personellen Mittel“(Ten Hompel, Schmidt, Nagel, 2007, S. 119). 
Das F\"ordermittel (technisches Transportmittel, zur Ortsver\"anderung von G\"utern oder Personen) und das F\"orderelement bilden das physikalische Bestandteil eines F\"ordervorgang. Der Ablauf und die Steuerung werden durch den F\"ordervorgang dargestellt. In Punkto F\"ordermittel kann auf verschiedenste Elemente der Materialflusstechnik zur\"uckgegriffen werden. Dies umfasst unter anderen Rollenbahnen, und FTS. Neben der M\"oglichkeit auf automatisierte F\"ordermittel zur\"uckzugreifen, kommen auch manuell mechanisierte bzw. rein manuelle Systeme zum Einsatz. In diesem Fall ist der Mensch oder der Bediener eines F\"ordermittels wesentlich f\"ur den Ablauf eines reibungslosen Materialflusses in Zusammenspiel mit den physikalischen Elementen sowie dem Prozessablauf verantwortlich. (Wulz, J, 2008, S. 8). Das Bild 4 gilt als Beispiel eines F\"ordersystems. 
	\begin{figure}[h!]
	\centering
  \includegraphics[width=0.7\textwidth]{Stetigfoerderer.jpg}
	\caption{Beispiel eines Stetigf\"orderer (entnommen aus Ten Hompel, Schmidt, Nagel, 2007, S. 131)}
	\label{Stetigfoerderer}
\end{figure}

\item \textbf{Funktion Lagern} \\
Das Lagern ist jedes geplante Liegen des Arbeitsgegenstandes im Materialfluss. Das Lager ist ein r\"aumlich abgegrenzter Bereich bzw. eine Fl\"ache zum Aufbewahren von St\"uck- und/oder Sch\"uttg\"utern in Form von Rohmaterialien, Zwischenprodukte oder Endprodukte, das mengenm\"a\"ssig erfasst wird (VDI-2411). Die Einlagerung von Lagereinheiten, die Aufbewahrung und Bereithaltung von Lagereinheiten auf Lagerpl\"atzen und die Auslagerung einer Lagereinheit, sind die grundlegenden Prozesse in einem Lager.
Aufgrund der starken Ver\"anderungen im Markt, m\"ussen auch die unternehmerischen Abl\"aufe an Lagersysteme schnell angepasst werden. In einem Lagersystem werden im Verlauf des Materialflusses Speicher- bzw. Lagerfunktionen sowie F\"orderfunktionen wahrgenommen. 
Aufgabe eines Lagers ist das Bevorraten, Puffern und Verteilen von G\"utern. W\"ahrend Vorratslager lang- und mittelfristige und Pufferlager kurzfristige Bedarfsschwankungen ausgleichen sollen, erf\"ullen Verteillager neben der Bevorratungs- noch eine Kommissionierfunktion. Daher k\"onnen die Aufgaben eines Lagers anhand folgender Ausgleichsma\"ssnahmen beschrieben werden: Zeitausgleich, Mengenausgleich, Raumausgleich und Sortimentsausgleich. (Stich, V.; Bruckner, A.; 2002). Ein Zeitausgleich ist immer dann erforderlich, wenn die Zeitfunktion der Nachfrage nicht der Zeitfunktion der Produktion entspricht. Beispielsweise steht eine losgr\"o\"ssenoptimierte Fertigung einer saisonalen Nachfrage gegen\"uber. Gerade in Bereichen mit Serienfertigung, in denen aus Kostengr\"unden in der Regel gr\"o\"ssere Mengen als die Nachfragemengen produziert werden, muss Mengenausgleich vollzogen werden. Sobald der Produktionsort nicht mit dem des Produktabnehmers \"ubereinstimmt, findet mit Hilfe von Verkehrstr\"agern ein Raumausgleich statt. Mit zunehmender Sortimentsbreite steigt die Wahrscheinlichkeit, dass die Anzahl der Produktionsstandardorte steigt. (Lagenbach, M, 2012, S. 14).

\item \textbf{Funktion Handhaben } \\
Der Begriff Handhaben wurde gedanklich von de menschlichen Hand abgeleitet, wird aber auch f\"ur automatische ablaufende Vorg\"ange zur Manipulation von Objekten gebraucht. Handhaben bedeutet etwas greifen, bewegen und an einem bestimmten Ort ablegen. Das hei\"sst, durch Handhaben wird die Lage oder Position von Objekten ge\"andert. Im \"ubertragenen Sinne bedeutet handhaben auch bewerkstelligen bzw. praktisch aus\"uben. Von Handhabungstechnik spricht man, wenn f\"ur die Handhabung Ger\"ate eingesetzt werden. 
Die Richtline VDI 2860 definiert die Funktion Handhaben als „das Schaffen, definiertes Ver\"andern oder vor\"ubergehendes Aufrechterhalten einer vorgegebenen r\"aumlichen Anordnung von geometrisch bestimmten K\"orpern.“ Die Teilfunktionen des Handhabens stellen das Speichern, das Bewegen, das Sichern, das Kontrollieren und das Ver\"andern von G\"utern dar. Das Handhaben kann sowohl als eine Funktion als auch eine Fertigung des Materialflusses betrachtet werden. Eine m\"ogliche Handhabungsfunktion im Materialfluss ist z.B. das Palettieren, worunter die Stapelung von St\"uckg\"utern zu einem St\"uckgutstapel nach einem gewissen Muster verstanden wird. Handhabungsfunktionen k\"onnen entweder von Automaten z.B. Roboter oder von Menschen durchgef\"uhrt werden. Auf Grund der Greifflexibilit\"at ist der Mensch jedoch meist un\"ubertroffen in der Handhabung.
\end{itemize}

\subsection{Fallbeispiele}
\subsubsection{FTS in der Gl\"asernen Manufaktur Dresden (Volkswagen)}
Volkswagen AG montiert das neue Modell der Luxusklasse "Phaeton" in der "Gl\"asernen Manufaktur" in Dresden. Die Materialversorgung \"ubernimmt ein fahrerloses Transportsystem mit 56 frei navigierenden Fahrzeugen. Die gesamte Steuerungs- und Navigationstechnik stammt von FROG Navigation Systems, dem Projektpartner des Generalunternehmers AFT (Mechanik).
Die Produktion ist auf drei Ebenen unterteilt: . Die eigentliche Montage findet auf den beiden oberen Montageebenen statt: Die Rohrkarosse befindet sich auf einer Montageplattform, die Teil des Schuppenbandes ist, das sich sicher in den Hallenboden einf\"ugt und mit konstanter Geschwindigkeit durch die Montagezyklen bewegt. Danach erfolgt die \"ubergabe an eine schwere Elektroh\"angebahn (EHB) zur H\"angemontage. W\"ahrend der H\"angemontage erfolgt die Hochzeit, d. h. das Zusammenf\"ugen von Karosse und Triebsatz, wobei der Triebsatz von einem Fahrerlosen Transportfahrzeug (FTF) herangebracht wird. Anschlie\"ssend wird die Karosse wieder auf eine Schubplattform, die sog Schuppe, zur Komplettierung und Qualit\"atskontrolle gestellt.

Im Untergeschoss, der Logistikebene, wird die verbauende Ausr\"ustung zur Verf\"ugung gestellt und in Betrieb genommen. Die FTS \"uberminnt die Versorgungsleitungen der Materialien und damit eine erhebliche logistische Funktion . Um zwischen den Ebenen zu wechseln, nutzen die  automatischen Fahrzeug-Hebeb\"uhnen .
Das FTS hat die grunds\"atzliche Aufgabe, die Montagelinien (Schuppenband oder EHB) zu versorgen. Dabei wird allerdings zwischen folgenden sechs Gewerken unterschieden:
\begin{itemize}
\item[1.] Anlieferung von Warenk\"orben auf die Schuppe
\item[2.] Anlieferung von Schalttafeln (Cockpits)
\item[3.] Anlieferung von Kabelstr\"angen
\item[4.] Anlieferung des Triebwerks mit Fahrwerk und Ausf\"uhrung der Hochzeit
\item[5.] Anlieferung von Warenk\"orben zur H\"angemontage
\item[6.] Anlieferung der T\"uren plus Warenk\"orbe
\end{itemize}
\subsubsection{FTS beim Automobilhersteller BMW im Werk Leipzig}
Das BMW-Werk in Leipzig hat im Jahre 2005 mit der Produktion der 3er reihe (E90) gestartet
Im Bereich der Teileversorgung \"ubernimmt erstmals in der Geschichte der Automobilindustrie ein Fahrerloses Transportsystem (FTS) umfangreiche Logistikfunktionen. Folgende Prozesse wurde f\"ur die Teilversorgung im Leipzig-Werk definiert:

\begin{itemize}
\item Direktanlieferung per LKW: Gro\"sse Teile mit geringer Komplexit\"at (z. B. Bodenmatte oder Kofferraumverkleidung) werden per LKW zeitnah und in unmittelbare N\"ahe des Verbauortes angeliefert.
\item Modulanlieferung per EHB8: Gro\"sse und komplexe Baugruppen (z. B. Cockpit) werden direkt auf dem Werksgel\"ande von externen Lieferanten oder BMW Mitarbeitern montiert.
\item Lagerware per FTS: Die Mehrzahl der Teile wird in einem Versorgungszentrum gelagert, kommissioniert und mit Fahrerlosen Transportfahrzeugen (FTF) an die jeweiligen Verbauorte in der Montage gebracht (G\"unter Ullrich, 2011 S. 36).\end{itemize}
Es sind 74 FTF im Einsatz, als Ladehilfsmittel werden mehr als 2.000 Rollwagen in zwei unterschiedlichen Ausf\"uhrungen eingesetzt. Je FTF werden entweder zwei kleine Rollwagen, zur Aufnahme von Beh\"altern bis DIN-Gr\"o\"sse, oder ein so genannter \"ubergro\"sser Rollwagen zur Aufnahme von Gro\"ssbeh\"altern eingesetzt. Zus\"atzlich gibt es noch die Sequenziergestelle mit Sonderaufbauten (G\"unter Ullrich, 2011 S. 37). Durch einen Laser-Scanner auf dem FTF wird den Personenschutz und Hinderniserkennung \"ubernommen. 

Die Fahrerlosen Transportfahrzeuge finden ihren Weg mit Hilfe der so genannten freien Navigation. Damit ist gemeint, das die Fahrzeuge ohne physikalische Leitspuren und nach einem kombinierten Prinzip aus Kopplung und Peilung arbeiten. Kopplung bedeutet die Auswertung von fahrzeuginternen Sensoren (Messr\"ader und ein faseroptischer Kreisel), wodurch der zur\"uckgelegte Weg samt Kurven bestimmt wird (G\"unter Ullrich, 2011 S. 37). Bei jeder Peilung werden aufgetretene Fahrfehler, die durch Schlupf der R\"ader oder durch Ver\"anderungen des Raddurchmessers auftreten k\"onnen, korrigiert. Die Vorteile dieses, auch Magnet Navigation genannten, Verfahrens liegen in der Zuverl\"assigkeit und der Flexibilit\"at bei zuk\"unftigen Layoutanpassungen (G\"unter Ullrich, 2011 S. 37).



	
	% Chapter 3 Projektorganisation
	\clearpage
	\ohead[Projektorganisation]{Projektorganisation}
	\chead[Uni Oldenburg]{Uni Oldenburg}
	\ihead[PG FAISE]{PG FAISE}
	\setheadtopline{1pt}
	\setheadsepline{0.5pt}
	\ofoot[Endbericht]{Endbericht}
	\cfoot[\pagemark]{\pagemark}
	\ifoot[31. Oktober 2014]{31. Oktober 2014}
	\setfootsepline{0.5pt}
	\setfootbotline{1pt}
	\section{Projektorganisation}
Die nachfolgenden Abschnitte beschreiben, wie die Projektgruppe organisiert ist, um die vorgegebene Aufgabenstellung umzusetzen. Ziel ist es zu beschreiben, wie die Aufteilung in Teilgruppen stattfindet, welche Vorgehensmodelle für die Gesamt- und Teilgruppen verwendet werden, welche Rollen einzelne Personen haben und welche Softwaretools zur Unterstützung angewandt werden.   
\subsection{Organisation in drei Teilgruppen}
Die Projektgruppe ist in die drei Teilgruppen Materialfluss, Fahrzeuge und Simulation unterteilt. Die Teilgruppe Materialfluss befasst sich mit Programmierung der Sensorik und Aktorik für die Rampen sowie dem Aufbau eines Sensornetzwerkes zur Kommunikation zwischen den verschiedenen Akteuren der Simulation. Die Teilgruppe Fahrzeuge befasst sich mit allen Aspekten, die für das Funktionieren der Fahrzeuge verantwortlich sind. Dazu zählen u.a. Navigation, Odometrie und Lokalisierung. Auch ist die Einrichtung der Versorgungsinfrastruktur für die Fahrzeuge in Form von Ladestationen Aufgabe der Fahrzeuggruppe. Zusammen entwickeln die Teilgruppen Fahrzeuge und Materialfluss das physische System, so dass Kommunikation und Abstimmung zwischen diesen beiden Gruppen besonders wichtig sind. Die Teilgruppe Simulation entwickelt die Software mit der eine virtuelle Simulation erstellt werden kann. Außerdem beinhaltet die Software einen hybriden Modus, in dem das physische System auf die Software abgebildet wird und beide Teilsysteme ein Gesamtsystem bilden. Für die Entwicklung des Hybridmodus muss ein funktionierendes physisches System vorliegen. Die Aufteilung in drei Teilgruppen gliedert das Gesamtprojekt in eindeutig abgrenzbare Aufgabenfelder, so dass Kompetenzen und Verantwortlichkeiten klar definiert werden können.
\subsection{Vorgehensmodell}
Für die Durchführung der Projektgruppe muss ein Vorgehensmodell sowohl für die Gesamt- als auch für die Teilgruppen festgelegt werden. Durch ein Vorgehensmodell wird die Arbeit im Team strukturiert und es wird festgelegt, wie bestimmte Aufgaben, wie z.B. Abgleich mit Kunden und Anwendern, umgesetzt werden sollen. Sowohl für die Teilgruppen als auch für die Gesamtgruppe wurde Scrum als Vorgehensmodell gewählt. Da es für ein sehr komplexes Projekt, wie das vorliegende, schwierig ist nur von einer groben Vision sowie von User Stories auszugehen, wurden zunächst auf Basis des vorliegenden Lastenhefts in jeder Teilgruppe Pflichtenhefte erstellt, um die Aufgabenstellung ausreichend genau zu definieren und Stabilität zu schaffen. Anschließend wurde dazu übergegangen User Stories zu definieren, die die Anforderungen aus dem Pflichtenheft berücksichtigen und aus Anwendersicht darstellen. Die Sprints in den Teilgruppen sind mit einem Monat bemessen und werden zur Durchführung der User Stories genutzt. Die Rolle des Product Owners wird von den beiden Betreuern eingenommen, die sowohl für die Teil- als auch für die Gesamtgruppen zur Verfügung stehen, um entwickelte Funktionalität abzugleichen. Die Durchführung von Daily Scrums ist zeitlich nicht möglich, da es sich um eine studentische Projektgruppe handelt, deren Stundenplan keine täglichen Treffen ermöglicht. Deshalb wurde das Scrum Vorgehensmodell dahingehend angepasst, dass statt Daily Scrums Weekly Scrums durchgeführt werden. Die Weekly Scrums finden sowohl in den Teilgruppen als auch in der Gesamtgruppe statt. In den Daily Scrums der Gesamtgruppe wird zunächst von jeder Person berichtet, welche Aufgaben in der vorherigen Woche erledigt wurden. Damit verbundene Probleme und Hindernisse können direkt in der Gruppe besprochen und eventuell beseitigt werden. Die Ergebnisse aus den Teilgruppen werden ebenfalls vorgestellt und mit den Product Ownern abgeglichen. Das Scrum Vorgehensmodell wird mit  Prototyping kombiniert. Durch das Prototyping sollen zu bestimmten Meilensteinen die kombinierten Ergebnisse aus den Teilgruppen vorgestellt werden, um den Stand des Gesamtsystems begutachten zu können. Die genauere Beschreibung des Scrum Vorgehens für die Teilgruppen wird in den entsprechenden Kapiteln beschrieben, in der die Arbeit in den Teilgruppen aufgegriffen wird.   
	
	% Chapter 4 Allgemeine Anforderungen
	\clearpage
	\ohead[Allgemeine Anforderungen]{Allgemeine Anforderungen}
	\chead[Uni Oldenburg]{Uni Oldenburg}
	\ihead[PG FAISE]{PG FAISE}
	\setheadtopline{1pt}
	\setheadsepline{0.5pt}
	\ofoot[Endbericht]{Endbericht}
	\cfoot[\pagemark]{\pagemark}
	\ifoot[31. Oktober 2014]{31. Oktober 2014}
	\setfootsepline{0.5pt}
	\setfootbotline{1pt}
	\section{Allgemeine Anforderungen}
In diesem Abschnitt sollen allgemeine Anforderungen beschrieben werden, die für das Gesamtsystem gelten. Zunächst soll ein Ablaufszenario beschrieben werden, dass beschreibt, wie Rampen und Volksbots im Lager miteinander kommunizieren.
\subsection{Ablaufszenario}\label{AL} 
Das Ablaufszenario beschreibt die logischen Schritte, die von den Akteuren ausgeführt werden, um das Ziel zu erreichen, ohne dabei auf die technischen Implementierungsdetails einzugehen. Es dient als Basis für die Implementierung sowohl des physischen Systems als auch der Simulationssoftware.
In der Software soll der Ablauf komplett umgesetzt werden. Aus zeitlichen Gründen ist es nicht möglich den gesamten Ablauf im physischen System umzusetzen, weshalb dort eine Anpassung erfolgt. Der Ablauf lässt sich in folgende Schritte unterteilen:

\begin{itemize}
\item Nachdem ein Paket am Eingang angekommen ist, wird ihm eine ID zugewiesen, so dass es eindeutig anhand seiner ID identifiziert werden kann.
\item Die Eingangsrampe auf der sich das Paket befindet, fragt alle Ausgänge, ob dieses Paket benötigt wird und zeitgleich werden die Zwischenrampen kontaktiert, um zu überprüfen, ob dort Platz frei ist. Falls ein Ausgang antwortet, wird dem Paket die ID des Ausgangs als Ziel zugewiesen. Falls nicht, wird dem Paket die ID eines der freien Zwischenlager zugeordnet.
\item
Falls der Ausgang eine oder mehrere Pakete benötigt, fragt der Ausgang die Zwischenrampen, ob ein Paket mit der vorhandenen ID verfügbar ist. Falls dies der Fall ist, wird dem Paket aus dem Zwischenlager die Ausgangs ID als Ziel zugewiesen.
\item
Sowohl Eingang als auch Ausgang stellen ihre Anfragen zyklisch, da bei einer einzigen Anfrage keine Garantie besteht, dass eine Zielrampe gefunden wird.
\item
Nachdem einem Paket ein Ziel zugewiesen wurde, versucht die Rampe ein Transportmittel zu finden. Dazu werden die Volksbots kontaktiert, die anhand des Energie- und Zeitaufwands ein Angebot abgeben. Die Rampe wählt den Bot mit dem besten Angebot aus.
\item
Der ausgewählte Volksbot bewegt sich zur Rampe mit dem Paket und meldet sich an der Rampe, wenn er seine Zielposition erreicht hat. Die Rampe übergibt dem Volksbot das Paket mit allen notwendigen Informationen (ID und Ziel).
\item
Der Volksbot fährt zur Zielrampe, meldet sich dort an und übergibt das Paket.
\end{itemize} 

	
	% Chapter 5 Teilbericht Simulation
	\clearpage
	\ohead[Teilbericht Simulation]{Teilbericht Simulation}
	\chead[Uni Oldenburg]{Uni Oldenburg}
	\ihead[PG FAISE]{PG FAISE}
	\setheadtopline{1pt}
	\setheadsepline{0.5pt}
	\ofoot[Endbericht]{Endbericht}
	\cfoot[\pagemark]{\pagemark}
	\ifoot[31. Oktober 2014]{31. Oktober 2014}
	\setfootsepline{0.5pt}
	\setfootbotline{1pt}
	\section{Teilbericht Simulation}
Der Teilbericht der Simulationsgruppe beschreibt die entwickelte Simulationssoftware von der Anforderungsanalyse über die Implementierung hin zum Testen und Validieren. 
\subsection{Lastenheft}
Mit der Komponente Simulation soll auf Basis des Ablaufkonzepts eine Software erstellt werden, die es erlaubt einen automatisierten Materialfluss auf Basis von FTS zu simulieren ohne dabei an die zahlenmäßigen Beschränkungen des physischen Systems gebunden zu sein. Vonseiten des Auftraggebers wurde ein Lastenheft vorgegeben, das die gewünschten Kernfunktionalitäten der Simulationssoftware beschreibt. Es enthält folgende Anforderungen:
\begin{enumerate}
\item \textbf{Akteure}: Die virtuellen Akteure sind in ihrem Verhalten und Eigenschaften (Geschwindigkeit, Dauer einer Paketübergabe etc.) den echten Objekten aus dem physischen System nachempfunden (Volksbots und passive Rampen).
\item \textbf{Ablauf}: Der in Abschnitt 4.1 beschriebene Ablauf wird in der Simulation umgesetzt. 
\item \textbf{Visualisierung}: Die Zustände der Akteure werden dynamisch visualisiert. Wird beispielsweise die Anzahl der Pakete auf einer Rampe um eins erhöht, dann soll dies unmittelbar in der Anzeige visualisiert werden.
\item \textbf{Generierung von Aufträgen}: Eingehende und ausgehende Transportaufträge können erstellt und simuliert werden. 
\item \textbf{Einstellungen}: Verschiedene Parameter der Simulation (Anzahl und Art der Akteure, Anzahl der Aufträge etc.) können vom Nutzer vor dem Starten der Simulation angepasst werden.
\item \textbf{Statistiken}: Es werden wichtige Daten geloggt, um am Ende eines Simulationslaufs aussagekräftige Analysen über Stromverbrauch, gefahrene Strecken, Vergabe von Aufträgen usw. machen zu können.
\end{enumerate}
\subsection{Grundlegende Designentscheidungen}
Vor der Entwicklung der Simulationssoftware mussten grundlegende Designentscheidung getroffen werden. Zum einen musste entschieden werden, ob die Simulation von einem autonomen Lager auf Basis eines vorhanden Tools oder komplett neu entwickelt werden sollte. Auch musste zwischen Desktop- und Webanwendung entschieden werden und ob die jeweilige Alternative mit oder ohne Zuhilfenahme eines Frameworks implementiert wird. In den nachfolgenden Abschnitten werden die getroffenen Designentscheidungen begründet.
\subsubsection{Eigenentwicklung}
Im Vorfeld der Entwicklung wurde der Teilgruppe Simulation das Player/Stage Tool als Alternative zu einer kompletten Neuentwicklung einer Software vorgeschlagen. Das Tool beinhaltet zum einen die Komponente Player, die eine Hardware Abstraktionsschicht darstellt. Mit dieser Komponente kann mit Robotern, wie beispielsweise einem Volksbot, interagiert werden, ohne dass technische Details der Komponenten (Laserscanner, Motor etc.) bekannt sein müssen. Auf Basis von selbstgeschriebenem Code können Roboter gesteuert werden. Die Komponente Stage horcht auf die Befehle, die Player ausführt und visualisiert diese in einem eigenen Graphical User Interface. Jedoch kann Stage auch ohne Hardware benutzt werden, indem man über Konfigurationsdateien ein eigenes Szenario erstellt und die virtuellen Roboter über den eigenen Code steuert. Somit bietet das Tool die Möglichkeit, eine Simulation mit Robotern zu erstellen und das gewünschte Verhalten der Roboter über eigenen Code abzubilden. Auch muss die Visualisierung nicht selbst entwickelt werden (Vgl.\cite{plstg}). 
\\\\
Dennoch wurde eine Eigenentwicklung der Nutzung des Tools vorgezogen. Die Benutzeroberfläche von Stage bietet die Möglichkeit die Anzahl der Roboter und das Layout eines Szenarios über die entsprechenden Konfigurationsdateien einzustellen (Vgl.\cite{plstg}). Jedoch gibt es beispielsweise keine Möglichkeit, Aufträge zu erstellen bzw. zu simulieren oder Statistiken anzuzeigen. Somit ist die Entwicklung einer eigenen Benutzeroberfläche unumgänglich. Das bedeutet, dass die Stage Oberfläche über eine eigene Benutzeroberfläche gesteuert werden muss. Somit hätte man eine Trennung zwischen Visualisierung und Konfiguration eines Szenarios, was die Benutzerfreundlichkeit erheblich beeinträchtigt, da ein Nutzer den Durchlauf einer Simulation über zwei Benutzeroberflächen hinweg verfolgen müsste. Die Entwicklung eines eigenen Systems bietet somit erheblich mehr Benutzerfreundlichkeit und ermöglicht es, alle Anforderungen an das Interface in einer Benutzeroberfläche zu integrieren.
\subsubsection{Entwicklung einer Webanwendung}\label{sec:Entwicklung einer Webanwendung} 
Die Software soll als Webanwendung implementiert werden. Gegenüber einer Desktopanwendung bietet eine Webapplikation folgende Vorteile:
\begin{itemize}
\item Das System ist plattformunabhängig und kann somit auf jedem Rechner, der über einen Webbrowser verfügt, ausgeführt werden.
\item Die Software muss nicht lokal installiert werden und kann direkt genutzt werden.
\item Werden Änderungen an der Software vorgenommen, sind diese direkt verfügbar, da Updates über den Webserver eingespeist werden. Die Software ist somit immer auf dem aktuellsten Stand. 
\end{itemize}
\subsubsection{Umsetzung durch GWT}\label{GWT} 
Die Entwicklung der Webanwendung sollte mithilfe eines Frameworks erfolgen, das es erlaubt, den Code sowohl für die Client- als auch für die Serverseite in einer Programmiersprache zu entwickeln. Außerdem sollte das Framework Schnittstellen bieten, um asynchrone Kommunikation und Push-Dienste zu nutzen, ohne sich um die exakten Details kümmern zu müssen. Ausgewählt wurde das Google Web Toolkit (GWT). GWT ist ein von Google entwickeltes Framework zur Erstellung von Webanwendungen. Der Java-Code für den Client wird von dem GWT Compiler in den entsprechenden Javascript- und HTML-Code übersetzt. Somit kann die Entwicklung sowohl für Client als auch für den Server auf Basis von Java erfolgen. Zudem entfällt die Anpassung des Javascript-Codes für die verschiedenen Browser, da GWT beim Kompilieren automatisch für jeden Browser eine lauffähige Version erzeugt. Weiterhin besitzt GWT  eine RCP-Schnittstelle für die asynchrone Kommunikation zwischen Client und Server und lässt sich um Komponenten erweitern, um Daten vom Server zum Client zu pushen (Vgl.\cite{gwt}). Somit erfüllt GWT sämtliche an ein Framework gestellte Anforderungen. Weitere Alternativen wurden nicht in Betracht gezogen, da drei von fünf Mitgliedern der Teilgruppe Simulation bereits positive Erfahrungen mit GWT gemacht haben und die anderen Mitglieder somit schnell einarbeiten konnten. 
\subsection{Konzeption der Systemkomponenten}
In diesem Abschnitt wird die Konzeption der Gesamtarchitektur als auch der einzelnen Systemkomponenten beschrieben, die im Rahmen der Sprints erarbeitet wurde. Die Konzeption beinhaltet zum einen die Anforderungen als auch die daraus abgeleiteten Implementierungsvorgaben.
\newpage
\subsubsection{Gesamtarchitektur}\label{GA}
Wie in Kapitel \ref{sec:Entwicklung einer Webanwendung} beschrieben, soll die Software als Webanwendung realisiert werden. Eine Webanwendung erfordert eine Client-Server Architektur. Abbildung \ref{Gesamtarchitektur} beschreibt die wesentlichen Komponenten des Systems und wie diese sich auf die Client- und Serverseite verteilen. Basis der Anwendung ist der Webserver, der die Basiskomponenten des Systems hosted: Zum einen stellt er die Laufzeitumgebung für Webanwendung und Datenbank bereit. Die Datenbank wird benötigt, um Daten, wie beispielsweise erstellte Szenarien, persistent zu speichern. Aus der Webanwendung heraus kann auf die Datenbank lesend und schreibend zugegriffen werden. In die Webanwendung soll ein Multiagentensystem (MAS) eingebettet werden. Ein MAS ist ein Netzwerk aus Softwareagenten. Softwareagenten sind Softwareeinheiten, die in in der Lage sind, Aufgaben selbstständig durchzuführen (Vgl.\cite{mas}). Mithilfe des MAS kann ein Lager simuliert werden, in dem der Warenfluss durch vollständig autonome Akteure durchgeführt wird. 
\\\\
Der Webserver beinhaltet die Logik des Systems. Auf dem Client soll die Visualisierung erfolgen und der Nutzer soll das Starten einer Simulation initiieren können. Das System soll von einem Webbrowser aus aufrufbar sein, in dem die Ergebnisse der serverseitigen Prozesse dargestellt werden. Mehrere Nutzer sollen das System gleichzeitig nutzen können, ohne Login und Registrierung.

\begin{figure}[h!]
	\centering
		\includegraphics[width=0.8\textwidth]{grobarchitektur.jpg}        
		\caption{Gesamtarchitektur}
	\label{Gesamtarchitektur}
\end{figure} 
\newpage
\subsubsection{Konzeption der Benutzeroberfläche}
Die Benutzeroberfläche soll einem Nutzer die Möglichkeit bieten, auf sämtliche Funktionalitäten, die für die Durchführung eines Simulationsdurchlaufs relevant sind, zuzugreifen. Abbildung \ref{GUI} zeigt den schematischen Aufbau der Gui anhand eines Mockups. Die Menüleiste beinhaltet drei Menu Items: Simulation, Auftragsliste und Statistiken. Über die Items Simulation und Auftragsliste, sollen erstellte Szenarien und Auftragslisten geladen und gespeichert werden können. Außerdem soll eine Simulation gestartet werden können. Das Statistik Item erlaubt den Zugriff auf Statistiken, die für einen Durchlauf generiert wurden. Links unter der Menüleiste befindet sich die Auftragsliste, über die Aufträge generiert und angezeigt werden können. Darunter befindet sich der Bereich, der für die Modellierung eines Szenarios relevant ist. Es sollen Rampen, Fahrzeuge und Wände als Modellelemente auswählbar sein und in der Zeichenfläche platziert werden können. Die Zeichenfläche selber befindet sich rechts unter der Menüleiste. Dort werden die Aktionen der Akteure, wie z.~B. Aufladen eines Pakets, visualisiert. Das unterste Element enthält eine Debug Konsole in der serverseitige Aktionen dargestellt werden können, die nicht in der Zeichenfläche dargestellt werden sollen. 
\begin{figure}[h!]
	\centering
		\includegraphics[width=0.8\textwidth]{Mockup.png}        
		\caption{Schematischer Aufbau der Benutzeroberfläche}
	\label{GUI}
\end{figure}
\subsubsection{Generierung von Aufträgen}
Die Simulationssoftware soll ein Umschlagslager simulieren. Das bedeutet, dass Pakete in das Lager geliefert, zwischengelagert und an den Ausgangsrampen wieder abgeholt werden, wenn ein bestimmtes Paket angefragt wird (Vgl.\ref{Einsatzszenario}). Das bedeutet, dass eine Unterscheidung getroffen werden muss zwischen einem physischen Paket und einer Nachfrage nach einem bestimmten Paket, die ein Ausgang stellt. Deshalb soll zwischen eingehenden und ausgehenden Aufträgen unterschieden werden. Es soll möglich sein eine festgelegte Anzahl an Aufträgen zufällig über die Gui zu generieren. Außerdem muss sichergestellt werden, dass die Menge an generierten eingehenden und ausgehenden Aufträgen in einem bestimmten Verhältnis zueinander stehen, um zu verhindern, dass nur eingehende oder nur ausgehende Aufträge generiert werden. 
\subsubsection{Anforderungen an ein Multiagenten-Framework}
Um ein Umschlagslager und die darin enthaltenden Akteure (Rampen und Fahrzeuge), zu simulieren, wird ein Multiagentensystem benötigt (Vgl.\ref{GA}). Zur Erstellung eines MAS soll ein Framework verwendet werden, dass die nachfolgenden Anforderungen erfüllt: Das Framework muss das Erstellen von verschiedenen Agententypen ermöglichen, um die Akteure und ihre spezifischen Aufgabenstellungen umzusetzen. Agenten müssen in der Lage sein untereinander Nachrichten auszutauschen und auf bestimmte Nachrichten oder Ereignisse mit definierten Verhaltensweisen zu reagieren. Weiterhin sollen die Aktionen der Agenten denen der realen Akteure hinsichtlich der Dauer ähneln. Außerdem muss das Framework Aktionen von verschiedenen Agenten parallel ausführen können, damit beispielsweise das gleichzeitige Fahren mehrerer Fahrzeuge möglich ist.
\subsubsection{Benötigte Agententypen}
Sowohl Rampen als auch Fahrzeuge müssen verschiedene Aktionen durchführen. Dazu gehören u.a. das Befördern von Paketen, die Vergabe von Aufträge, Durchführung von Auktionen usw. Würde man alle Aufgaben, die ein Akteur durchführen muss, in einem Agenten bündeln, so wäre ein solcher Agent nur schwer wartbar und es könnten abhängig vom Agenten-Framework Probleme bei der Parallelisierung von Aktionen auftreten. Es bietet sich an die erforderlichen Aufgaben eines Akteurs auf mehreren Agenten zu verteilen. Durch die Modularisierung kann die Entwicklung des Systems parallelisiert werden und Änderungen an einem Agenten haben geringere Auswirkungen auf das Gesamtsystem. Die Wartbarkeit des Systems erhöht sich. Für das zu entwickelnde System wurden die folgenden vier Agententypen konzipiert:  
\begin{itemize}
\item Paketagent: Verwaltung der Paketdaten
\item Orderagent: Ermittlung von Zielrampen und Zuweisung von Zielen (Wird nur bei Rampen benötigt)
\item Routingagent: Durchführung von Auktionen und Berechnung von möglichen Pfaden
\item Plattformagent: Durchführung physischer Aktionen (Fahren, Aufladen von Paketen etc.)
\end{itemize}
\subsubsection{Kommunikation zwischen den Agenten}
Die zu entwickelnde Software soll die in Abschnitt \ref{AL} beschriebenen Abläufe umsetzen. Die Aufgaben der verschiedenen Akteure müssen auf die Agenten verteilt werden. Die entworfenen Kommunikationsschritte der Agenten sollen für den Fall dass ein Eingang Ausgänge und Zwischenlager fragt, beschrieben werden, um die Aufgaben der einzelnen Agenten genauer abzugrenzen. Im Rahmen der Implementierung können sich noch Änderungen ergeben, die technisch notwendig sind. 
\begin{enumerate}
\item Trifft ein Paket ein, verlangt der Paketagent vom Orderagent eine Destination für das entsprechende Paket.
\item Der Orderagent einer Eingangsrampe fragt die Orderagenten der Ausgangs- und Zwischenrampen.
\item Die Orderagenten prüfen im Abgleich mit ihren Paketagenten, ob Platz frei ist (Zwischenrampe) oder die Paket-ID benötigt wird (Ausgang). Anschließend antworten sie dem Orderagenten am Eingang.  
\item Wurde ein Ziel für das Paket gefunden, soll der Orderagent den Start einer Auktion initiieren, indem er den Routingagenten benachrichtigt. 
\item Der Routingagent verlangt eine Aufwändsschätzung von allen Routingagenten der Fahrzeuge.
\item Der Routingagent eines Fahrzeugs berechnet eine Aufwandsschätzung anhand seiner Position und antwortet dem Routingagenten der Rampe. Fährt der Volksbot oder nimmt an einer anderen Auktion teil, so wird -1 als Aufwandsschätzung zurückgeschickt.
\item Der Routingagent einer Rampe, wählt, sofern vorhanden, den Bot aus, der den geringsten Aufwand benötigt, um ein Paket abzuholen und weist ihm den Auftrag zu.
\item Der Plattformagent fährt zu der jeweiligen Eingangsrampe und lädt das Paket auf. Dies geschieht durch einen Nachrichtenaustausch mit dem jeweiligen Plattformagenten der Rampe, der die Paketdaten übergibt. Das Fahren zur Zielrampe und das Abladen des Pakets erfolgt analog. 
\end{enumerate}
\subsubsection{Konzeption des Pathfindings}
\subsubsection{Konzeption der Statistiken}
\subsubsection{Interaktion der Komponenten}
Durch das Zusammenspiel der Komponenten der verschiedenen Systemkomponenten soll eine Simulation durchgeführt werden können. Die Aufträge müssen entsprechend ihrer Zeiten an den Server geschickt werden. Dies soll clientseitig durch einen Timer durchgeführt werden, um das MAS zu entlasten. Abbildung \ref{Int} zeigt den Ablauf und die Komponenten, die für das Starten einer Simulation erforderlich sind: 
\begin{enumerate}
\item Ein potenzieller Nutzer startet eine Simulation über die Benutzeroberfläche
\item Der Server wird durch die GUI informiert, dass die Simulation gestartet werden soll.
\item Der Server startet das Multiagentensystem.
\item Der Server meldet dem Client den Start des MAS.
\item Der Client weiß nun, dass die Agenten bereit sind, Aufträge entgegenzunehmen und startet den Timer für die Jobliste.
\item Die Aufträge werden gemäß ihrer Startzeit an den Server geschickt.
\item Der Server leitet die Aufträge an das MAS weiter
\item Die Daten über sichtbare Zustandsveränderungen werden an den Client geschickt und dort in der GUI visualisiert.
\end{enumerate}

\begin{figure}[h!]
	\centering
		\includegraphics[width=0.8\textwidth]{Interaktion.jpg}        
		\caption{Interaktion der Systemkomponenten}
	\label{Int}
\end{figure}
\newpage
\subsection{Implementierung der Systemkomponenten}
In diesem Abschnitt soll die Implementierung, der zuvor konzipierten Systemkomponenten beschrieben werden. Zum einen soll die Auswahl von Technologien und Frameworks zur Umsetzung beschrieben und falls notwendig begründet sowie die Funktionalität des Systems auf technischer Ebene dargestellt werden.
\subsection{Implementierte Gesamtarchitektur}
Abbildung \ref{GAI} zeigt die konkrete Gesamtarchitektur des Systems, die durch die Auswahl von Umsetzungstechnologien entstanden ist. Die Webanwendung soll, wie bereits in Abschnitt \ref{GWT} beschrieben, durch das GWT Framework implementiert werden. Der Code für Server und Client, der für Visualisierung, Client-Server Kommunikation u.ä., benötigt wird, wird durch GWT-Bilbiotheken bereitgestellt. Eingebettet in die GWT-Klassen wird das Multiagentensystem, das mithilfe des Java Agent Development Framework (JADE) implementiert wurde. Als Datenbankmanagementsystem wurde PostgreSQL ausgewählt. Die Applikation wird durch einen Jetty Server gehostet, der die Java Laufzeitumgebung und das GWT Software Development Kit ebenfalls bereitstellt.  
\begin{figure}[h!]
	\centering
		\includegraphics[width=0.8\textwidth]{architekturSimu.jpg}        
		\caption{Implementierte Gesamtarchitektur}
	\label{GAI}
\end{figure}
	

	% Chapter 6 Komponentenbeschreibung
	\clearpage
	\ohead[Komponentenbeschreibung]{Komponentenbeschreibung}
	\chead[Uni Oldenburg]{Uni Oldenburg}
	\ihead[PG FAISE]{PG FAISE}
	\setheadtopline{1pt}
	\setheadsepline{0.5pt}
	\ofoot[Endbericht]{Endbericht}
	\cfoot[\pagemark]{\pagemark}
	\ifoot[31. Oktober 2014]{31. Oktober 2014}
	\setfootsepline{0.5pt}
	\setfootbotline{1pt}
	\section{Teilbericht Materialfluss}
Das Ziel der Materialflussgruppe ist das dezentrale Management der Fördereinheiten auf den Rampen und auf dem Volksbots.

\subsection{Anforderungen}
In diesem Abschnitt werden die gestellten Anforderungen zusammengetragen. Wir unterscheiden dabei zwischen funktionalen Anforderungen, die die direkte Funktionalität des fertigen Systems beschreiben, und nicht-funktionalen Anforderungen, die die qualitativen Eigenschaften des Systems widerspiegeln.
\subsubsection{Funktionale Anforderungen}
\begin{enumerate}
\item \textbf{Plattform}: Die physische Zelle wird als Netzwerk von Knoten in einem drahtlosen Sensornetzwerk implementiert. Als Plattform dienen MICAz-Module mit Atmel ATMega 128 Mikrocontroller und CC2420 Funkchip (siehe \autoref{MICAZ}).
 \item \textbf{Aktorik/Sensorik}: Die Rampen verfügen über Magnetstifte zum Vereinzeln der Pakete und Lichtschranken zum Erkennen von Paketen. Sie werden von den MICAz-Modulen angesteuert beziehungsweise ausgelesen.
 \item \textbf{Kommunikation}: Die MICAz-Module auf Rampen und Volksbots kommunizieren drahtlos untereinander auf Basis von Agenten-Nachrichten.
 \item \textbf{Synchronisation}: Die Simulation wird über ein Micaz-Modul, das als Gateway fungiert, an die drahtlose Kommunikation angebunden. Die Synchronisation der Zustände erfolgt über eine serielle Schnittstelle.
 \item \textbf{Disposition}: Die Controller kennen den Belegungszustand der Rampe und generieren nach dem FIFO-Prinzip Aufträge, die sie an die Volksbots vergeben.
 \item \textbf{Übergabe}: Wenn eine Ein- oder Auslagerung an einer Rampe ausgeführt werden soll, so übernimmt der Controller der Rampe die Kontrolle über die Fördereinheit des Fahrzeugs und sorgt dafür, dass das Paket verladen wird.
 \item \textbf{Kooperation}: Einsatz kooperativer Lösungsstrategien für die Materialflusssteuerung, Überwachung und Steuerung mittels Multi-Agentensystem.
\end{enumerate}

\subsubsection{Nicht-funktionale Anforderungen}
\begin{enumerate}
\item \textbf{Ressourcen}: Bei der Entwicklung muss 
auf den sparsamen Umgang mit Hardwareressourcen (Rechenzeit, Kommunikationsbandbreite, Speicher) geachtet werden. Insbesondere der vorhandene Arbeitsspeicher und das Kommunikationsmedium dürfen nicht überlastet werden, um einen stabilen Betrieb zu garantieren. 
\item \textbf{Stabilität}: Es müssen Maßnahmen getroffen werden, um ein stabiles System zu schaffen. Dies gilt insbesondere für die möglichst verlustfreie Übertragung von drahtlosen Nachrichten.
\item \textbf{Volksbots}: Die Module des Materialfluss über eine definierte Schnittstelle mit den Fahrzeugen kommunizieren, um auf die Aktorik und Sensorik der Volksbots zugreifen und schließlich einen Transport der Pakete gewährleisten zu können.
\end{enumerate}

\subsection{Beschreibung der Komponenten}
\subsubsection{Rampen}
Rampen stellen Ein- und Ausgänge, sowie Zwischenlager im physischen System dar. Auf einer Rampe finden bis zu vier Pakete Platz. Bolzen hinter dem ersten Paket, separiert dieses von den anderen Dreien. Damit das vorderste Paket nicht vorne von der Rampe herunterfällt, sind an der Vorderseite zwei weitere Bolzen angebracht. Alle vier Bolzen sind seitlich der Rampe befestigt. Eine autonome Steuerung der Rampen, wird durch ein angebrachtes MICAz-Modul ermöglicht. Durch vier Lichtschranken, wird eine Überwachung der Rampe ermöglicht. Diese beinhaltet zum einen das Abfragen, wie viele Pakete auf einer Rampe liegen. Zum anderen kann durch die Überwachung überprüft werden, an welcher Stelle Pakete liegen. \autoref{fig:skiram} zeigt ein Beispiel solch einer Rampe.

\begin{figure}[h!]
	\centering
		\includegraphics[width=0.9\textwidth]{SkizzeRampe.png}
	\caption{Beispiel einer eingesetzten Rampe}
	\label{fig:skiram}
\end{figure}

\subsubsection{Mikrocontroller}
Durch einen Mikrocontroller (MC) wird im Prinzip ein Mikrocomputer auf einem Chip dargestellt. Solch ein Mikrocomputer ist ein Rechner, dessen Zentraleinheit aus einem oder mehreren Mikroprozessen besteht. Zusätzlich enthält ein Mikrocomputer Speicher, ein Verbindungssystem und Ein- bzw. Ausgabeschnittstellen. Das Ziel eines Mikrocontroller ist es, eine Kommunikations- oder Steuerungsaufgabe mit möglichst wenig Bauteilen zu lösen. Der in einem Mikrocontroller verbaute Prozessorkern, Speicher und die Aus- und Eingabeschnittstellen eines Mikrocontroller, sind auf die Lösung solcher Aufgaben zugeschnitten. Die große Anzahl an potenzieller Aufgabenstellung hat zur Folge, dass es eine Vielfalt von Mikrocontrollern gibt. Meist sind die Mikrocontroller deshalb in Mikrocontrollerfamilien aufgeteilt. Innerhalb einer Familie unterscheiden sich die Controller nicht im Prozessorkern, sondern im Speicher und in den Ein- und Ausgabeschnittstellen \cite{ECHT2005}. In \autoref{fig:aufbmc} ist der schematische Aufbau eines solchen Controllers dargestellt.
\begin{figure}[th]
	\centering
		\includegraphics[width=0.9\textwidth]{Aubau_eines_Mikrocontrollers.png}
	\caption{Schematischer Aufbau eines Mikrocontrollers \cite{habil:Ostermeye:2014:Online}}
	\label{fig:aufbmc}
\end{figure}

\begin{itemize}
\item Prozessor (CPU)
\begin{itemize}
          \item Arithmetic Logic Unit, kurz ALU (Rechenwerk)
          \item 32 GPIO-Register (Arbeitsregister für ALU)
          \item Programmcounter (Programmposition)
					\item Statusregister (Status der aktuellen Operation) 
\end{itemize}
\item Speicher
\begin{itemize}
          \item SRAM Datenspeicher (Static Random-Access Memory)
					\item Flash ROM Programmspeicher (Read Only Memory)
					\item EEPROM Festspeicher (Electrically Erasable Programmable Read-Only Memory)
\end{itemize}
\item Peripheriekomponenten
    \begin{itemize}
          \item I/O-Ports Primärfunktion der Pins (Ein- und Ausgänge)
          \item A/D-Wandler (Einlesen von analogen Spannungen)
          \item Timer/Counter (Zeitintervall-/PBM-Generator)
					\item Interrupts (Programmunterbrechungsroutinen)
					\item USART, I2C/TWI und SPI (Kommunikationsschnittstellen)
					\item Watch-Dog (Absicherung gegen Systemfehler)
					\item ISP (Schnittstelle zum \"{U}bertragen des kompilierten Programms)
	\end{itemize}
\end{itemize}
Mikrocontroller sind im heutigen Leben weit verbreitet und es gibt eine große Anzahl von Herstellern, die Mikrocontroller anbieten.
Im Folgenden werden einige Hersteller mit ihren MC-Familien beispielhaft aufgeführt:
\begin{itemize}
\item Intel (8051-Serie)
\item Renesas (H8)
\item Zilog (Z8)
\item Microchip (Pic)
\item Freescale (früher Motorola) (68HC08 bzw. 68HCS08)
\item Atmel (AVR, 8051-Serie)
\end{itemize}
Für das Projekt FAISE wurde die Atmel-Serie eingesetzt. Es sind Mikrocontroller mit erweiterten Peripherien und Funktionen, 
die auf der 8-Bit-AVR-Architektur basieren. Bei AVR handelt es sich um einen RISC-Kern, der an der Universität von Trondheim 
in Norwegen entwickelt und von Atmel aufgekauft wurde. Die CPU besitzt 32 allgemeine 8-Bit Register (general purpose registers) 
und ist in der Lage in einem einzigen Taktzyklus Daten aus zwei beliebigen Registern in die ALU zu laden, diese zu verarbeiten 
und das Ergebnis in einem beliebigen Register zu speichern \cite[vgl.]{Viktor:Seib:2014:Online}. Die Konfiguration eines Atmega 8 der Firma Atmel sieht so aus:
\begin{figure}[h!]
	\centering
		\includegraphics[width=0.9\textwidth]{Atmel8.png}
	\caption{Atmel 8 \\ \url{(http://www.ids.tu-bs.de/tl\_files/Lehre/Vorlesungen/Simulation2/Einfuehrung\_in\_die\_MC\_Programmierung\_Teil1.pdf)}}
	\label{Atmel 8}
\end{figure}
\begin{figure}[h!]
	\centering
		\includegraphics[width=0.9\textwidth]{LegendeAtmel8.png}
	\caption{Atmel 8 \url{(http://www.ids.tu-bs.de/tl\_files/Lehre/Vorlesungen/Simulation2/Einfuehrung\_in\_die\_MC\_Programmierung\_Teil1.pdf)}}
	\label{Legende Atmel8}
\end{figure}
\begin{itemize}
\item Pins für die Minimalbeschaltung
\begin{itemize}
          \item Spannungsversorgung
          \item Referenzspannung/Taktgeber
          \item Reset      
					\end{itemize}
\item Primärfunktion eines Pins
\begin{itemize}
          \item Ein- bzw. Ausgang
					\end{itemize}
\item Sekundärfunktion des Pins
\begin{itemize}
          \item A/D-Wandlereingang
          \item Ext. Interrupt
          \item PBM-Ausgang   
\end{itemize}
\end{itemize}
Für die Programmierung der AVR-Controller gibt es eine kostenlose Entwicklungsumgebung AVR-Studio, die das Einbinden des Compilers problemlos erlaubt.

\paragraph{ Sensorik und Aktorik}
Hauptziel der Teilgruppe Materialfluss ist das Management von Paketen auf einer Rampe.
Die Aufgabe der Sensorik ist dabei, dass die mit Lichtschranken ausgestatteten Rampen Pakete detektieren und auf Änderungen der Positionen der Pakete reagieren.
Die Lichtschranken bestehen aus einer Lichtstrahlenquelle (dem Sender) und einem Sensor (dem Empfänger) f\"{u}r diese Strahlung.
Als Lichtquelle kommt Infrarotlicht zum Einsatz und der Vorteil besteht in der einfachen Einstellung des Sensorsystems durch den
sichtbaren Lichtfleck. Das Funktionsprinzip der Lichtschranke besteht darin, den sich  ändernden Zustand durch die Lichtintensität mit dem Sensor zu registrieren. 
Die Rampen werden auf Hardwareebene um eine Aktorik zum Arretieren der Kisten ergänzt. Diese Aktoren (in unserem Fall die
eingesetzte Bolzenpaare) sind für das Ausführen von Bewegungen zuständig.
Sie sind aktive Stellelemente, die in der Antriebs- und Steuerungstechnik vom  Mikrorechner angesteuert werden, um das Verhalten des Prozesses durch das vom Sensor kommende Signal in einer gew\"{u}nschten Weise zu ermöglichen. In dieser allgemeinen Darstellung stehen die 
Ausgangssignale eines Sensors und die Stellsignale der Aktoren mit einem
Informationsverarbeitungssystem (IVS) in Verbindung.


\subsection{Werkzeuge}

\subsubsection{Robot Operating System (ROS)}

Das Robot Operating System ist ein frei verfügbares Framework zur Entwicklung von Roboter-Anwendungen. Die Verwaltung einer Vielzahl von Treibern, Bibliotheken und Tools vereinfacht die Arbeit an Robotern. Durch den Einsatz von Tutorials und Beispielen wird der Einstieg in die Entwicklung mit ROS sehr erleichtert. Die Kommunikation der entwickelten Programme, die sogenannten Nodes, erfolgt über Topics. Wie in Abbildung \ref{fig:ROS_concepts} schematisch dargestellt ist, werden die Daten, die von einer Node
ausgegeben werden sollen mit einem Titel, dem Topic, versehen und einem Pool an Topics hinzugefügt.

\begin{figure}[h!]
 \centering
		\includegraphics[width=0.5\textwidth]{drive/ROS_concepts.png}
	\caption{ROS Nachrichtenkonzept}
	\label{fig:ROS_concepts}
\end{figure}

Falls diese Daten folgend von einer anderen Node benötigt werden, kann diese das Topic einfach abhören und bekommt ausschließlich die gewünschten Daten übermittelt. Mit Hilfe des Nachrichtensystems lässt sich der Ablauf und die Synchronisation von einzelnen Funktionen des Roboters übersichtlich und einfach gestalten. Darüber hinaus bietet ROS praktische Tools zur Visualisierung von Messdaten, wie z.B. Laserscans.

\subsubsection*{Was will ROS?}
\begin{itemize}
 \item ROS will unterstützen, Code für Forschung und Entwicklung wiederzuverwenden
 \item loser Verbund von individuellen Programmteilen (Nodes)
 \item einzelne Programmteile können einfach geteilt und verbreitet werden (Packages und Stacks)
 \item ROS stellt Repositories zu Verfügung, um dort Code zu teilen \cite{ROS:2014:Online}
(http://www.ros.org/browse)
\end{itemize}
\subsubsection*{Was kann ROS?}
Die Hauptbestandteile und Hauptaufgaben von ROS sind Hardwareabstraktion; Gerätetreiber; Implementierung 
von viel genutzten Funktionalitäten; Inter-Prozess-Kommunikation; Paket-Management
\subsubsection*{Aufgaben des ROS}
\begin{itemize}
 \item Interprozesskommunikation (IPC)
 \begin{itemize}
\item Problematik der Kommunikation zwischen verschiedenen Systemen des Roboters
\item Sicherheitseinstellung bei der Übertragung
\item Anforderung an die Geschwindigkeit / Schnelligkeit der Kommunikation
\item Koordination von Nachrichten durch zentralen Master
\end{itemize}
\item Paketverwaltung – Packages
\begin{itemize}
 \item ROS ist durch Softwarepakete (sogn. Packages) aufgebaut
 \item Ein Package beinhaltet Laufzeitprozesse (Nodes); ROS abhängige Bibliotheken;
Datensätze; Konfigurationsdateien;3rd Party Software
 \item Packages sind dazu, da um Code wiederverwendbar zu machen
\end{itemize}
\item Paketverwaltung – Stacks
\begin{itemize}
\item Sammlung von Paketen (Packages)
\item Der Sinn ist, dass Stacks die Verteilung und Verwendbarkeit von Code
vereinfachen
\item Meist viele Packages ähnlicher Aufgaben in einem Stack verpackt
\end{itemize}
\item Message (msg)
\begin{itemize}
 \item  Messages werden verwendet um unter ROS Nachrichten zwischen Knoten und
Topics auszutuaschen
\item Dafür verwendet ROS eine einfache Beschreibung der Datentypen in Textdateien
\item Durch diese Beschreibung kann für unterschiedliche Sprachen Code autogeneriert
werden
\item Diese sind in .msg-Dateien im msg- Unterverzeichnis eines ROS-Pakets abgelegt
\item Eigene Message-Typen sind mit Paket Ressource-Namen bezeichnet
\item Standard Messages sind mit std\_msg/msg/String.msg bezeichnet
\end{itemize}
\item Service
\begin{itemize}
 \item ROS verwendet eine eigene vereinfachte Service Description Language ("srv") für die
Beschreibung von ROS Service-Typen
\item Setzt direkt auf die ROS msg-Format auf
\item Ermöglicht die Anfrage / Antwort-Kommunikation zwischen den Knoten
\item Service-Beschreibungen sind in .srv-Dateien im srv- Unterverzeichnis eines Pakets
gespeichert
\item Service-Beschreibungen werden für die Verwendung mit dem Paket Ressource-
Namen bezeichnet
\item Z. B.: wird die Datei robot\_srvs/srv/SetJointCmd.srv als Service
robot\_srvs/SetJointCmd bezeichnet
\end{itemize}
\item Notes
\begin{itemize}
 \item Der Nachrichtenaustausch findet bei Nodes durch 3 Möglichkeiten statt: Parameter
Server;Topics; Services
\item Nodes werden wie in einem Graph angeordnet
\item In einem System laufen viele Nodes Parallel
\item Diese werden zu Beginn gestartet
\item Beispiele sind Nodes für: Laserscanner; Kinect; Pfadplanung
\end{itemize}
\item Topica
\begin{itemize}
 \item Topics verhalten sich wie ein virtuelles BUS-System Nodes können von Topics lesen
(subscribe)
\item Nodes können an Topics senden (publish)
\item Es gibt keine Begrenzung wie viele Nodes publsih oder subscribe auf ein Topic
machen
\end{itemize}
\end{itemize}
\subsubsection*{ROS-Datensystem}
ROS-Ressourcen sind in rangmäßiger Gliederung eingeordnet. Zwei Konzepte sind zu
verstehen:
\begin{itemize}
\item \textbf{Le package}: Es handelt sich hier um die Zentraleinheit der Softwareorganisation von
ROS. Ein Package ist ein Verzeichnis der die Knoten beinhaltet (wir werden hier
unten erklären, was ein Knoten ist) sowie die externen Librairies, Daten und XML
Konfigurationsdateien die manifest.xml genannt wird.
\item \textbf{Stack}: Stack bezeichnet eine Sammlung von Packagen. Sie ermöglicht mehrere
Funktionen wie Navigation, Lokalisierung und viele mehr. Ein Stack beinhaltet
mehrere Verzeichnisse sowie eine Konfigurationsdatei die stack.xml genannt wird.
\begin{itemize}
\item Vorhandene wichtige Stacks
\begin{itemize}
\item TF – Koordinatentransformation
\item Navigationstack
\item URDF - Modelle
\end{itemize}
\begin{itemize}
\item Beispiel Navigationstack
\begin{itemize}
\item Wertet Sensordaten aus z.B.: Laserdaten
\item Baut daraus mit gmapping (ebenfalls ein ROS-Stack) eine Begehbarkeitskarte
\item Warum? Zur Kollisionsvermeidung
\item Bei erfolgreicher Erstellung einer Map kann dann ein Ziel übergeben werde (Pfadplanung durch Navigationstack,
Kollisionsvermeidung, Reaktion auf sich ändernde Umgebung, Aufbau einer globalen Karte)
\end{itemize}
\end{itemize}
\end{itemize}
\end{itemize}
\subsubsection*{Vorteile und Nachteile des ROS}
\paragraph*{Vorteile}
\begin{itemize}
 \item Nachrichten-basierte Software Architektur
\begin{itemize}
\item Verschiedene Komponenten sind unabhängig voneinander mit dem System verbunden
\item Unterschiedliche Komponenten können miteinander verbunden werden, ohne jedes Mal das Programm neu zu Kompilieren
\item Netzwerkfähigkeit
\item Einfaches Debugging und Simulieren
\end{itemize}
\item Absturz eines Nodes führt nicht zum Absturz des ganzen
Systems
\item Für ROS lässt sich in mehreren Sprachen programmieren
\item ROS hat eine große Community, die viele Daten und Programme zu Verfügung
stellen
\end{itemize}
\paragraph*{Nachteile}
\begin{itemize}
 \item Durch Nachrichten-basierte Systemarchitektur Bottleneck bei großer Datenmenge
\item Steuerung des Systems über Kommandozeile
\end{itemize}

\subsubsection{Epos Control}

\subsubsection{SOPAS Engineeringtool}

Bei SOPAS ET handelt es sich um ein Entwicklungsprogramm von Sick. Dieses wird zur Ansteuerung und Konfiguration der Laserscanner und Hallsensosren verwendet.

\begin{itemize}
\item \textbf{ Erster Start }

Beim ausführen von SOPAS wird ein neues Projekt erstellt, in dem man die gewünschte Hardware selektiert und einbindet. Die Wahl der Hardware erfolgt hierbei über den Netzwerkscanassistenten oder manuell über den Gerätekatalog. Sobald die Kommunikation mit der Hardware aktiv ist, kann diese angesteuert werden. Änderungen der Konfiguration der Hardware sind im Projektbaum möglich oder sogar notwendig ( siehe Kapitel 7.7 Herausforderungen).
\end{itemize}


\subsection{Systemarchitektur}

Das Materialflusssystem auf den \textsc{Mica}z-Modulen ist in einer Schichtenarchitektur aufgebaut. Diese ist an den Aufbau von AUTOSAR \cite{AUTOSAR:2014:Online} angelehnt. Abbildungen \ref{fig:architecture_ramp} und \ref{fig:architecture_vb} zeigen den Aufbau des Systems auf den Rampen beziehungsweise Volksbots.

\begin{figure}[h!]
 \centering
		\includegraphics[width=1\textwidth]{flow/Architektur_Rampe.png}
	\caption{Architektur der Rampe \cite{Stasch:Hahn}}
	\label{fig:architecture_ramp}
\end{figure}

\begin{figure}[h!]
 \centering
		\includegraphics[width=1\textwidth]{flow/Architektur_VB.png}
	\caption{Architektur der Volksbots aus Sicht des Materialfluss \cite{Stasch:Hahn}}
	\label{fig:architecture_vb}
\end{figure}

Ganz unten in der Hierarchie befindet sich die eigentliche \textsc{Mica}z Hardware (Hardware Level) mit allen Peripherie-Komponenten. Diese wird vom den darüber liegenden Background Level angesteuert. Im Backgrund Level befinden sich im je nach Modul hardwareabhängige Treiber für drahtlose und serielle Kommunikation, Lichtschranken, Bolzen und einen externen Flash-Speicher. Diese agieren meist auf Pin-Ebene, steuern also die einzelnen GPIOs des Mikrocontrollers. Eine Besonderheit stellt hier der Radio-Driver dar, der nicht direkt auf die Hardware, sondern auf den Kommunikationsstack des Echtzeitbetriebssystems Contiki zugreift.

Darüber finden sich Interfaces, die die Funktionen der Treiber aufbereiten und in Funktionen gliedern, die es den oberen Schichten erlauben, ohne großen Aufwand und Kenntnis der Implementierungsdetails (konkreter Ein- beziehungsweise Ausgangs-Pin, Timing, usw.) auf die Hardware zuzugreifen. Neben den Treibern und Interfaces befindet sich im Background Level auch das Echtzeitbetriebssystem Contiki OS. Dieses beinhaltet unter anderem einen Scheduler, eine Prozessverwaltung und den Kommunikationsstack \textit{Rime} (siehe \autoref{sec:rime}). 

Schließlich folgt auf der höchsten Hierarchieebene das Agent Level. Hier befindet sich zunächst das AgentRTE, eine Laufzeitumgebung für Agenten. Dieses ist weitgehend hardwareunabhängig. Lediglich bei der Prozessverwaltung gibt es noch Unterschiede die in \autoref{sec:AgentRTE} noch näher betrachtet werden.
Aufgaben des AgentRTE sind vor allem die Verwaltung aller Agenten auf der Plattform und deren Scheduling, sowie der Austausch von Nachrichten untereinander.

Das letzte Glied in der Kette bilden letztendlich die Agenten. Sie werden vom AgentenRTE verwaltet und sind grundsätzlich hardwareunabhängig. Die Agenten bilden die echte Betriebslogik des Systems ab und kommunizieren dafür untereinander mit Nachrichten. Auf jedem Modul gibt es einen Platform-, einen Order- und einen Routing-Agenten. Dazu können auf den Volksbots ein und auf den Rampen bis zu vier Paket-Agenten registriert sein.

In den folgenden Abschnitten wird nun die Implementierung des Materialflusssystems anhand dieser Architektur erläutert, beginnend beim Echtzeitbetriebssystem Contiki, über die Treiber und Interfaces hin zum AgentenRTE und schließlich den Agenten.

\subsubsection{Contiki}
Contiki ist ein quelloffenes Echtzeitbetriebssystem (RTOS: Real Time Operating System), das in dieser Projektgruppe auf den \textsc{Mica}z-Modulen eingesetzt wird \cite{Contiki:2014:Online}. Es ist speziell für die Anforderungen des Internet of Things und von Wireless Sensor Networks zugeschnitten und bietet einen einfachen ereignisgesteuerten Betriebssystemkern mit sogenannten Protothreads (Threads, die sich einen gemeinsamen Stack teilen und daher schnell gewechselt werden können), optionalem präemptives Multithreading, Interprozess-Kommunikation via Message-Passing mit Events, eine dynamische Prozessstruktur mit Unterstützung für das Laden und Beenden von Prozessen und einen nativen Kommunikationsstack für die drahtlose Kommunikation gemäß dem IEEE-Standard \textit{802.15.4} \cite{IEEE802154:2014:Online}.
 
\paragraph{Build-Vorgang}\mbox{}\\
Es existieren Implementierungen von und Treiber für Contiki für eine Vielzahl von Plattformen. Dazu gehören neben \textsc{Mica}z-Modulen auch etwa der \textit{MSP430x} von Texas Instruments oder der \textit{Atmega128 RFA1} von Atmel. Für welche Plattform ein Contiki-System und die darauf geplanten Anwendungen gebaut wird, wird zur Compile-Zeit entschieden. Das heißt, um die selbe Anwendung mit Contiki auf mehrere Plattformen zu bringen, muss die Anwendung für jede Zielplattform neu gebaut werden.

Für jede Plattform existiert dafür ein eigener Ordner im \textit{platforms}-Verzeichnis der Contiki-Quelldateien. Um nun das Zielsystem zu wählen, muss lediglich das \textit{TARGET} beim Aufruf des entsprechenden Makefiles angegeben werden und das Build-System inkludiert automatisch die passenden Treiber und Definitionen.

Projektdateien können über die Makefile-Variable \textit{PROJECT\_SOURCEFILES} hinzugefügt werden. \autoref{lst:contikimakefile} zeigt exemplarisch das Makefile der Rampen.

\lstinputlisting[language=C, style=customc, captionpos=b, caption={Makefile des Contiki-Systems der Rampen}, label=lst:contikimakefile]{src/flow/lst/makefile.lst}


\paragraph{Prozesse}\mbox{}\\
Prozesse in Contiki implementieren folgen einem Konzept namens Protothreads. Dies erlaubt es Prozessen, ohne den Speicher-Overhead und die langen 
Prozesswechselzeiten von normalen Threads auszukommen, indem sie sich einen gemeinsamen Stack auf dem Hauptspeicher teilen.
Einzige Einschränkungen dieser Entwicklung sind, dass in Prozessen keine Switch-Case-Anweisungen auftreten dürfen und dass nur statische und globale Variablen zwischen zwei Aufrufen erhalten bleiben. Dynamisch erzeugte Variablen werden dagegen überschrieben. Entsprechend sollte der Zustand eines Prozesses mithilfe von statischen Variablen gespeichert werden. \autoref{lst:process} zeigt eine solche statische Variable (i) und den vollständigen Aufbau eines Prozesses.

\lstinputlisting[language=C, style=customc, captionpos=b, caption={Einfacher Beispiel-Prozess in Contiki}, label=lst:process]{src/flow/lst/process_example.lst}

In Zeile 1 wird der Prozess initialisiert und in Zeile 2 automatisch beim Boot von Contiki gestartet. Zeile 4 beinhaltet die Deklaration. So können andere Prozesse diesem Prozess Events (mit oder ohne Daten) schicken, auf die unser Beispielprozess mit ev und data zugreifen kann. Zeile 6 kennzeichnet den Beginn der tatsächlichen Ablauflogik. Code über dieser Zeile wird bei jedem Prozessaufruf ausgeführt, dies wird jedoch in den meisten Fällen nicht benötigt. Zeile 13 schließlich beendet den Prozess und entfernt ihn aus der Prozess-Liste des Kernels. In diesem Beispiel wird die Zeile jedoch nie erreicht, sodass der Prozess immer wieder aufgerufen wird, bis er von einem anderen Prozess beendet wird.

\paragraph{Prozesskommunikation}\mbox{}\\
In Contiki kommunizieren Prozesse über Events. Auch der Kernel versendet Events, um Prozesse über ihren Zustand (Init, Continue, Exit) oder über abgelaufene Timer zu informieren. Zur Identifikation werden dabei Event IDs genutzt. Die Event IDs 0-127 können vom Benutzer frei vergeben werden, während die Prozess IDs ab 128 vom System genutzt werden. Grundsätzlich unterscheidet Contiki zwischen synchronen und asynchronen Events. 

\begin{itemize}
\item \textbf{Asynchrone Events} werden vom Kernel in einer Warteschlange gespeichert. Die Scheduling-Funktion des Kernels läuft nach Systemstart in einer Endlosschleife. In jedem Durchlauf wird ein Event aus der Schlange entnommen und an den Zielprozess weitergeleitet.
\item \textbf{Synchrone Events} gleichen einem Funktionsaufruf.
Sie werden ohne Umweg über die Warteschlange direkt an den Empfänger-Prozess
zugestellt \cite{Contiki:2014:Online}.  Mit der Funktion \textit{process\_post\_synch(\&example\_process, EVENT\_ID, msg)} wird gezielt ein Prozess aufgerufen (ein Broadcast ist nicht möglich). Während der aufgerufene Prozess aktiv ist, blockiert der Aufrufer und setzt seine Ausführung erst fort, wenn der aufgerufene Prozess die Kontrolle wieder abgibt.
\end{itemize}

Um auf Events zu reagieren, können in Prozessen die folgenden Funktionen genutzt werden:

\begin{itemize}
\item PROCESS\_WAIT\_EVENT() - Wartet auf ein beliebiges Event, bevor die Ausf\"{u}hrung fortgesetzt wird.
\item PROCESS\_WAIT\_EVENT\_UNTIL(condition) - Wartet auf ein beliebiges Event, setzt die Ausf\"{u}hrung aber nur fort, wenn die Bedingung erf\"{u}llt ist.
\item PROCESS\_WAIT\_UNTIL() - Wartet, bis die Bedingung erf\"ullt ist. Muss den Prozess nicht zwangsl\"{a}ufig anhalten.
\end{itemize}

Prozesse können neben Events auch über Polling-Anfragen kommunizieren. Polls werden bei der Bearbeitung von Hardware-Interrupts genutzt, da Interrupt-Handler keine Events absetzen dürfen. Sie können als Events mit erhöhter Priorität betrachtet werden. Ein Prozess, der einen Poll erhalten hat, wird in der Warteschlange für Prozesse priorisiert. \cite{Contiki:2014:Online, Walter:2010}.

\paragraph{Scheduling und Timer}\mbox{}\\
Grundsätzlich nutzt Contiki ein Event-getriebenes Modell von Nebenläufigkeit, wobei einzelne Events nach dem Run-To-Completion (RTC) Prinzip abgearbeitet werden. Das heißt, einmal angelaufen können Prozesse nur noch von Hardware-Interrupts unterbrochen werden oder selbst die Kontrolle abgeben. Dies ermöglicht es, alle Prozesse auf dem selben Stack arbeiten zu lassen und so Hauptspeicher zu sparen. Auf diese Weise muss kaum Speicher dynamisch alloziert werden. Außerdem werden so Race-Conditions auf geteilten Speicher nahezu ausgeschlossen. Dabei haben alle Prozesse und Events vorerst die gleiche Priorität und werden streng nacheinander in Reihenfolge abgearbeitet.

Es existiert eine Bibliothek die auch echte Threads mit jeweils einzelnen Stacks ermöglicht. Aufgrund des ohnehin knappen Hauptspeichers auf den \textsc{Mica}z-Modulen wurde diese Möglichkeit jedoch in der Projektgruppe nicht weiter betrachtet.

Ein Problem der Event-getrieben Nebenläufigkeit ist jedoch das Reaktionsvermögen auf Echtzeitanforderungen und externe Events: Sollte ein Prozess eine aufwändige Berechnung durchführen, kann es zu spät sein, bis er die Kontrolle abgibt. Aus diesem Grund führt Contiki eine zweite Prioritätsebene ein, sogenannte Polls. Diese werden zwischen asynchron auftretende Events geplant und rufen in Reihenfolge einer Priorität alle Prozesse auf, die ein Polling-Flag gesetzt haben. Üblicherweise sind dies insbesondere hardwarenahe Prozesse, die auf Änderungen an den Ein- und Ausgangspins beziehungsweise auf Timer reagieren müssen.
\paragraph{Der Rime Kommunikationsstack}\mbox{}\\
Die drahtlose Kommunikation in Contiki erfolgt über einen leichtgewichtigen Netzwerstack namens \textit{Rime} \cite{Dunkels:2007:Proc}. . Dieser übergibt seine Daten an und erhält seine Daten von der sogenannten \textit{Charmeleon}-Architektur

Der \textit{Rime}-Stack implementiert das Network- und MAC- (beziehungsweise Data Link-)Layer des ISO OSI Referenzmodells. Darunter folgt eine sogenannte \textit{Radio Duty Cycling}-Schicht (RDC), die der Stromersparnis in drahtlosen Sensornetzwerken dient: Es ermöglicht, die Übertragungs- und Empfangseinheit des Moduls auszuschalten, während es nicht benötigt wird.

Auf der physikalischen Schicht schließlich wird die Übertragungseinheit über die Ein- und Ausgangspins angesteuert. Im Falle des \textsc{Mica}z-Moduls ist dies in CC2420-Chip für paketbasierte Funkübertragung auf einer Frequenz von 2.4 GHz \cite{CC2420:2014:Online}.

\autoref{fig:rime} zeigt den kompletten Stack, ausgehend vom Radio Driver, der in der Projektgruppe implementiert wurde bis hin zum Treiber des Funkmoduls.

\begin{figure}[h!]
 \centering
		\includegraphics[width=0.7\textwidth]{flow/Rime.png}
	\caption{Der Rime Netzwerkstack \cite{Dunkels:2007:Proc}}
	\label{fig:rime}
\end{figure}

Der Radio Driver der Projektgruppe implementiert ein einfach Network-Flooding, auf das im nächsten Abschnitt näher eingegangen wird. Dieses greift schließlich auf einen \textit{Atomic Broadcast Channel} (abc) zu. Dieser einfachste Channel in Contiki fügt der Nachricht lediglich die ID des Senders und eine Time-To-Live Angabe für das Network-Flooding als Header-Informationen hinzu. Der Channel sendet das Paket an den \textit{Rime Network Layer}. Dieser ruft den Chameleon-Service auf. Dieser sorgt für eine Trennung von Header-Informationen und Ebenen und reduziert den Header, indem er redundante Informationen entfernt. 

Anschließend wird das Paket der \textit{Carrier Sense Multiple Access}-Schicht (CSMA) übergeben. Da drahtlose Kommunikation immer über ein geteiltes Medium erfolgt, muss vor dem Senden geprüft werden, ob nicht gerade ein anderer Knoten sendet. Ist dies der Fall, wird mit der Übertragung gewartet, bis das Medium wieder frei ist.

Wurde ein freies Medium erkannt, wird das Paket an den RDC-Treiber übergeben. Im Falle unser Projektgruppe implementiert dieser das \textit{X-Mac}-Protokoll für energiesparende Kommunikation in drahtlosen Sensornetzwerken \cite{Buettner:2006:Proc}. Dieses nutzt Early Acking und möglichst frühe Übertragung der Adresse, um die nötigen Wachzeiten der Übertraungseinheiten der teilnehmenden Module möglichst gering zu halten und so Strom zu sparen. X-Mac sorgt weiterhin dafür, dass auch der Empfänger aufgeweckt und damit empfangsbereit ist. Ist dies der Fall, wird das Paket schließlich an den CC2420-Driver übergeben, der die passenden Ausgangspins bedient, um den externen Funkchip anzusprechen.

Ein eingehendes Paket nimmt exakt den umgekehrten Weg: Der CC2420-Chip löst einen Interrupt aus, sobald er ein Paket empfangen hat. Dieses wird von jeder der genannten Schichten verarbeitet und bis zum Radio Driver aus der Projektgruppe weitergeleitet.
\label{sec:rime}

\subsubsection{Treiber, Services und Interfaces}
Auf dieser Ebene werden der Agenten RTE alle Funktionalitäten zur Verfügung gestellt. Das RTOS stellt Infrastruktur wie z.~B. Task Management, Timing, Events usw.
Die Driver Ebene kümmert sich um die Schnittstellen/Pins des Controllers wie z.~B. Protokolle angeschlossener
Devices\cite[S. 26]{Stasch:Hahn}. Die Aufgabe des Services ist es, Funktionen für spezielle Anwendungen zur Verfügung zu stellen.
Darüber kommen die Interfaces. Diese sind nötig, damit bestimmte Funktionen immer gleich der Agenten RTE zur Verfügung gestellt werden, auch wenn der Service anders ist oder wenn ein Service verschiedene Interfaces bedienen soll\cite[S. 26]{Stasch:Hahn}. Im Laufe der Projektarbeit wurden unterschiedliche Treiber, Services und Interfaces selbst geschrieben. Der folgende Abschnitt soll diese etwas näher beleuchten.

\paragraph{Treiber Bolzen}

\paragraph{Interface Bolzen}

\paragraph{Treiber Externer Speicher}

\paragraph{Interface Externer Speicher}

\paragraph{Service Externer Speicher}

\paragraph{Treiber Lichtschranken}

\paragraph{Interface Lichtschranken}

\paragraph{Treiber Funkmodul}

\paragraph{Treiber UART-Schnittstelle}

\paragraph{Interface UART-Schnittstelle}


\subsubsection{Agenten RTE}

Zur Umsetzung der dezentralen Steuerung werden Softwareagenten eingesetzt. Bei Software-Agenten handelt es sich um Prozesse, die lose gekoppelt 
und leicht austauschbar sind \cite[vgl.][S. 31-37]{GH:2010}. Es existieren verschiedene Definitionen eines Agenten, von denen
sich keine als Standard etablieren konnte. Die hier verwendete Definition stammt von
Brenner, Zarnekow und Wittig. Sie definieren einen Agenten als „ ... ein Softwareprogramm,
das für einen Benutzer bestimmte Aufgaben erledigen kann und dabei einen Grad an
Intelligenz besitzt, der es befähigt, seine Aufgaben in Teilen autonom durchzuführen und mit
seiner Umwelt auf sinnvolle Art und Weise zu interagieren“\cite{BZW:1998}. Die Fähigkeit von Agenten, miteinander zu kommunizieren 
und zu interagieren, ermöglicht das Erstellen eines Multiagentensystems (MAS). Ein wesentlicher Vorteil von MAS 
bzw. von verteilten Steuerungssystemen ist die Fähigkeit, dynamisch auf Veränderungen zu reagieren. Ein Beispiel für eine solche Veränderung ist der
Ausfall einer Steuerungseinheit bzw. eines Agenten. Der Ausfall einer Einheit hat nicht unbedingt zur Folge, dass das gesamte System ausfällt. 
Die restlichen Einheiten können sich eigenständig auf eine solche Veränderung einstellen und diese beim weiteren Ablauf
berücksichtigen\cite[S. 13]{Roidl:2012}. Diese Eigenschaft bringt eine ganze Reihe von Vorteilen für ein dezentral
gesteuertes Materialflusssystem mit sich.\\
Die Modellierung von agentenbasierten Systemen für industrielle Bereiche wird durch die
Entwicklung von Standards festgelegt. Diese beschreiben Modelle für die Architektur
sowie die Kommunikation zwischen Agenten. Die FIPA (Foundation of Intelligent Physical Agents) ist das Standardisierungsorgan für Agentensysteme.
Seit der Gründung 1996 in der Schweiz wurden verschiedene Standardisierungen veröffentlicht, so zum Beispiel auch die Agentenkommunikation (Agent Communication), die als FIPA/ACL (Agent Communication Language, ACL) bekannt geworden ist. Jeder Agent ist mit einem eindeutigen
Identifikationsnamen (Agent Identifier oder AID) versehen und wird im Agent Management System verwaltet\cite[S. 24]{Roidl:2012}.
Ein Verzeichnisdienst kann optional vom Directory Facilitator gestellt
werden. Die Komponenten sind über das Message Transport System (MTS) verbunden,
sodass alle Komponenten untereinander kommunizieren können\cite[S. 24]{Roidl:2012}. \\
Als Referenzmodell zur Verwaltung von Softwareagenten wurde die Softwarearchitektur der Materialflussgruppe laut den FIPA Standards aufgebaut. 
Die nächste Abbildung zeigt den Aufbau der Softwarearchitektur und ist an die AUTOSAR Softwarearchitektur angelehnt:
\begin{figure}[h!]
	\centering
		\includegraphics[width=0.9\textwidth]{ArchitekturMicazRampe.png}
	\caption{Architektur Micaz Rampe\cite{Stasch:Hahn}}
	\label{ArchitekturMicazRampe}
\end{figure}
\paragraph{Hardware Level}
Auf der Hardwareebene werden die Treiber für die Steuerung der Lichtschranke und Bolzen implementiert. Die Treiber bekommen ein
allgemeines Interface für die Agenten RTE.
\subsection{Agenten}
Oberhalb des AgentRTE sind die Agenten implementiert. Sie bilden die oberste Ebene des Systems und implementieren die eigentliche Funktionalität. Dabei greifen sie auf das AgentenRTE und die verschiedenen Interfaces zu und kommunizieren untereinander über Agenten-Nachrichten. Auf jedem Modul agieren ein Plattform-Agent, der die Sensoren und Aktoren der Plattform steuert, ein Order-Agent, der zusammen mit anderen Order-Agenten die Betriebslogik des Materialflusssystems darstellt, ein Routing-Agent, der gemeinsam mit anderen Routing-Agenten die Pakete durch das System leitet und eventuell bis zu vier Paket-Agenten, die die physischen Pakete repräsentieren und durch das System wandern. Die einzelnen Agenten sind im Folgenden beschrieben.
\subsubsection{Paket-Agent}
Ein Paket-Agent repräsentiert ein physisches Paket. Seine ID ist gleichzeitig auch Paketnummer. Beim Eintritt in das System wird vom Plattform-Agenten ein neuer Paket-Agent initialisiert. Wechselt ein Paket von einem Modul auf das nächste, so wandert auch der Paket-Agent auf das neue Modul. Dies wird erreicht, indem der Agent auf dem einen Modul beendet und auf dem nächsten mit seinem Ziel und seiner ID neu initialisiert wird. Dies ist möglich, da sich verschiedene Pakete nur durch ihr Ziel und ihre ID voneinander unterscheiden. Die Übertragung der Paket-Agenten wird von den Plattform-Agenten der jeweiligen Module übernommen.

Das Ziel von Paket-Agenten ist dynamisch. Es wird von den Order-Agenten verwaltet. Kommt ein neuer Auftrag ins System, wird vom Order-Agenten, der den Auftrag verwaltet, eine Nachricht an das entsprechende Paket gesendet. Erhält der Paket-Agent eine solche Nachricht, wird dem Order-Agenten der Empfang bestätigt und das eigene Ziel wird angepasst.

Hat der Paket-Agent ein gültiges Ziel und wird ihm vom Plattform-Agenten per Flag die Erlaubnis erteilt, sendet er dem Routing-Agenten seiner Plattform eine Routing-Anfrage, bestehend aus dem Ziel des Pakets. Wenn diese nicht abgelehnt wird, wartet der Paket-Agent, bis sein Ziel geändert wird, oder er sich auf einer Plattform befindet, die nicht seinem aktuellen Ziel entspricht und er die erneute Erlaubnis bekommt, eine Routing-Anfrage zu stellen. Wird die Anfrage dagegen abgelehnt, stellt er beim nächsten Aufruf eine neue, bis die Anfrage vom Routing-Agenten angenommen wird.

Schließlich kann der momentane Zustand des Pakets über eine sogenannte \textit{UPDATE\_PHYSICAL}-Nachricht von einem Gateway abgefragt werden. Der Agent antwortet darauf mit der ID der Plattform, auf der er sich zur Zeit befindet und dem eigenen Ziel.
\subsubsection{Plattform-Agent}
Der Plattform-Agent ist der einzige plattformabhängige Agent des Systems. Während alle anderen Agenten auf allen Modultypen identisch sind, ist der Plattform-Agent modulspezifisch. Seine Aufgabe ist die Steuerung und Überwachung seines Moduls.

Beiden gemeinsam ist jedoch die Übertragung, also das Beenden und die erneute Initialisierung, von Paket-Agenten. Diese wird immer vom Volksbot initialisiert. Es wird eine Anfrage an den Plattform-Agenten der jeweiligen Rampe gesendet, entweder nach einem Lagerplatz oder nach einem speziellen Paket, abhängig davon, ob ein Paket abgelegt oder aufgenommen werden soll. Soll ein Paket abgelegt werden, muss die Anfrage bestätigt werden, bevor ein Registrierungs-Auftrag mit den Details des Paket-Agenten gesendet wird und der Agent auf seiner momentanen Plattform terminiert wird. Soll ein Paket aufgenommen werden, prüft die Rampe, ob das Paket mit der geforderten ID vorhanden ist und ausgegeben werden kann und sendet im Erfolgsfall ebenfalls einen Registrierungs-Auftrag und terminiert seinerseits den  entsprechenden Paket-Agenten, der dann auf dem Volksbot neu initialisiert wird.

Außerdem geben beide auf eine \textit{UPDATE\_PHYSICAL}-Nachricht die IDs aller Paket-Agenten zurück, die sich derzeit auf dem Modul befinden.

Im Folgenden werden nun die plattformspezifischen Eigenschaften der Plattform-Agenten beschrieben.

\paragraph{Plattform-Agent der Rampen}\mbox{}\\
Der Plattform-Agent auf einer Rampe ist neben der Verwaltung der Paket-Agenten auch für die Steuerung und das Auslesen der Bolzen über das Bolt\_Interface beziehungsweise Lichtschranken über das Photosensor\_Interface verantwortlich. Er sorgt dafür, dass die Pakete korrekt vereinzelt werden und verwaltet ihre Reihenfolge. Außerdem prüft er auf die Anfrage eines Volksbots, ein Paket abzulegen, anhand der Lichtschranken, ob noch ein Platz auf der Rampe zur Verfügung steht.
\paragraph{Plattform-Agent der Volksbots}\mbox{}\\
Der Plattform-Agent auf einem Volksbot kommuniziert mit dem Laptop, der den Volksbot steuert. Er erhält eine Nachricht, wenn der Volksbot seine Ziel-Position erreicht hat. Daraufhin initiiert er die Übergabe des Paketes. War die Übergabe des Paket-Agenten erfolgreich, sendet der dem Laptop eine Nachricht über die serielle UART-Nachricht, um das Fließband auf dem Volksbot zu starten und die physische Übernahme des Pakets zu starten.
\subsubsection{Order-Agent}
Die Order-Agenten übernehmen die Rolle eines zentralen Materialflussrechners im Materialflusssystem. Ihre Aufgabe ist die Verarbeitung von Aufträgen, sprich die Zuweisung von Zielen an Pakete. Aufträge bestehen aus Paket- und Ziel-ID und werden über ein Gateway in das System eingegeben. Dafür wird eine entsprechende Agenten-Nachricht an einen einzelnen Order-Agenten gesendet, der den Empfang bestätigt oder ablehnt, falls kein Platz in seinem Speicher zur Verfügung stand. In diesem Fall muss ein anderer Order-Agent mit dem Auftrag betraut werden. Ein Auftrag wird mit seiner Paket- und Ziel-ID sowie einem Status ("`zu bearbeiten"' beziehungsweise "`in Verteilung"') in einer Warteschlange ablegt.

Hat der Order-Agent in einem Aufruf keine Nachricht erhalten, durchsucht er diese Warteschlange nach zu bearbeitenden Aufträgen und sendet eine Nachricht an das entsprechende Paket, sein Ziel zu ändern. Wird der Empfang dieser Nachricht vom Paket bestätigt, wird der Auftrag gelöscht, von nun an ist das Paket für die Erreichung seines Ziels verantwortlich. Sind alle Aufträge in Verteilung muss davon ausgegangen werden, dass die entsprechenden Nachrichten nicht angekommen sind oder die Pakete noch nicht im System sind. Daher wird in diesem Fall der Status aller Aufträge zurückgesetzt und es wird erneut versucht, Nachrichten an die einzelnen Pakete zu senden.
\subsubsection{Routing-Agenten}

Die Routing-Agenten kümmern sich um die Wegplanung der Pakete im Materialflusssystem. Sie suchen nach einem Volksbot, der ein Paket möglichst günstig zu seinem Ziel bringen kann. Dafür führen sie untereinander eine Auktion durch, bei der ein Transportauftrag zu möglichst geringen Kosten an einen Volksbot vergeben wird. Ein Routing-Agent reagiert dabei auf die Routing-Anfrage eines Paket-Agenten. Ein Routing-Agent kann gleichzeitig nur an einer Auktion teilnehmen beziehungsweise diese initiieren. Hiermit wird verhindert, dass ein Routing-Agent gleichzeitig zwei Auktionen gewinnt und deshalb eine der Auktionen zurückgerollt und wiederholt werden muss.

Die Routing-Agenten werden als kommunizierende Zustandsautomaten implementiert. Zustandsübergänge können durch eingehende Nachrichten oder Timer ausgelöst werden. \autoref{fig:routing_agent_fsm} zeigt den zugrunde liegenden Automaten.

\begin{figure}[h!]
  \centering
    \includegraphics[width = 1.35\textwidth, angle=90]{flow/RoutingAgent_FSM.png}
    \caption{Zustandsautomat des Routing Agenten}
    \label{fig:routing_agent_fsm}
\end{figure}

Ein Routing-Agent startet stets in Zustand 0. In diesem Zustand wartet er auf eingehende Routing-Anfragen. Diese können entweder von Paketen auf der eigenen Plattform oder von anderen Routing-Agenten kommen. Sie unterscheiden sich im ersten Byte der Conversation-ID einer Agenten-Nachricht. Hier ist die Auktions-ID gespeichert, die für neue Routing-Anfragen von Paketen erst durch den Routing-Agenten bestimmt werden muss und daher auf 0 gesetzt ist. 

Bei Eingang einer Routing-Anfrage durch ein Paket wird eine neue Routing-Anfrage an alle Routing-Agenten versendet, ein Timer gestartet, der abläuft, wenn die Bearbeitungszeit erreicht ist, und in Zustand 3 übergegangen. Kommt die Anfrage von einem anderen Routing-Agenten prüft der Agent, ob das Ziel erreichbar ist. Für Module auf einem Volksbot ist dies immer wahr, für Module an Rampen immer falsch. Ist das Ziel erreichbar, wird eine UART-Nachricht an den Volksbot gesendet, der die Kosten der Fahrt vom anfragenden Routing-Agenten zum Ziel des Pakets berechnen soll. Anschließend geht der Automat in Zustand 1 über.

In Zustand 1 wartet der Agent auf die Antwort des Volksbots auf seine Kostenanfrage. Geht diese ein und sind die Kosten größer als null, bedeutet dies, dass das Ziel erreichbar ist. Der Agent sendet anschließend diese Kosten an den Initiator der Auktion und geht in Zustand 2 über, wo er auf die Antwort des Initiators wartet. Wird das Angebot bestätigt, wird der Volksbot beauftragt, sich zum Ausgang des Initiators zu bewegen und der Automat geht in Zustand 6 über. Wird die Anfrage abgelehnt, geht der Automat zurück in Zustand 0 und wartet auf neue Anfragen. In Zustand 6 wartet der Routing Agent auf eine Nachricht des Plattform-Agenten, der bestätigt, dass das Paket abgegeben wurde und neue Anfragen angenommen werden können.

In Zustand 3 wartet der Routing-Agent auf Angebote von anderen Routing-Agenten, nachdem er eine Routing-Anfrage für ein Paket auf der eigenen Plattform verschickt hat. Er speichert diese Angebote mit Absender-ID und Kosten. Läuft schließlich der Bearbeitungs-Timer ab, wird geprüft ob mindestens ein Angebot eingangen ist. Ist das der Fall, wird das beste Angebot bestimmt und der Agent geht in Zustand 4 über. Andernfalls wird ein neuer Timer gesetzt, bis die Anfrage wiederholt wird und der Agent geht in Zustand 5. In Zustand 5 wartet der Agent auf den Ablauf des Timers, versendet die Routing-Anfrage erneut an alle Routing-Agenten und geht zurück in Zustand 3. Zustand 4 dagegen bleibt aktiv, bis alle Teilnehmer der Auktion benachrichtigt wurden. In jedem Aufruf des Agenten wird eine Nachricht mit Cancel oder Acknowledge an einen weiteren Teilnehmer der Auktion gesendet. Sind alle Teilnehmer benachrichtigt, geht der Agent zurück in Zustand 0 und wartet auf neue Anfragen.



\subsection{Validierung}

\subsubsection{Erreichte Funktionalität}

\subsubsection{Probleme und Herausforderungen}

\subsubsection{Ausblick}


	
		% Chapter 7 Teilberichtfarhrzeug
	\clearpage
	\ohead[Teilberichtfarhrzeug]{Teilberichtfarhrzeug}
	\chead[Uni Oldenburg]{Uni Oldenburg}
	\ihead[PG FAISE]{PG FAISE}
	\setheadtopline{1pt}
	\setheadsepline{0.5pt}
	\ofoot[Endbericht]{Endbericht}
	\cfoot[\pagemark]{\pagemark}
	\ifoot[31. Oktober 2014]{31. Oktober 2014}
	\setfootsepline{0.5pt}
	\setfootbotline{1pt}
	\section{Teilbericht Fahrzeuge}

Ziel der Teilgruppe Fahrzeuge ist es, einen autonomen Transport zwischen den Rampen zu realisieren. Dafür sollen die Volksbots in der Lage sein, auf die eingehenden Aufträge zu reagieren.
Dies beinhaltet die Teilnahme an Auktionen ( Jobverteilung ), welche durch eine Aufwandsabschätzung des Auftrages eines jeden Volksbots entschieden werden. Nach der Zuweisung an den besten geeigneten Volksbot soll dieser das Paket von der entsprechenden Startrampe holen und zur Zielrampe transportieren.

Folgendes wird in den nächsten Unterkapiteln behandelt:

\begin{itemize}
	\item Anforderungskatalog an das Endsystem
	\item Beschreibung der Komponenten
	\item Werkzeuge
	\item Architektur
	\item Implementierung
	\item Erreichte Funktionalitäten
	\item Herausforderungen
	\item Ausblick
\end{itemize} 

\subsection{Anforderungen}

\begin{itemize}

\item \textbf{Aufträge}: Aufträge enthalten den Start- und Zielpunkt, welche in der Umgebungskarte gesetzt werden. Diese werden für weitere Berechnungen verwendet.

\item \textbf{Bieten}: Ein Volksbot ist in der Lage an einer Auftragsverteilung teilzunehmen, sofern er noch keinen Auftrag ausführt und sein Energievorrat nicht das kritische Minimum erreicht hat. Die Teilnahme beinhaltet eine Aufwandsschätzung anhand einer Distanzfunktion, sowie dem aktuellen Energiezustand. 

\item \textbf{Planung}: Bei der Erteilung eines Auftrages und dem Bieten für einen Auftrag, muss der Volksbot die Route vom aktuellen Standort zum Startpunkt berechnen. Dies gilt ebenso vom Startpunkt zum Endpunkt.

\item \textbf{Navigation}: Für die Routenplanung soll der kürzeste Weg verwendet werden, sofern möglich die direkte Verbindung zum gewüschten Ziel. Die Planung, Navigation und Positionierung erfolgt dabei auf dem entsprechenden Notebook.

\item \textbf{Positionierung}: Die Umgebungskarte ist jedem Volksbot bekannt. Bei Bewegungen aktualisiert dieser seine Position mittels Odometrie, um die geplante Route korrekt zu befahren. Zur lokalen Unterstützung werden die Laserscanner verwendet werden. Startpunkt eines Volksbots wird statisch definiert.

\item \textbf{Synchronisation}: Jegliche Information eines Volksbots wird an die Simulation übermittelt. Dies beinhaltet neben der aktuellen Position auch den Ladezustand, sowie den Energievorrat eines Volksbots.

\item \textbf{Feinsteuerung}: Das genaue Heranfahren an ein Objekt, sei es die Ladestation oder eine Rampe, erfolgt mit Hilfe der Laserscanner. 

\item \textbf{Überabe}: Bei Einnahme der korrekten Position zum Be- oder Entladen des Volksbots findet ein Datenaustausch mit der entsprechenden Rampe statt. Es folgt eine kooperative Interaktion beider, bis der Volksbot das Paket erhalten oder abgeladen hat.

\item \textbf{Energiemanagement}: Sobald ein Volksbot seinen kritischen Energiezustand erreicht, werden alle Auftragsverteilungen ignoriert und der Bot setzt die Dockingstation als primäres Ziel. Sollten mehrere Bots die Ladestation ansteuern, oder diese bereits belegt sein, so wird der Folgeablauf durch eine Queue oder ein anderes Verfahren geregelt.

\item \textbf{Kollisionsvermeidung}: Sobald eine Kollision mit einem festen oder mobilen Objekt erkannt wird, wird eine Neuberechnung, Umplanung oder ein Ausweichverfahren eingeleitet um entsprechend zu reagieren.

\end{itemize}



\subsection{Beschreibung der Komponenten}

Bei dem Volksbot handelt es sich um einen modular aufgebauten und mobilen Roboter, der vom Frauenhofer-Institut IAIS entwickelt wurde. Die in diesem Projekt verwendeten Prototypen bestehen aus drei zentralen Elementen. 

	\begin{figure}[h!]
		\centering
			\includegraphics[width=0.9\textwidth]{Volksbot.png}
			\caption{Darsellung eines Volksbots}
			\label{Volksbot}
	\end{figure}	

\begin{itemize}

\item \textbf{Fahreinheit}
Die Fahreinheit wird aus zwei Maxonantrieben links und rechts, sowie dem Gerüst des Bots gebildet. Vorne befindet sich ein SICK LMS100 Laserscanner,  hinten ein SICK TiM310 Laserscanner. Angesteuert werden die Hardwarekomponenten mit Hilfe von vier Epos2 Controllern, die mit Hilfe einer CAN-Verbindung ansteuerbar sind. 

\item \textbf{Hub-Förderband}
Zum Transportieren der Pakete ist der Volksbot mit einer Hubeinheit ausgestattet, die zum Schutz vor Überdrehungen zwei Hallsensor auf beiden seiten beinhaltet. Des weiteren befindet sich auf der Hubeinheit ein Förderband, an dem sich zwei Lichtschranken befinden. Diese sollen zur Ermittlung des Beladungszustandes dienen. 

\item \textbf{Steuereinheit}
Gesteuert wird das System mit Hilfe eines Notebooks, welches unter Ubuntu mit Hilfe des Robot Operating System (ROS) die Hardware anspricht. Auf dem Notebook werden ausserdem sämtliche 
Berechnungen durchgeführt. Neben der Ansteuerung der EPOS2 Controller wird der SICK LMS100 über eine Ethernet Schnittstelle angeschlossen. Zur Kommunikation mit den Rampen wird zusätzlich ein MICAZ Modul verwendet, welches mittels einer USB Schnittstelle mit dem Notebook verbunden ist.

\end{itemize}
\subsection{Werkzeuge}
Beschreibung ROS. 
Wichtige Funktionen in ROS:
\begin{itemize}
\item Publish
\item Subscribe
\item ...
\end{itemize}
\subsubsection{Robot Operating System (ROS)}

\subsubsection{Was ist ROS}
Es gibt viele Robotik-Frameworks, die spezifisch für präzise Anwendungen, für Prototypen,
erstellt wurden. ROS strebt da eher das Allgemeine an. Das »Robot Operating System«
(ROS) ist ein Open-Source Framework für individuelle Roboter, das sich in der
Robotikforschung in den letzten Jahren etabliert hat und ein großes Repertoire an Software-Komponenten und -Werkzeugen für Robotikapplikationen bietet.\\
Die Entwicklung begann 2007 am Stanford Artificial Intelligence Laboratory im Rahmen des
Stanford-AI-Robot-Projektes (STAIR). Heute wird es hauptsächlich am Robotik Institut Willow
Garage weiterentwickelt. Seit April 2012 wird ROS von der neu gegründeten,
gemeinnützigen Organisation Open Source Robotics Foundation (OSRF) unterstützt. Die
Bibliotheken von ROS setzen auf Betriebssysteme wie Linux, Mac OS X oder Windows auf.
ROS ist nicht von einer spezifischen Sprache abhängig. Heutzutage gibt es 3 Grundlibraries
für ROS, die jeweils auf Python, Lisp und C++ ausgerichtet sind. Zwei Exmperimentier-
Librairies sind für Java und Lua erhältlich.
\subsubsection{Was will ROS?}
\begin{itemize}
 \item ROS will unterstützen, Code für Forschung und Entwicklung wiederzuverwenden
 \item loser Verbund von individuellen Programmteilen (Nodes)
 \item einzelne Programmteile können einfach geteilt und verbreitet werden (Packages und Stacks)
 \item ROS stellt Repositories zu Verfügung, um dort Code zu teilen \cite{ROS:2014:Online}
(http://www.ros.org/browse)
\end{itemize}
\subsubsection{Was kann ROS?}
Die Hauptbestandteile und Hauptaufgaben von ROS sind Hardwareabstraktion; Gerätetreiber; Implementierung 
von viel genutzten Funktionalitäten; Inter-Prozess-Kommunikation; Paket-Management
\subsubsection{Aufgaben des ROS}
\begin{itemize}
 \item Interprozesskommunikation (IPC)
 \begin{itemize}
\item Problematik der Kommunikation zwischen verschiedenen Systemen des Roboters
\item Sicherheitseinstellung bei der Übertragung
\item Anforderung an die Geschwindigkeit / Schnelligkeit der Kommunikation
\item Koordination von Nachrichten durch zentralen Master
\end{itemize}
\item Paketverwaltung – Packages
\begin{itemize}
 \item ROS ist durch Softwarepakete (sogn. Packages) aufgebaut
 \item Ein Package beinhaltet Laufzeitprozesse (Nodes); ROS abhängige Bibliotheken;
Datensätze; Konfigurationsdateien;3rd Party Software
 \item Packages sind dazu, da um Code wiederverwendbar zu machen
\end{itemize}
\item Paketverwaltung – Stacks
\begin{itemize}
\item Sammlung von Paketen (Packages)
\item Der Sinn ist, dass Stacks die Verteilung und Verwendbarkeit von Code
vereinfachen
\item Meist viele Packages ähnlicher Aufgaben in einem Stack verpackt
\end{itemize}
\item Message (msg)
\begin{itemize}
 \item  Messages werden verwendet um unter ROS Nachrichten zwischen Knoten und
Topics auszutuaschen
\item Dafür verwendet ROS eine einfache Beschreibung der Datentypen in Textdateien
\item Durch diese Beschreibung kann für unterschiedliche Sprachen Code autogeneriert
werden
\item Diese sind in .msg-Dateien im msg- Unterverzeichnis eines ROS-Pakets abgelegt
\item Eigene Message-Typen sind mit Paket Ressource-Namen bezeichnet
\item Standard Messages sind mit std\_msg/msg/String.msg bezeichnet
\end{itemize}
\item Service
\begin{itemize}
 \item ROS verwendet eine eigene vereinfachte Service Description Language ("srv") für die
Beschreibung von ROS Service-Typen
\item Setzt direkt auf die ROS msg-Format auf
\item Ermöglicht die Anfrage / Antwort-Kommunikation zwischen den Knoten
\item Service-Beschreibungen sind in .srv-Dateien im srv- Unterverzeichnis eines Pakets
gespeichert
\item Service-Beschreibungen werden für die Verwendung mit dem Paket Ressource-
Namen bezeichnet
\item Z. B.: wird die Datei robot\_srvs/srv/SetJointCmd.srv als Service
robot\_srvs/SetJointCmd bezeichnet
\end{itemize}
\item Notes
\begin{itemize}
 \item Der Nachrichtenaustausch findet bei Nodes durch 3 Möglichkeiten statt: Parameter
Server;Topics; Services
\item Nodes werden wie in einem Graph angeordnet
\item In einem System laufen viele Nodes Parallel
\item Diese werden zu Beginn gestartet
\item Beispiele sind Nodes für: Laserscanner; Kinect; Pfadplanung
\end{itemize}
\item Topica
\begin{itemize}
 \item Topics verhalten sich wie ein virtuelles BUS-System Nodes können von Topics lesen
(subscribe)
\item Nodes können an Topics senden (publish)
\item Es gibt keine Begrenzung wie viele Nodes publsih oder subscribe auf ein Topic
machen
\end{itemize}
\end{itemize}
\subsubsection{ROS-Datensystem}
ROS-Ressourcen sind in rangmäßiger Gliederung eingeordnet. Zwei Konzepte sind zu
verstehen:
\begin{itemize}
\item \textbf{Le package}: Es handelt sich hier um die Zentraleinheit der Softwareorganisation von
ROS. Ein Package ist ein Verzeichnis der die Knoten beinhaltet (wir werden hier
unten erklären, was ein Knoten ist) sowie die externen Librairies, Daten und XML
Konfigurationsdateien die manifest.xml genannt wird.
\item \textbf{Stack}: Stack bezeichnet eine Sammlung von Packagen. Sie ermöglicht mehrere
Funktionen wie Navigation, Lokalisierung und viele mehr. Ein Stack beinhaltet
mehrere Verzeichnisse sowie eine Konfigurationsdatei die stack.xml genannt wird.
\begin{itemize}
\item Vorhandene wichtige Stacks
\begin{itemize}
\item TF – Koordinatentransformation
\item Navigationstack
\item URDF - Modelle
\end{itemize}
\begin{itemize}
\item Beispiel Navigationstack
\begin{itemize}
\item Wertet Sensordaten aus z.B.: Laserdaten
\item Baut daraus mit gmapping (ebenfalls ein ROS-Stack) eine Begehbarkeitskarte
\item Warum? Zur Kollisionsvermeidung
\item Bei erfolgreicher Erstellung einer Map kann dann ein Ziel übergeben werde (Pfadplanung durch Navigationstack,
Kollisionsvermeidung, Reaktion auf sich ändernde Umgebung, Aufbau einer globalen Karte)
\end{itemize}
\end{itemize}
\end{itemize}
\end{itemize}
\subsubsection{Vorteile und Nachteile des ROS}
\paragraph{Vorteile}
\begin{itemize}
 \item Nachrichten-basierte Software Architektur
\begin{itemize}
\item Verschiedene Komponenten sind unabhängig voneinander mit dem System verbunden
\item Unterschiedliche Komponenten können miteinander verbunden werden, ohne jedes Mal das Programm neu zu Kompilieren
\item Netzwerkfähigkeit
\item Einfaches Debugging und Simulieren
\end{itemize}
\item Absturz eines Nodes führt nicht zum Absturz des ganzen
Systems
\item Für ROS lässt sich in mehreren Sprachen programmieren
\item ROS hat eine große Community, die viele Daten und Programme zu Verfügung
stellen
\end{itemize}
\paragraph{Nachteile}
\begin{itemize}
 \item Durch Nachrichten-basierte Systemarchitektur Bottleneck bei großer Datenmenge
\item Steuerung des Systems über Kommandozeile
\end{itemize}

\subsection{Architektur}
Volksbot: ROS, MAXON, MICAZ
\subsection{Implementierung}
Einführung in den Quellcode
\begin{itemize}
\item Was machen die Dateien? 
\item Diagramm Michael

folgt Sonntag im laufe des Tages


\end{itemize}
Wegfindung/ Navigation
\begin{itemize}
\item Odemetrie, Laserscan
\end{itemize}
Hub, Flow
Kommunikation Volksbot, Micaz, Materialfluss

\subsection{Erreichte Funktionalitäten}

Aktualisieren!!
Dem Volksbot wird manuell eine Position auf der Karte zugewiesen. Danach ist dieser in der Lage Ziele, die ihm auf der Karte angegeben werden, mit Hilfe des Dijkstra Algorithmus anzusteuern und durch die Berechnun einer Route zu erreichen. Er kann ausserdem mit den Rampen über das MICAZ Modul kommunizieren und somit einfache Aufträge annehmen. Mit Hilfe der Odometrie kennt er seine aktuelle Position, wobei der LMS100 zur Unterstützung verwendet wird. Einfache Hindernisse, die im Sichtfeld des Volksbots liegen werden umfahren und die Route zum entsprechenden Ziel wird fortgesetzt. Beim Anfahren einer Rampe kann er die mittels des Hubs die Pakete auf die entsprechende Höhe bringen und durch das Fließband an die Rampe übergeben oder dieses aufnehmen.

\subsection{Herausforderungen}:

\begin{itemize}

\item Durch die Fehleranfälligkeit der einfachen Odometrieberechnung aus den Motordaten, wurde die in ROS vorhandene \textit{adaptive Monte Carlo localization} Methode zur Verbesserung der Selbstlokalisierung verwendet.

\item Die Erkennung von Hindernissen muss angepasst werden, da zwar die Räder nicht mit anderen Hindernissen kollidieren, dafür der Rahmen des Hubs bei abstehenden Hindernissen oberhalb des Laserscanns weiterhin Kollisionen verursachen kann.

\item Die optimale Ansteuerung des Maxon Controller erwies sich als komplizierter wie erwartet, da kleine Werteänderungen schon zum Absturz des Volksbots führten und behoben werden mussten.

\item ROS bietet zwar viele nützliche Erweiterungen, deren Parametrisierung teilweise sehr aufwendig und unübersichtlich war.

\item Die Programmierung und Parametrisierung der Teilfunktionen musst mit Blick auf die CPU-Auslastung der Steuereinheit geschehen, da einige Berechnungen bei falscher Verwendung zu sehr hohen Auslastungen und einer unzureichenden Performance des Roboters führten.

\end{itemize}


\subsection{Ausblick}

Der Volksbot sollte um folgende Funktionalitäten erweitert werden:

\begin{itemize}

\item \textbf{Genauigkeit bei der Übergabe}: Die Genauigkeit des Heranfahrens an eine Zielposition, wie z.B. den Ausgang einer Rampe, sollte verbessert werden, um eine erfolgreiche Übergabe zu gewährleisten. Schon kleinere Abweichungen können dazu führen, dass Pakete an einigen Stellen festhängen oder sogar zu Boden fallen.

\item \textbf{Kollisionserkennung mobile Hindernisse}: Der Volksbot sollte Hindernisse, die sich selbstständig bewegen, erkennen und entsprechend reagieren. Dies könnte mittels einer Neuplanung der Route, einer kurzen Unterbrechung der Fahrt oder eines Ausweichmanövers umgesetzt werden. 

\item \textbf{Rückkopplung an Simulation}: Die Volksbots sollten ihre aktuelle Position, sowie den Beladungs- und Eneriezustand an die Simulation zurückgeben.

\item \textbf{Fahralgorithmus}: Es können alternative Alorithmen implementiert werden, um die Berechnungszeiten zu verkürzen oder optimalere Routen zu finden. 

\item \textbf{Positionen}: Die Positionen der Rampen und Volksbots sollten auf der Karte bekannt gegeben werden, damit der Volksbot in der Lage ist, durch die Daten des Auftrages selbstständig sein Ziel anzusteuern.

\item \textbf{Kostenabschätzung}: Eine optimale Berechnung zur Bestiimmung der benötigten Zeit, Strecke und Energieverbrauchs würde die Jobverteilung effizienter gestalten.

\item \textbf{Ladestation}: Die Ladestation muss angebracht und getestet werde. Danach fehlt die automatische Erkennung des kritischen Zustandes, sowie das selbstständige Landen des Speichers.

\end{itemize}




	
	% Chapter Integration
	\clearpage
	\ohead[Integration der Teilsysteme]{Integration}
	\chead[Uni Oldenburg]{Uni Oldenburg}
	\ihead[PG FAISE]{PG FAISE}
	\setheadtopline{1pt}
	\setheadsepline{0.5pt}
	\ofoot[Endbericht]{Endbericht}
	\cfoot[\pagemark]{\pagemark}
	\ifoot[31. Oktober 2014]{31. Oktober 2014}
	\setfootsepline{0.5pt}
	\setfootbotline{1pt}
	\section{Integration}
Ablauf zwischen Flow und Drive.
	
		% Chapter 9 Ergebnisse
	\clearpage
	\ohead[Ergebnisse]{Ergebnisse}
	\chead[Uni Oldenburg]{Uni Oldenburg}
	\ihead[PG FAISE]{PG FAISE}
	\setheadtopline{1pt}
	\setheadsepline{0.5pt}
	\ofoot[Endbericht]{Endbericht}
	\cfoot[\pagemark]{\pagemark}
	\ifoot[31. Oktober 2014]{31. Oktober 2014}
	\setfootsepline{0.5pt}
	\setfootbotline{1pt}
	\section{Ergebnisse}
Dieses Kapitel gibt einen abschließenden Überblick über die erreichten Ergebnisse der Projektgruppe. Zunächst sollen zusammenfassend beschrieben werden, welche Komponenten erfolgreich entwickelt wurden. Im Fazit werden die aus der Projektarbeit gewonnenen Erkenntnisse und Erfahrungen beschrieben. Der letzte Teil dieses Kapitels beinhaltet eine Vision, in der potenzielle Weiterentwicklungen und Ideen aufgegriffen werden. 

\subsection{Realisierte Ziele}
%Todo Was haben Materialfluss und Fahrzeuge letztendlich umgesetzt?

Die Teilgruppe Simulation hat ein Simulationstool erstellt, das als Webanwendung realisiert wurde. Das Tool erlaubt das Erstellen von Szenarien mit einer vom Nutzer festgelegten Anzahl von Fahrzeugen und Rampen. Auch kann ein Nutzer eine beliebige Anzahl an Aufträgen generieren, die als Input für das Simulieren eines Szenarios dienen. Die Aktionen der Akteure werden serverseitig durch das Multiagentenframework JADE realisiert und clientseitig dargestellt. Jeder Akteur wird durch mehrere Agenten realisiert, die Zielsuche und Suche eines Transportmittels autonom und ohne zentrale Steuerung durchführen. Die Agenten der Fahrzeuge sind in der Lage mit einem selbst erstellten Pathfinding-Algorithmus Routen zu berechnen und diese anschließend abzufahren. Für einen Simulationsdurchlauf werden Daten über Paketdurchlaufzeit und Auslastung der Roboter mitgeloggt, die in der Statistik eingesehen werden können.

Die Teilgruppe Materialfluss hat während der Projektarbeit ein lauffähiges System für den Materialfluss auf Agentenbasis fertiggestellt. Zunächst wurde dafür ein vollständiges Agenten-System auf den unterschiedlichen Modulen umgesetzt. Es erlaubt die parallele Arbeit von bis zu sieben Agenten pro \textsc{Mica}z-Modul und den Austausch von Agenten-Nachrichten sowohl innerhalb eines Moduls als auch modulübergreifend. So ermöglicht es die Überwachung und Lokalisierung aller Rampen und Pakete, die sich im System befinden. Ein am Eingang liegendes Paket kann durch das System bis zu einem gewünschten Ausgang geleitet werden. Dazu gehört zum einen, dass die Module untereinander über ihre Funkschnittstelle kommunizieren können. Es wurde außerdem die Kommunikation zwischen den \textsc{Mica}z-Modulen und Volksbots über ein definiertes Kommunikationsprotokoll erfolgreich umgesetzt. Ein weiteres Merkmal in der Funktionalität ist das gelungene Zusammenspiel zwischen Rampen und Modulen, also die Abfrage und die Ansteuerung von Sensorik und Aktorik. Dies erlaubt die Vereinzelung von Paketen durch gezieltes öffnen und schließen der Bolzen. Außerdem wird die Überwachung der einzelnen Rampenplätze durch die angebrachten Lichtschranken ermöglicht. Eine weitere erfolgreich umgesetzte Funktionalität, ist die Möglichkeit Pakete durch das System routen zu lassen. Dabei wird ein verteiltes Protokoll auf Basis von kommunizierenden Zustandsautomaten mit plattformübergreifender Kommunikation der Agenten eingesetzt.

Die Teilgruppe Fahrzeuge hat einen autonom agierenden Pakettransport auf Basis des Volksbot zwischen den Rampen realisiert. Die Funktionen auf dem Volksbot sind in drei Bereiche zu unterteilen. Zunächst wurde mithilfe des Laserscanner und AMCL für eine erfolgreiche Navigation und Lokalisation gesorgt. Als weiteres Funktionspaket konnte die Paketübergabe realisiert werden. Des Weiteren ist eine Kommunikationsschnittstelle mit dem Materialfluss durch den MICAz implementiert worden.
Insgesamt ist der Volksbot in der Lage sich autonom innerhalb der physischen Zelle zu navigieren und Aufträge entgegen zu nehmen. 

\subsection{Fazit}
Im Rahmen der Projektgruppe wurde von zwölf Teilnehmern ein System entwickelt, dass auf die drei Teilgruppen Materialfluss, Fahrzeuge und Simulation verteilt wurde. Durch die gemeinsame Arbeit an einem Projekt konnten Erfahrungen und Erkenntnisse auf dem Gebiet der Projektarbeit und Softwareentwicklung gewonnen werden. Ein wesentlicher Erfolgsfaktor eines Projekts ist das effektive Nutzen der vorhanden Zeit bei einem Projekt mit begrenzter Dauer. Grundsätzlich wurden Aufgaben sorgfältig und möglichst schnell durchgeführt. Für zukünftige Projekte empfiehlt es sich jedoch das Zeitmanagement zu optimieren, indem versucht wird Aufgaben noch weiter zu parallelisieren. Im Rahmen der Anforderungserhebung hatte sich bereits herausgestellt, dass bestimmte Technologien, wie beispielsweise ein Multiagentensystem zu einem späteren Zeitpunkt benötigt werden. Die Einarbeitung in benötigte Technologien erfolgte jeweils kurz vor der Implementierung einer Technologie, wodurch sich die Entwicklungszeit verlängerte. Um die Entwicklungszeit zu verkürzen, sollten Teilnehmer eines Projekts sich vor der Implementierung einer Komponente in die benötigte Technologie einarbeiten und diese für die anderen Teilnehmer aufbereiten. 
\\\\
Die Durchführung des Projekts mithilfe des Scrum Vorgehensmodell beinhaltete Sitzungen mit den Auftraggebern in denen die festgelegten Anforderungen mit den tatsächlich realisierten abgeglichen wurden. So konnten Eigenschaften des Produkts, die nicht den Vorstellungen der Auftraggeber entsprachen, direkt und mit geringerem Aufwand korrigiert werden. Ein solcher Abgleich sollte unabhängig vom dem gewählten Vorgehensmodell regelmäßiger Bestandteil eines Projekts sein. 
\\\\
Für die Implementierung der Agenten sowohl im physischen als auch im softwarebasierten System, wurden die Funktionalitäten, die ein Fahrzeug oder eine Rampe besitzen soll, auf mehrere Agenten verteilt. Ziel war es die Agentensysteme zu modularisieren. Modularisierung soll grundsätzlich die Wartbarkeit und Übersichtlichkeit eines Systems erhöhen. Jedoch hat sich gezeigt, dass eine Modularisierung auch Nachteile haben kann. Durch die Vielzahl der Agenten und die Vielzahl an Nachrichten, die ausgetauscht werden, wurde die Testbarkeit des Systems verringert, da auftretende Fehler schwer einzelnen Methoden der Agenten zugeordnet werden konnten. Für zukünftige Projekte sollte differenzierter betrachtet werden, ob eine Modularisierung unter Berücksichtigung der systemspezifischen Eigenschaften sinnvoll ist. 

\subsection{Vision}
%Todo Vision für Materialfluss und Fahrzeuge
Für die Simulationssoftware bieten sich folgende Optimierungsmöglichkeiten: Die Fahrzeuge fahren selbstständig zu einem Ziel ohne dabei Kollisionen mit anderen Fahrzeugen zu vermeiden. Die Fahrzeuge fahren durcheinander durch. Um Kollisionen vorzubeugen, könnten Pfade entweder reserviert werden oder Roboter könnten untereinander Nachrichten austauschen, um festzustellen, ob ein Feld frei ist oder nicht. Um den Kommunikationsaufwand zu verringern, empfiehlt es sich Pfade zu reservieren. Weiterhin sollte es die Möglichkeit geben Roboter zur Laufzeit hinzuzufügen, da ein physisches System auch dynamisch skaliert werden kann, indem ein weiterer Roboter hinzugefügt wird. Durch den Einbau der beschriebenen Features erweitern sich die Möglichkeiten des Tools und die Ergebnisse eines Durchlaufs sind aussagekräftiger.
\\\\
Für die Implementierung der Agenten wurde JADE ausgewählt. JADE ermöglicht es Agenten in Echtzeit miteinander kommunizieren zu lassen. Das ebenfalls javabasierte MASON Framework, das von der George Mason Universität aus den U.S.A entwickelt wurde, bietet keine Echtzeitkommunikation, sondern steuert alle Agenten mit einem Scheduler. Der Scheduler unterteilt eine Simulation in verschiedene diskrete Schritte und führt bei jedem Schritt alle Agenten aus, die an der Reihe sind. Ein solcher Scheduler bietet den Vorteil, dass Scheduling Probleme reduziert werden können. Jedoch stellt sich die Frage, ob man einen Simulationslauf weiterhin in Echtzeit laufen lassen kann bzw. Echtzeit simulieren kann, wenn man ein Framework nutzt, dass einen Scheduler beinhaltet. Für zukünftige Modifikationen stellt sich die Frage, welche Möglichkeiten sich durch die Nutzung eines Schedulers ergeben.
\\\\
Im Materialfluss sollten im nächsten Schritt auch Zwischenlager-Rampen in das Routing der Pakete einbezogen werden. So könnte etwa berechnet werden, ob es kostengünstiger ist, ein Paket vom Eingang zum Ausgang über eine Zwischenrampe zu liefern oder nicht. Außerdem können Pakete, denen noch kein Ziel im System zugewiesen wurde, hier zwischengelagert werden, um Ein- und Ausgangs-Rampen nicht unnötigerweise zu blockieren.

Das Routing kann noch weiter ausgebaut werden, indem Zeitslots und Reservierungen auf Plattformen eingeführt werden werden. So können die Durchlaufzeiten der Pakete und die Gefahr von Deadlocks verringert werden.

Es ist zudem möglich, den Materialfluss auf existierende Stetigförderer auszuweiten. Für diese muss das AgentenRTE teilweise angepasst und es müssen dedizierte Plattformagenten für jeden Stetigförderer entwickelt werden. So kann das Gesamtsystem deutlich erweitert werden: Durch die Möglichkeit der Stetigförderer, mehrere Pakete in unterschiedlichen Richtungen auszugeben, werden Durchsatz und Flexibilität deutlich erhöht.
\\\\
Bezogen auf das Gesamtsystem bietet es sich an das physische und das virtuelle Teilsystem mithilfe eines hybriden Modus miteinander zu verknüpfen. Ein solcher Hybridmodus bietet viele weitere Möglichkeiten und könnte auf unterschiedlichen Wegen realisiert werden. Beispielsweise könnte die Software das physische System steuern, indem generierte Aufträge an das reale System geschickt werden. Auch könnten die Aktionen der Fahrzeuge und Rampen in der Software visualisiert werden. Besteht ein Teil der Visualisierung aus der Darstellung der realen Akteure und der andere aus rein virtuellen Akteuren, so könnten physische und virtuelle Akteure ein Gesamtsystem bilden, dass die Skalierbarkeit des physischen Systems erhöht. Dies setzt jedoch voraus, dass die Akteure aus beiden Systemen in ihren Eigenschaften (Geschwindigkeit etc.) weitestgehend vollständig aneinander angeglichen sind. Die technischen Voraussetzungen für das einbringen und abhören von drahtlosen Nachrichten in der physischen Zelle sind bereits durch Gateways realisiert. Die einzelnen Module müssten jedoch zusätzlich regelmäßig über ihren Zustand informieren, um eine sinnvolle Auswertung der physischen Zelle in der Simulationsoberfläche zu erlauben.





	
	% Bibliography
	\clearpage
	\clearscrheadfoot
	\ohead[Inhaltsverzeichnis]{Inhaltsverzeichnis}
	\chead[Uni Oldenburg]{Uni Oldenburg}
	\ihead[Malte Falk]{Malte Falk}
	\ofoot[Seminararbeit]{Seminararbeit}
	\cfoot[\pagemark]{\pagemark}
	\ifoot[WS 2013/14]{WS 2013/14}
	\ohead[Literaturverzeichnis]{Literaturverzeichnis}
	\printbibliography

\end{document}

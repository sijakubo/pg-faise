%%
%% Template of the department Very Large Business Applications,
%%    CvO University Oldenburg for scientific papers
%%
%% Created by Dipl.-Inform. Daniel Süpke
%%    For questions, comments, suggestions etc. send an email to:
%%    suepke@wi-ol.de or suepke@gmx.de
%%
%% Version: April 16, 2010
%%
%% Note: Has only been tested with pdflatex, not latex (dvi). Still, there is
%% theoretical support also for latex.
%%


\documentclass[11pt]{scrartcl}

%% packages
%\usepackage[utf8]{inputenc}       % Standard for Linux
\usepackage[latin1]{inputenc}    % Standard for Windows
\usepackage{ngerman}              % For German language
\usepackage{fancyhdr}
\usepackage{geometry}
\usepackage{ifpdf}
\usepackage{setspace}             % For line spread



% For pdflatex
\ifpdf
  % One of these two:
  \usepackage[pdftex]{graphicx}
  %\usepackage[pdftex]{epsfig}

  \usepackage[pdftex]{hyperref}
% For latex (dvi)
\else
  % One of these two:
  \usepackage[dvips]{graphicx}
  %\usepackage[dvips]{epsfig}

  % make the command \href from hyperref available as a 'print only'
  \newcommand{\href}[2]{#2}
\fi


%% Picture options
\graphicspath{{pictures/}}         % Default path to pictures used
\DeclareGraphicsExtensions{.png}   % More extensions can be added


%% Pagestyle options
\pagestyle{fancy}
%\lhead{}
%\chead{}
%\rhead{}
%\lfoot{Daniel Süpke}
%\cfoot{}
%\rfoot{}
\renewcommand{\headrulewidth}{0.4pt}



\geometry{a4paper,left=3cm,right=3cm}
%\geometry{a4paper,left=3cm,right=2.5cm}   % Please use these settings for a PhD-thesis




%% Document start
\begin{document}

%% Title page
\begin{titlepage}
  \begin{centering}
  \begin{figure}[h!]
    \centering
    \includegraphics[width=20pt]{CvO-Oldenburg-Logo}    % Ggf. Copyright beachten - ansonsten nur für Gebrauch an der CvO
  \end{figure}

  \vspace*{0.8cm}

  \begin{figure}[h!]
    \centering
    \includegraphics[width=20pt]{faise_logo_gross}    % Ggf. Copyright beachten - ansonsten nur für Gebrauch an der CvO/VLBA
  \end{figure}

  \vspace*{0.4cm}
  
  \textsf{\Huge \textbf{Fully Autonomous Intralogistic Swarm Experiment (FAISE)}}

  \vspace*{0.5cm}
  %\noindent Referat / Diplomarbeit\\
  %\emph{Bei Referaten noch mit Zusatz:} im Rahmen des..     % Insert correct type
	\begin{table}[h]
\centering
  \begin{tabular}{ll}
   \textbf{\underline{Betreuer}} &  \\
   & Betreuer 1 \\
	 & Betreuer 2 \\
	 & Betreuer 3 \\ 
	 &  \\
	 \underline{Projektbeginn:}& 10.11.2013 \\
		\underline{Projektende:}& 30.11.2014 \\
		&  \\
	  \underline{Kontakt} &  \\
		& PLZ Ort\\
    & Telefonnummer\\
		& faise@uni-oldenburg.de\\
		& http://www.faise.uni-oldenburg.de
\end{tabular}
\end{table}
  \end{centering}
  
\end{titlepage}



%\thispagestyle{empty}
\newpage

\tableofcontents
\newpage


\section*{Glossar}             % Alternatively a glossary package can be used
\addcontentsline{toc}{section}{Glossar}
\section*{Symbolverzeichnis}   % If needed
\addcontentsline{toc}{section}{Symbolverzeichnis}
\newpage

\listoffigures
\addcontentsline{toc}{section}{Abbildungsverzeichnis}
\listoftables
\addcontentsline{toc}{section}{Tabellenverzeichnis}
\newpage

%% Line spread
\onehalfspacing

% Start of content
\include{./kapitel/Einleitung} 

\section{Kapitel 1}
TODO ...

\section{Kapitel 2}
TODO...

% Appendix
\begin{appendix}

\section{Anhang}
TODO ...

\begin{verbatim}
TODO...
\end{verbatim}

\newpage

\addcontentsline{toc}{section}{Literaturverzeichnis}
\bibliographystyle{alpha}
\bibliography{literatur} % Point to BibTeX literature file e.g. literatur.bib

\end{appendix}



\newpage
\section{Abschlie�ende Erkl\"arung}
TODO...


\vspace*{3cm}
\noindent Oldenburg, den \today \hspace*{2cm} 

\end{document}
